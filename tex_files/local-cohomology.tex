\input{preamble}

% OK, start here.
%
\begin{document}

\title{Local Cohomology}


\maketitle

\phantomsection
\label{section-phantom}

\tableofcontents

\section{Introduction}
\label{section-introduction}

\noindent
This chapter continues the study of local cohomology.
A reference is \cite{SGA2}.
The definition of local cohomology can be found in
Dualizing Complexes, Section \ref{dualizing-section-local-cohomology}.
For Noetherian rings taking local cohomology is the same
as deriving a suitable torsion functor as is shown in
Dualizing Complexes, Section
\ref{dualizing-section-local-cohomology-noetherian}.
The relationship with depth can be found in
Dualizing Complexes, Section
\ref{dualizing-section-depth}.

\medskip\noindent
In the first part of this chapter we discuss finiteness properties of
local cohomology leading to a proof of a fairly general version of
Grothendieck's finiteness theorem, see Theorem \ref{theorem-finiteness}
and Lemma \ref{lemma-finiteness-Rjstar} (higher direct images
of coherent modules under open immersions).
Our methods incorporate a few very slick arguments the reader
can find in papers of Faltings, see
\cite{Faltings-annulators} and \cite{Faltings-finiteness}.

\medskip\noindent
The second part of this chapter is devoted to theorems on
formal functions and algebraization of formal functions,
mainly in the local Noetherian case (we discuss
the global case elsewhere -- insert future reference here).
Section \ref{section-formal-functions-principal}
discusses some of the tricks one has in the case
of formal functions along an effective Cartier divisor
cut out by a global regular function.
Section \ref{section-derived-completion}
discusses derived completion with respect to
a finite type sheaf of ideals in complete generality.
We show how this material relates to the usual
theorem on formal functions in Section \ref{section-formal-functions}.
In Sections \ref{section-algebraization-sections} and
\ref{section-algebraization-modules}
we algebraize formal sections and coherent formal modules.




\section{Generalities}
\label{section-generalities}

\noindent
The following lemma tells us that the functor $R\Gamma_Z$
is related to cohomology with supports.

\begin{lemma}
\label{lemma-local-cohomology-is-local-cohomology}
Let $A$ be a ring and let $I$ be a finitely generated ideal.
Set $Z = V(I) \subset X = \Spec(A)$. For $K \in D(A)$ corresponding
to $\widetilde{K} \in D_\QCoh(\mathcal{O}_X)$ via
Derived Categories of Schemes, Lemma \ref{perfect-lemma-affine-compare-bounded}
there is a functorial isomorphism
$$
R\Gamma_Z(K) = R\Gamma_Z(X, \widetilde{K})
$$
where on the left we have
Dualizing Complexes, Equation (\ref{dualizing-equation-local-cohomology})
and on the right we have the functor of
Cohomology, Section \ref{cohomology-section-cohomology-support}.
\end{lemma}

\begin{proof}
By Cohomology, Section \ref{cohomology-section-cohomology-support}
there exists a distinguished triangle
$$
R\Gamma_Z(X, \widetilde{K})
\to R\Gamma(X, \widetilde{K})
\to R\Gamma(U, \widetilde{K})
\to R\Gamma_Z(X, \widetilde{K})[1]
$$
where $U = X \setminus Z$. We know that $R\Gamma(X, \widetilde{K}) = K$
by Derived Categories of Schemes, Lemma
\ref{perfect-lemma-affine-compare-bounded}.
Say $I = (f_1, \ldots, f_r)$. Then we obtain a finite affine
open covering $\mathcal{U} : U = D(f_1) \cup \ldots \cup D(f_r)$.
By Derived Categories of Schemes, Lemma
\ref{perfect-lemma-alternating-cech-complex-complex-computes-cohomology}
the alternating {\v C}ech complex
$\text{Tot}(\check{\mathcal{C}}_{alt}^\bullet(\mathcal{U},
\widetilde{K^\bullet}))$
computes $R\Gamma(U, \widetilde{K})$ where $K^\bullet$ is any
complex of $A$-modules representing $K$. Working through the
definitions we find
$$
R\Gamma(U, \widetilde{K}) =
\text{Tot}\left(
K^\bullet \otimes_A
(\prod\nolimits_{i_0} A_{f_{i_0}} \to
\prod\nolimits_{i_0 < i_1} A_{f_{i_0}f_{i_1}} \to
\ldots \to A_{f_1\ldots f_r})\right)
$$
It is clear that
$K^\bullet = R\Gamma(X, \widetilde{K^\bullet}) \to
R\Gamma(U, \widetilde{K}^\bullet)$
is induced by the diagonal map from $A$ into $\prod A_{f_i}$.
Hence we conclude that
$$
R\Gamma_Z(X, \mathcal{F}^\bullet) =
\text{Tot}\left(
K^\bullet \otimes_A
(A \to \prod\nolimits_{i_0} A_{f_{i_0}} \to
\prod\nolimits_{i_0 < i_1} A_{f_{i_0}f_{i_1}} \to
\ldots \to A_{f_1\ldots f_r})\right)
$$
By Dualizing Complexes, Lemma \ref{dualizing-lemma-local-cohomology-adjoint}
this complex computes $R\Gamma_Z(K)$ and we see the lemma holds.
\end{proof}

\begin{lemma}
\label{lemma-local-cohomology}
Let $A$ be a ring and let $I \subset A$ be a finitely generated ideal.
Set $X = \Spec(A)$, $Z = V(I)$, $U = X \setminus Z$, and $j : U \to X$
the inclusion morphism. Let $\mathcal{F}$ be a quasi-coherent
$\mathcal{O}_U$-module. Then
\begin{enumerate}
\item there exists an $A$-module $M$ such that $\mathcal{F}$ is the
restriction of $\widetilde{M}$ to $U$,
\item given $M$ there is an exact sequence
$$
0 \to H^0_Z(M) \to M \to H^0(U, \mathcal{F}) \to H^1_Z(M) \to 0
$$
and isomorphisms $H^p(U, \mathcal{F}) = H^{p + 1}_Z(M)$ for $p \geq 1$,
\item we may take $M = H^0(U, \mathcal{F})$ in which case
we have $H^0_Z(M) = H^1_Z(M) = 0$.
\end{enumerate}
\end{lemma}

\begin{proof}
The existence of $M$ follows from
Properties, Lemma \ref{properties-lemma-extend-trivial}
and the fact that quasi-coherent sheaves on $X$ correspond
to $A$-modules (Schemes, Lemma \ref{schemes-lemma-equivalence-quasi-coherent}).
Then we look at the distinguished triangle
$$
R\Gamma_Z(X, \widetilde{M}) \to R\Gamma(X, \widetilde{M}) \to
R\Gamma(U, \widetilde{M}|_U) \to R\Gamma_Z(X, \widetilde{M})[1]
$$
of Cohomology, Section \ref{cohomology-section-cohomology-support}.
Since $X$ is affine we have $R\Gamma(X, \widetilde{M}) = M$
by Cohomology of Schemes, Lemma
\ref{coherent-lemma-quasi-coherent-affine-cohomology-zero}.
By our choice of $M$ we have $\mathcal{F} = \widetilde{M}|_U$
and hence this produces an exact sequence
$$
0 \to H^0_Z(X, \widetilde{M}) \to M \to H^0(U, \mathcal{F}) \to
H^1_Z(X, \widetilde{M}) \to 0
$$
and isomorphisms $H^p(U, \mathcal{F}) = H^{p + 1}_Z(X, \widetilde{M})$
for $p \geq 1$. By Lemma \ref{lemma-local-cohomology-is-local-cohomology}
we have $H^i_Z(M) = H^i_Z(X, \widetilde{M})$ for all $i$.
Thus (1) and (2) do hold.
Finally, setting $M' = H^0(U, \mathcal{F})$ we see that
the kernel and cokernel of $M \to M'$ are $I$-power torsion.
Therefore $\widetilde{M}|_U \to \widetilde{M'}|_U$ is an isomorphism
and we can indeed use $M'$ as predicted in (3). It goes without saying
that we obtain zero for both $H^0_Z(M')$ and $H^0_Z(M')$.
\end{proof}

\begin{lemma}
\label{lemma-already-torsion}
Let $I, J \subset A$ be finitely generated ideals of a ring $A$.
If $M$ is an $I$-power torsion module, then the
canonical map
$$
H^i_{V(I) \cap V(J)}(M) \to H^i_{V(J)}(M)
$$
is an isomorphism for all $i$.
\end{lemma}

\begin{proof}
Use the spectral sequence of
Dualizing Complexes, Lemma \ref{dualizing-lemma-local-cohomology-ss}
to reduce to the statement $R\Gamma_I(M) = M$ which is immediate
from the construction of local cohomology
in Dualizing Complexes, Section \ref{dualizing-section-local-cohomology}.
\end{proof}

\begin{lemma}
\label{lemma-multiplicative}
Let $S \subset A$ be a multiplicative set of a ring $A$.
Let $M$ be an $A$-module with $S^{-1}M = 0$. Then
$\colim_{f \in S} H^0_{V(f)}(M) = M$ and
$\colim_{f \in S} H^1_{V(f)}(M) = 0$.
\end{lemma}

\begin{proof}
The statement on $H^0$ follows directly from the definitions.
To see the statement on $H^1$ observe that $R\Gamma_{V(f)}$
and $H^1_{V(f)}$ commute with colimits. Hence we may assume
$M$ is annihilated by some $f \in S$. Then
$H^1_{V(ff')}(M) = 0$ for all $f' \in S$ (for example by
Lemma \ref{lemma-already-torsion}).
\end{proof}

\begin{lemma}
\label{lemma-elements-come-from-bigger}
Let $I \subset A$ be a finitely generated ideal of a ring $A$.
Let $\mathfrak p$ be a prime ideal. Let $M$ be an $A$-module.
Let $i \geq 0$ be an integer and consider the map
$$
\Psi :
\colim_{f \in A, f \not \in \mathfrak p} H^i_{V((I, f))}(M)
\longrightarrow
H^i_{V(I)}(M)
$$
Then
\begin{enumerate}
\item $\Im(\Psi)$ is the set of elements which map to zero in
$H^i_{V(I)}(M)_\mathfrak p$,
\item if $H^{i - 1}_{V(I)}(M)_\mathfrak p = 0$, then $\Psi$ is injective,
\item if $H^{i - 1}_{V(I)}(M)_\mathfrak p = H^i_{V(I)}(M)_\mathfrak p = 0$,
then $\Psi$ is an isomorphism.
\end{enumerate}
\end{lemma}

\begin{proof}
For $f \in A$, $f \not \in \mathfrak p$ the spectral sequence of
Dualizing Complexes, Lemma \ref{dualizing-lemma-local-cohomology-ss}
degenerates to give short exact sequences
$$
0 \to H^1_{V(f)}(H^{i - 1}_{V(I)}(M)) \to
H^i_{V((I, f))}(M) \to H^0_{V(f)}(H^i_{V(I)}(M)) \to 0
$$
This proves (1) and part (2) follows from this and
Lemma \ref{lemma-multiplicative}.
Part (3) is a formal consequence.
\end{proof}

\begin{lemma}
\label{lemma-isomorphism}
Let $I \subset I' \subset A$ be finitely generated ideals of a
Noetherian ring $A$. Let $M$ be an $A$-module. Let $i \geq 0$ be an integer.
Consider the map
$$
\Psi : H^i_{V(I')}(M) \to H^i_{V(I)}(M)
$$
The following are true:
\begin{enumerate}
\item if $H^i_{\mathfrak pA_\mathfrak p}(M_\mathfrak p) = 0$
for all $\mathfrak p \in V(I) \setminus V(I')$, then
$\Psi$ is surjective,
\item if $H^{i - 1}_{\mathfrak pA_\mathfrak p}(M_\mathfrak p) = 0$
for all $\mathfrak p \in V(I) \setminus V(I')$, then
$\Psi$ is injective,
\item if $H^i_{\mathfrak pA_\mathfrak p}(M_\mathfrak p) =
H^{i - 1}_{\mathfrak pA_\mathfrak p}(M_\mathfrak p) = 0$
for all $\mathfrak p \in V(I) \setminus V(I')$, then
$\Psi$ is an isomorphism.
\end{enumerate}
\end{lemma}

\begin{proof}
Proof of (1).
Let $\xi \in H^i_{V(I)}(M)$. Since $A$ is Noetherian, there exists a
largest ideal $I \subset I'' \subset I'$ such that $\xi$ is the image
of some $\xi'' \in H^i_{V(I'')}(M)$. If $V(I'') = V(I')$, then we are
done. If not, choose a generic point $\mathfrak p \in V(I'')$ not in $V(I')$.
Then we have $H^i_{V(I'')}(M)_\mathfrak p =
H^i_{\mathfrak pA_\mathfrak p}(M_\mathfrak p) = 0$ by assumption.
By Lemma \ref{lemma-elements-come-from-bigger} we can increase $I''$
which contradicts maximality.

\medskip\noindent
Proof of (2). Let $\xi' \in H^i_{V(I')}(M)$ be in the kernel of $\Psi$.
Since $A$ is Noetherian, there exists a
largest ideal $I \subset I'' \subset I'$ such that $\xi'$
maps to zero in $H^i_{V(I'')}(M)$. If $V(I'') = V(I')$, then we are
done. If not, then choose a generic point $\mathfrak p  \in V(I'')$
not in $V(I')$. Then we have $H^{i - 1}_{V(I'')}(M)_\mathfrak p =
H^{i - 1}_{\mathfrak pA_\mathfrak p}(M_\mathfrak p) = 0$ by assumption.
By Lemma \ref{lemma-elements-come-from-bigger} we can increase $I''$
which contradicts maximality.

\medskip\noindent
Part (3) is formal from parts (1) and (2).
\end{proof}





\section{Finiteness of local cohomology, I}
\label{section-finiteness}

\noindent
We will follow Faltings approach to finiteness of local cohomology
modules, see \cite{Faltings-annulators} and \cite{Faltings-finiteness}.
Here is a lemma which shows that it suffices to prove
local cohomology modules have an annihilator in order to prove that
they are finite modules.

\begin{lemma}
\label{lemma-check-finiteness-local-cohomology-by-annihilator}
\begin{reference}
This is a special case of \cite[Lemma 3]{Faltings-annulators}.
\end{reference}
Let $A$ be a Noetherian ring, $I \subset A$ an ideal, $M$ a finite
$A$-module, and $n \geq 0$ an integer. Let $Z = V(I)$.
The following are equivalent
\begin{enumerate}
\item $H^i_Z(M)$ is finite for $i \leq n$,
\item there exists an $e \geq 0$ such that $I^e$ annihilates
$H^i_Z(M)$ for $i \leq n$, and
\item there exists an ideal $J \subset A$ with $V(J) \subset Z$
such that $J$ annihilates $H^i_Z(M)$ for $i \leq n$.
\end{enumerate}
\end{lemma}

\begin{proof}
We prove the lemma by induction on $n$. For $n = 0$ we have
$H^0_Z(M) \subset M$ is finite, hence (1), (2), and (3) are true.
Assume that $n > 0$.

\medskip\noindent
If (1) is true, then, since $H^i_Z(M) = H^i_I(M)$
(Dualizing Complexes, Lemma \ref{dualizing-lemma-local-cohomology-noetherian})
is $I$-power torsion, we see that (2) holds.
It is clear that (2) implies (3).

\medskip\noindent
Assume (3) is true. Let $N = H^0_Z(M)$ and $M' = M/N$.
By Dualizing Complexes, Lemma \ref{dualizing-lemma-divide-by-torsion}
we may replace $M$ by $M'$.
Thus we may assume that $H^0_Z(M) = 0$.
This means that $\text{depth}_I(M) > 0$
(Dualizing Complexes, Lemma \ref{dualizing-lemma-depth}).
Pick $f \in I$ a nonzerodivisor on $M$. After raising $f$ to a suitable
power, we may assume $f \in J$ as $V(J) \subset V(I)$. Then the
long exact local cohomology sequence associated to the short
exact sequence
$$
0 \to M \to M \to M/fM \to 0
$$
turns into short exact sequences
$$
0 \to H^i_Z(M) \to H^i_Z(M/fM) \to H^{i + 1}_Z(M) \to 0
$$
for $i < n$. We conclude that $J^2$ annihilates $H^i_Z(M/fM)$
for $i < n$. By induction hypothesis we see that $H^i_Z(M/fM)$
is finite for $i < n$. Using the short exact sequence once more
we see that $H^{i + 1}_Z(M)$ is finite for $i < n$ as desired.
\end{proof}

\noindent
The following result of Faltings allows us to prove finiteness
of local cohomology at the level of local rings.

\begin{lemma}
\label{lemma-check-finiteness-local-cohomology-locally}
\begin{reference}
This is a special case of \cite[Satz 1]{Faltings-finiteness}.
\end{reference}
Let $A$ be a Noetherian ring, $I \subset A$ an ideal, $M$ a finite
$A$-module, and $n \geq 0$ an integer. Let $Z = V(I)$.
The following are equivalent
\begin{enumerate}
\item the modules $H^i_Z(M)$ are finite for $i \leq n$, and
\item for all $\mathfrak p \in \Spec(A)$ the modules
$H^i_Z(M)_\mathfrak p$, $i \leq n$ are finite $A_\mathfrak p$-modules.
\end{enumerate}
\end{lemma}

\begin{proof}
The implication (1) $\Rightarrow$ (2) is immediate. We prove the converse
by induction on $n$. The case $n = 0$ is clear because both (1) and
(2) are always true in that case.

\medskip\noindent
Assume $n > 0$ and that (2) is true. Let $N = H^0_Z(M)$ and $M' = M/N$.
By Dualizing Complexes, Lemma \ref{dualizing-lemma-divide-by-torsion}
we may replace $M$ by $M'$.
Thus we may assume that $H^0_Z(M) = 0$.
This means that $\text{depth}_I(M) > 0$
(Dualizing Complexes, Lemma \ref{dualizing-lemma-depth}).
Pick $f \in I$ a nonzerodivisor on $M$ and consider the short
exact sequence
$$
0 \to M \to M \to M/fM \to 0
$$
which produces a long exact sequence
$$
0 \to H^0_Z(M/fM) \to H^1_Z(M) \to H^1_Z(M) \to H^1_Z(M/fM) \to
H^2_Z(M) \to \ldots
$$
and similarly after localization. Thus assumption (2) implies that
the modules $H^i_Z(M/fM)_\mathfrak p$ are finite for $i < n$. Hence
by induction assumption $H^i_Z(M/fM)$ are finite for $i < n$.

\medskip\noindent
Let $\mathfrak p$ be a prime of $A$ which is associated to
$H^i_Z(M)$ for some $i \leq n$. Say $\mathfrak p$ is the annihilator
of the element $x \in H^i_Z(M)$. Then $\mathfrak p \in Z$, hence
$f \in \mathfrak p$. Thus $fx = 0$ and hence $x$ comes from an
element of $H^{i - 1}_Z(M/fM)$ by the boundary map $\delta$ in the long
exact sequence above. It follows that $\mathfrak p$ is an associated
prime of the finite module $\Im(\delta)$. We conclude that
$\text{Ass}(H^i_Z(M))$ is finite for $i \leq n$, see
Algebra, Lemma \ref{algebra-lemma-finite-ass}.

\medskip\noindent
Recall that
$$
H^i_Z(M) \subset
\prod\nolimits_{\mathfrak p \in \text{Ass}(H^i_Z(M))}
H^i_Z(M)_\mathfrak p
$$
by Algebra, Lemma \ref{algebra-lemma-zero-at-ass-zero}. Since by
assumption the modules on the right hand side are finite and $I$-power
torsion, we can find integers $e_{\mathfrak p, i} \geq 0$, $i \leq n$,
$\mathfrak p \in \text{Ass}(H^i_Z(M))$ such that
$I^{e_{\mathfrak p, i}}$ annihilates $H^i_Z(M)_\mathfrak p$. We conclude
that $I^e$ with $e = \max\{e_{\mathfrak p, i}\}$ annihilates $H^i_Z(M)$
for $i \leq n$. By
Lemma \ref{lemma-check-finiteness-local-cohomology-by-annihilator}
we see that $H^i_Z(M)$ is finite for $i \leq n$.
\end{proof}

\begin{lemma}
\label{lemma-annihilate-local-cohomology}
Let $A$ be a ring and let $J \subset I \subset A$ be finitely generated ideals.
Let $i \geq 0$ be an integer. Set $Z = V(I)$. If
$H^i_Z(A)$ is annihilated by $J^n$ for some $n$, then
$H^i_Z(M)$ annihilated by $J^m$ for some $m = m(M)$
for every finitely presented $A$-module $M$ such that
$M_f$ is a finite locally free $A_f$-module for all $f \in I$.
\end{lemma}

\begin{proof}
Consider the annihilator $\mathfrak a$ of $H^i_Z(M)$.
Let $\mathfrak p \subset A$ with $\mathfrak p \not \in Z$.
By assumption there exists an $f \in I$, $f \not \in \mathfrak p$
and an isomorphism $\varphi : A_f^{\oplus r} \to M_f$
of $A_f$-modules. Clearing denominators (and using that
$M$ is of finite presentation) we find maps
$$
a : A^{\oplus r} \longrightarrow M
\quad\text{and}\quad
b : M \longrightarrow A^{\oplus r}
$$
with $a_f = f^N \varphi$ and $b_f = f^N \varphi^{-1}$ for some $N$.
Moreover we may assume that $a \circ b$ and $b \circ a$ are equal to
multiplication by $f^{2N}$. Thus we see that $H^i_Z(M)$ is annihilated by
$f^{2N}J^n$, i.e., $f^{2N}J^n \subset \mathfrak a$.

\medskip\noindent
As $U = \Spec(A) \setminus Z$ is quasi-compact we can find finitely many
$f_1, \ldots, f_t$ and $N_1, \ldots, N_t$ such that $U = \bigcup D(f_j)$ and
$f_j^{2N_j}J^n \subset \mathfrak a$. Then $V(I) = V(f_1, \ldots, f_t)$
and since $I$ is finitely generated we conclude
$I^M \subset (f_1, \ldots, f_t)$ for some $M$.
All in all we see that $J^m \subset \mathfrak a$ for
$m \gg 0$, for example $m = M (2N_1 + \ldots + 2N_t) n$ will do.
\end{proof}

\begin{lemma}
\label{lemma-local-finiteness-for-finite-locally-free}
Let $A$ be a Noetherian ring. Let $I \subset A$ be an ideal. Set $Z = V(I)$.
Let $n \geq 0$ be an integer. If $H^i_Z(A)$ is finite for $0 \leq i \leq n$,
then the same is true for $H^i_Z(M)$, $0 \leq i \leq n$ for
any finite $A$-module $M$ such that $M_f$ is a finite locally free
$A_f$-module for all $f \in I$.
\end{lemma}

\begin{proof}
The assumption that $H^i_Z(A)$ is finite for $0 \leq i \leq n$
implies there exists an $e \geq 0$ such that $I^e$ annihilates
$H^i_Z(A)$ for $0 \leq i \leq n$, see
Lemma \ref{lemma-check-finiteness-local-cohomology-by-annihilator}.
Then Lemma \ref{lemma-annihilate-local-cohomology}
implies that $H^i_Z(M)$, $0 \leq i \leq n$ is annihilated
by $I^m$ for some $m = m(M, i)$. We may take the same $m$
for all $0 \leq i \leq n$. Then
Lemma \ref{lemma-check-finiteness-local-cohomology-by-annihilator}
implies that $H^i_Z(M)$ is finite for $0 \leq i \leq n$
as desired.
\end{proof}





\section{Finiteness of pushforwards, I}
\label{section-finiteness-pushforward}

\noindent
In this section we discuss the easiest nontrivial case of the
finiteness theorem, namely, the finiteness of the first local
cohomology or what is equivalent, finiteness of $j_*\mathcal{F}$
where $j : U \to X$ is an open immersion, $X$ is locally Noetherian, and
$\mathcal{F}$ is a coherent sheaf on $U$. Following a method of Koll\'ar
(\cite{Kollar-variants} and \cite{Kollar-local-global-hulls})
we find a necessary and sufficient condition, see
Proposition \ref{proposition-kollar}. The reader who is interested
in higher direct images or higher local cohomology groups should skip
ahead to Section \ref{section-finiteness-pushforward-II} or
Section \ref{section-finiteness-II} (which are developed
independently of the rest of this section).

\begin{lemma}
\label{lemma-check-finiteness-pushforward-on-associated-points}
Let $X$ be a locally Noetherian scheme. Let $j : U \to X$ be the inclusion
of an open subscheme with complement $Z$. For $x \in U$ let
$i_x : W_x \to U$ be the integral closed subscheme with generic point $x$.
Let $\mathcal{F}$ be a coherent $\mathcal{O}_U$-module.
The following are equivalent
\begin{enumerate}
\item for all $x \in \text{Ass}(\mathcal{F})$ the
$\mathcal{O}_X$-module $j_*i_{x, *}\mathcal{O}_{W_x}$ is coherent,
\item $j_*\mathcal{F}$ is coherent.
\end{enumerate}
\end{lemma}

\begin{proof}
We first prove that (1) implies (2). Assume (1) holds.
The statement is local on $X$, hence we may assume $X$ is affine.
Then $U$ is quasi-compact, hence $\text{Ass}(\mathcal{F})$ is finite
(Divisors, Lemma \ref{divisors-lemma-finite-ass}). Thus we may argue by
induction on the number of associated points. Let $x \in U$ be a generic
point of an irreducible component of the support of $\mathcal{F}$.
By Divisors, Lemma \ref{divisors-lemma-finite-ass} we have
$x \in \text{Ass}(\mathcal{F})$. By our choice of $x$ we have
$\dim(\mathcal{F}_x) = 0$ as $\mathcal{O}_{X, x}$-module.
Hence $\mathcal{F}_x$ has finite length as an $\mathcal{O}_{X, x}$-module
(Algebra, Lemma \ref{algebra-lemma-support-point}).
Thus we may use induction on this length.

\medskip\noindent
Set $\mathcal{G} = j_*i_{x, *}\mathcal{O}_{W_x}$. This is a coherent
$\mathcal{O}_X$-module by assumption. We have $\mathcal{G}_x = \kappa(x)$.
Choose a nonzero map
$\varphi_x : \mathcal{F}_x \to \kappa(x) = \mathcal{G}_x$.
By Cohomology of Schemes, Lemma \ref{coherent-lemma-map-stalks-local-map}
there is an open $x \in V \subset U$ and a map
$\varphi_V : \mathcal{F}|_V \to \mathcal{G}|_V$ whose stalk
at $x$ is $\varphi_x$. Choose $f \in \Gamma(X, \mathcal{O}_X)$
which does not vanish at $x$ such that $D(f) \subset V$. By
Cohomology of Schemes, Lemma \ref{coherent-lemma-homs-over-open}
(for example) we see that $\varphi_V$ extends to
$f^n\mathcal{F} \to \mathcal{G}|_U$ for some $n$.
Precomposing with multiplication by $f^n$ we obtain a map
$\mathcal{F} \to \mathcal{G}|_U$ whose stalk at $x$ is nonzero.
Let $\mathcal{F}' \subset \mathcal{F}$ be the kernel.
Note that $\text{Ass}(\mathcal{F}') \subset \text{Ass}(\mathcal{F})$, see
Divisors, Lemma \ref{divisors-lemma-ses-ass}.
Since
$\text{length}_{\mathcal{O}_{X, x}}(\mathcal{F}') = 
\text{length}_{\mathcal{O}_{X, x}}(\mathcal{F}) - 1$
we may apply the
induction hypothesis to conclude $j_*\mathcal{F}'$ is coherent.
Since $\mathcal{G} = j_*(\mathcal{G}|_U) = j_*i_{x, *}\mathcal{O}_{W_x}$
is coherent, we can consider the exact sequence
$$
0 \to j_*\mathcal{F}' \to j_*\mathcal{F} \to \mathcal{G}
$$
By Schemes, Lemma \ref{schemes-lemma-push-forward-quasi-coherent}
the sheaf $j_*\mathcal{F}$ is quasi-coherent.
Hence the image of $j_*\mathcal{F}$ in $j_*(\mathcal{G}|_U)$
is coherent by Cohomology of Schemes, Lemma
\ref{coherent-lemma-coherent-Noetherian-quasi-coherent-sub-quotient}.
Finally, $j_*\mathcal{F}$ is coherent by
Cohomology of Schemes, Lemma \ref{coherent-lemma-coherent-abelian-Noetherian}.

\medskip\noindent
Assume (2) holds. Exactly in the same manner as above we reduce
to the case $X$ affine. We pick $x \in \text{Ass}(\mathcal{F})$
and we set $\mathcal{G} = j_*i_{x, *}\mathcal{O}_{W_x}$.
Then we choose a nonzero map
$\varphi_x : \mathcal{G}_x = \kappa(x) \to \mathcal{F}_x$
which exists exactly because $x$ is an associated point of $\mathcal{F}$.
Arguing exactly as above we may assume $\varphi_x$
extends to an $\mathcal{O}_U$-module map
$\varphi : \mathcal{G}|_U \to \mathcal{F}$.
Then $\varphi$ is injective (for example by
Divisors, Lemma \ref{divisors-lemma-check-injective-on-ass})
and we find and injective map
$\mathcal{G} = j_*(\mathcal{G}|_V \to j_*\mathcal{F}$.
Thus (1) holds.
\end{proof}

\begin{lemma}
\label{lemma-finiteness-pushforwards-and-H1-local}
Let $A$ be a Noetherian ring and let $I \subset A$ be an ideal.
Set $X = \Spec(A)$, $Z = V(I)$, $U = X \setminus Z$, and $j : U \to X$
the inclusion morphism. Let $\mathcal{F}$ be a coherent $\mathcal{O}_U$-module.
Then
\begin{enumerate}
\item there exists a finite $A$-module $M$ such that $\mathcal{F}$ is the
restriction of $\widetilde{M}$ to $U$,
\item given $M$ there is an exact sequence
$$
0 \to H^0_Z(M) \to M \to H^0(U, \mathcal{F}) \to H^1_Z(M) \to 0
$$
and isomorphisms $H^p(U, \mathcal{F}) = H^{p + 1}_Z(M)$ for $p \geq 1$,
\item given $M$ and $p \geq 0$ the following are equivalent
\begin{enumerate}
\item $R^pj_*\mathcal{F}$ is coherent,
\item $H^p(U, \mathcal{F})$ is a finite $A$-module,
\item $H^{p + 1}_Z(M)$ is a finite $A$-module,
\end{enumerate}
\item if the equivalent conditions in (3) hold for $p = 0$, we may take
$M = \Gamma(U, \mathcal{F})$ in which case we have $H^0_Z(M) = H^1_Z(M) = 0$.
\end{enumerate}
\end{lemma}

\begin{proof}
By Properties, Lemma \ref{properties-lemma-extend-finite-presentation}
there exists a coherent $\mathcal{O}_X$-module $\mathcal{F}'$
whose restriction to $U$ is isomorphic to $\mathcal{F}$.
Say $\mathcal{F}'$ corresponds to the finite $A$-module $M$
as in (1).
Note that $R^pj_*\mathcal{F}$ is quasi-coherent
(Cohomology of Schemes, Lemma
\ref{coherent-lemma-quasi-coherence-higher-direct-images})
and corresponds to the $A$-module $H^p(U, \mathcal{F})$.
By Lemma \ref{lemma-local-cohomology-is-local-cohomology}
and the general facts in
Cohomology, Section \ref{cohomology-section-cohomology-support}
we obtain an exact sequence
$$
0 \to H^0_Z(M) \to M \to H^0(U, \mathcal{F}) \to H^1_Z(M) \to 0
$$
and isomorphisms $H^p(U, \mathcal{F}) = H^{p + 1}_Z(M)$ for $p \geq 1$.
Here we use that $H^j(X, \mathcal{F}') = 0$ for $j > 0$ as $X$ is affine
and $\mathcal{F}'$ is quasi-coherent (Cohomology of Schemes,
Lemma \ref{coherent-lemma-quasi-coherent-affine-cohomology-zero}).
This proves (2).
Parts (3) and (4) are straightforward from (2); see also
Lemma \ref{lemma-local-cohomology}.
\end{proof}

\begin{lemma}
\label{lemma-finiteness-pushforward}
Let $X$ be a locally Noetherian scheme.
Let $j : U \to X$ be the inclusion of an
open subscheme with complement $Z$. Let $\mathcal{F}$ be a coherent
$\mathcal{O}_U$-module. Assume
\begin{enumerate}
\item $X$ is Nagata,
\item $X$ is universally catenary, and
\item for $x \in \text{Ass}(\mathcal{F})$ and
$z \in Z \cap \overline{\{x\}}$ we have
$\dim(\mathcal{O}_{\overline{\{x\}}, z}) \geq 2$.
\end{enumerate}
Then $j_*\mathcal{F}$ is coherent.
\end{lemma}

\begin{proof}
By Lemma \ref{lemma-check-finiteness-pushforward-on-associated-points}
it suffices to prove $j_*i_{x, *}\mathcal{O}_{W_x}$ is coherent
for $x \in \text{Ass}(\mathcal{F})$.
Let $\pi : Y \to X$ be the normalization of $X$ in $\Spec(\kappa(x))$, see
Morphisms, Section \ref{morphisms-section-normalization}. By
Morphisms, Lemma \ref{morphisms-lemma-nagata-normalization-finite-general}
the morphism $\pi$ is finite. Since $\pi$ is finite
$\mathcal{G} = \pi_*\mathcal{O}_Y$ is a coherent $\mathcal{O}_X$-module by
Cohomology of Schemes, Lemma \ref{coherent-lemma-finite-pushforward-coherent}.
Observe that $W_x = U \cap \pi(Y)$. Thus
$\pi|_{\pi^{-1}(U)} : \pi^{-1}(U) \to U$ factors through $i_x : W_x \to U$
and we obtain a canonical map
$$
i_{x, *}\mathcal{O}_{W_x}
\longrightarrow
(\pi|_{\pi^{-1}(U)})_*(\mathcal{O}_{\pi^{-1}(U)}) =
(\pi_*\mathcal{O}_Y)|_U = \mathcal{G}|_U
$$
This map is injective (for example by Divisors, Lemma
\ref{divisors-lemma-check-injective-on-ass}). Hence
$j_*i_{x, *}\mathcal{O}_{W_x} \subset j_*\mathcal{G}|_U$
and it suffices to show that $j_*\mathcal{G}|_U$ is coherent.

\medskip\noindent
It remains to prove that $j_*(\mathcal{G}|_U)$ is coherent. We claim
Divisors, Lemma \ref{divisors-lemma-check-isomorphism-via-depth-and-ass}
applies to
$$
\mathcal{G} \longrightarrow j_*(\mathcal{G}|_U)
$$
which finishes the proof.
Let $z \in X$. If $z \in U$, then the map is an isomorphism
on stalks as $j_*(\mathcal{G}|_U)|_U = \mathcal{G}|_U$.
If $z \in Z$, then $z \not \in \text{Ass}(j_*(\mathcal{G}|_U))$
(Divisors, Lemmas \ref{divisors-lemma-weakass-pushforward} and
\ref{divisors-lemma-weakly-ass-support}).
Thus it suffices to show that $\text{depth}(\mathcal{G}_z) \geq 2$.
Let $y_1, \ldots, y_n \in Y$ be the points mapping to $z$.
By Algebra, Lemma \ref{algebra-lemma-depth-goes-down-finite}
it suffices to show that
$\text{depth}(\mathcal{O}_{Y, y_i}) \geq 2$ for $i = 1, \ldots, n$.
If not, then by Properties, Lemma \ref{properties-lemma-criterion-normal}
we see that $\dim(\mathcal{O}_{Y, y_i}) = 1$ for some $i$.
This is impossible by the dimension formula
(Morphisms, Lemma \ref{morphisms-lemma-dimension-formula})
for $\pi : Y \to \overline{\{x\}}$ and assumption (3).
\end{proof}

\begin{lemma}
\label{lemma-sharp-finiteness-pushforward}
Let $X$ be an integral locally Noetherian scheme. Let $j : U \to X$
be the inclusion of a nonempty open subscheme with complement $Z$. Assume
that for all $z \in Z$ and any associated prime $\mathfrak p$ of
the completion $\mathcal{O}_{X, z}^\wedge$
we have $\dim(\mathcal{O}_{X, z}^\wedge/\mathfrak p) \geq 2$.
Then $j_*\mathcal{O}_U$ is coherent.
\end{lemma}

\begin{proof}
We may assume $X$ is affine.
Using Lemmas \ref{lemma-check-finiteness-local-cohomology-locally} and
\ref{lemma-finiteness-pushforwards-and-H1-local} we reduce to
$X = \Spec(A)$ where $(A, \mathfrak m)$ is a Noetherian local domain
and $\mathfrak m \in Z$.
Then we can use induction on $d = \dim(A)$.
(The base case is $d = 0, 1$ which do not happen by
our assumption on the local rings.)
Set $V = \Spec(A) \setminus \{\mathfrak m\}$.
Observe that the local rings of $V$ have dimension strictly smaller than $d$.
Repeating the arguments for $j' : U \to V$ we
and using induction we conclude that $j'_*\mathcal{O}_U$ is
a coherent $\mathcal{O}_V$-module.
Pick a nonzero $f \in A$ which vanishes on $Z$.
Since $D(f) \cap V \subset U$ we find an $n$ such that
multiplication by $f^n$ on $U$ extends to a map
$f^n : j'_*\mathcal{O}_U \to \mathcal{O}_V$ over $V$
(for example by Cohomology of Schemes, Lemma
\ref{coherent-lemma-homs-over-open}). This map is injective
hence there is an injective map
$$
j_*\mathcal{O}_U = j''_* j'_* \mathcal{O}_U \to j''_*\mathcal{O}_V
$$
on $X$ where $j'' : V \to X$ is the inclusion morphism.
Hence it suffices to show that $j''_*\mathcal{O}_V$ is coherent.
In other words, we may assume that $X$ is the spectrum
of a local Noetherian domain and that $Z$
consists of the closed point.

\medskip\noindent
Assume $X = \Spec(A)$ with $(A, \mathfrak m)$ local and $Z = \{\mathfrak m\}$.
Let $A^\wedge$ be the completion of $A$.
Set $X^\wedge = \Spec(A^\wedge)$, $Z^\wedge = \{\mathfrak m^\wedge\}$,
$U^\wedge = X^\wedge \setminus Z^\wedge$, and
$\mathcal{F}^\wedge = \mathcal{O}_{U^\wedge}$.
The ring $A^\wedge$ is universally catenary and Nagata (Algebra, Remark
\ref{algebra-remark-Noetherian-complete-local-ring-universally-catenary} and
Lemma \ref{algebra-lemma-Noetherian-complete-local-Nagata}).
Moreover, condition (3) of Lemma \ref{lemma-finiteness-pushforward}
for $X^\wedge, Z^\wedge, U^\wedge, \mathcal{F}^\wedge$
holds by assumption! Thus we see that
$(U^\wedge \to X^\wedge)_*\mathcal{O}_{U^\wedge}$
is coherent. Since the morphism $c : X^\wedge \to X$
is flat we conclude that the pullback of $j_*\mathcal{O}_U$ is
$(U^\wedge \to X^\wedge)_*\mathcal{O}_{U^\wedge}$
(Cohomology of Schemes, Lemma
\ref{coherent-lemma-flat-base-change-cohomology}).
Finally, since $c$ is faithfully flat we conclude that
$j_*\mathcal{O}_U$ is coherent by
Descent, Lemma \ref{descent-lemma-finite-type-descends}.
\end{proof}

\begin{remark}
\label{remark-closure}
Let $j : U \to X$ be an open immersion of locally Noetherian schemes.
Let $x \in U$. Let $i_x : W_x \to U$ be the integral closed subscheme
with generic point $x$ and let $\overline{\{x\}}$ be the closure in $X$.
Then we have a commutative diagram
$$
\xymatrix{
W_x \ar[d]_{i_x} \ar[r]_{j'} & \overline{\{x\}} \ar[d]^i \\
U \ar[r]^j & X
}
$$
We have $j_*i_{x, *}\mathcal{O}_{W_x} = i_*j'_*\mathcal{O}_{W_x}$.
As the left vertical arrow is a closed immersion we see that
$j_*i_{x, *}\mathcal{O}_{W_x}$ is coherent if and only of
$j'_*\mathcal{O}_{W_x}$ is coherent.
\end{remark}

\begin{remark}
\label{remark-no-finiteness-pushforward}
Let $X$ be a locally Noetherian scheme. Let $j : U \to X$ be the inclusion of
an open subscheme with complement $Z$. Let $\mathcal{F}$ be a coherent
$\mathcal{O}_U$-module. If there exists an $x \in \text{Ass}(\mathcal{F})$ and
$z \in Z \cap \overline{\{x\}}$ such that
$\dim(\mathcal{O}_{\overline{\{x\}}, z}) \leq 1$, then $j_*\mathcal{F}$ is not
coherent. To prove this we can do a flat base change to the spectrum
of $\mathcal{O}_{X, z}$. Let $X' = \overline{\{x\}}$.
The assumption implies $\mathcal{O}_{X' \cap U} \subset \mathcal{F}$.
Thus it suffices to see that $j_*\mathcal{O}_{X' \cap U}$ is not
coherent. This is clear because $X' = \{x, z\}$, hence
$j_*\mathcal{O}_{X' \cap U}$ corresponds to $\kappa(x)$ as an
$\mathcal{O}_{X, z}$-module which cannot be finite as $x$ is not
a closed point.

\medskip\noindent
In fact, the converse of Lemma \ref{lemma-sharp-finiteness-pushforward}
holds true: given an open immersion $j : U \to X$ of integral Noetherian
schemes and there exists a $z \in X \setminus U$ and an associated prime
$\mathfrak p$ of the completion $\mathcal{O}_{X, z}^\wedge$
with $\dim(\mathcal{O}_{X, z}^\wedge/\mathfrak p) = 1$,
then $j_*\mathcal{O}_U$ is not coherent. Namely, you can pass to
the local ring, you can enlarge $U$ to the punctured spectrum,
you can pass to the completion, and then the argument above gives
the nonfiniteness.
\end{remark}

\begin{proposition}[Koll\'ar]
\label{proposition-kollar}
\begin{reference}
Theorem of Koll\'ar stated in an email dated Wed, 1 Jul 2015.
\end{reference}
\begin{slogan}
Weak analogue of Hartogs' Theorem: On Noetherian schemes, the
restriction of a coherent sheaf to an open set with complement
of codimension 2 in the sheaf's support, is coherent.
\end{slogan}
Let $j : U \to X$ be an open immersion of locally Noetherian schemes
with complement $Z$. Let $\mathcal{F}$ be a coherent $\mathcal{O}_U$-module.
The following are equivalent
\begin{enumerate}
\item $j_*\mathcal{F}$ is coherent,
\item for $x \in \text{Ass}(\mathcal{F})$ and
$z \in Z \cap \overline{\{x\}}$ and any associated prime
$\mathfrak p$ of the completion $\mathcal{O}_{\overline{\{x\}}, z}^\wedge$
we have $\dim(\mathcal{O}_{\overline{\{x\}}, z}^\wedge/\mathfrak p) \geq 2$.
\end{enumerate}
\end{proposition}

\begin{proof}
If (2) holds we get (1) by a combination of
Lemmas \ref{lemma-check-finiteness-pushforward-on-associated-points},
Remark \ref{remark-closure}, and
Lemma \ref{lemma-sharp-finiteness-pushforward}.
If (2) does not hold, then $j_*i_{x, *}\mathcal{O}_{W_x}$ is not finite
for some $x \in \text{Ass}(\mathcal{F})$ by the discussion in
Remark \ref{remark-no-finiteness-pushforward}
(and Remark \ref{remark-closure}).
Thus $j_*\mathcal{F}$ is not coherent by
Lemma \ref{lemma-check-finiteness-pushforward-on-associated-points}.
\end{proof}

\begin{lemma}
\label{lemma-kollar-finiteness-H1-local}
Let $A$ be a Noetherian ring and let $I \subset A$ be an ideal.
Set $Z = V(I)$. Let $M$ be a finite $A$-module. The following
are equivalent
\begin{enumerate}
\item $H^1_Z(M)$ is a finite $A$-module, and
\item for all $\mathfrak p \in \text{Ass}(M)$, $\mathfrak p \not \in Z$
and all $\mathfrak q \in V(\mathfrak p + I)$ the completion of
$(A/\mathfrak p)_\mathfrak q$ does not have associated primes
of dimension $1$.
\end{enumerate}
\end{lemma}

\begin{proof}
Follows immediately from Proposition \ref{proposition-kollar}
via Lemma \ref{lemma-finiteness-pushforwards-and-H1-local}.
\end{proof}

\noindent
The formulation in the following lemma has the advantage that conditions
(1) and (2) are inherited by schemes of finite type over $X$.
Moreover, this is the form of finiteness which we will generalize
to higher direct images in Section \ref{section-finiteness-pushforward-II}.

\begin{lemma}
\label{lemma-finiteness-pushforward-general}
Let $X$ be a locally Noetherian scheme.
Let $j : U \to X$ be the inclusion of an
open subscheme with complement $Z$. Let $\mathcal{F}$ be a coherent
$\mathcal{O}_U$-module. Assume
\begin{enumerate}
\item $X$ is universally catenary,
\item for every $z \in Z$ the formal fibres of $\mathcal{O}_{X, z}$
are $(S_1)$.
\end{enumerate}
In this situation the following are equivalent
\begin{enumerate}
\item[(a)] for $x \in \text{Ass}(\mathcal{F})$ and
$z \in Z \cap \overline{\{x\}}$ we have
$\dim(\mathcal{O}_{\overline{\{x\}}, z}) \geq 2$, and
\item[(b)] $j_*\mathcal{F}$ is coherent.
\end{enumerate}
\end{lemma}

\begin{proof}
Let $x \in \text{Ass}(\mathcal{F})$. By Proposition \ref{proposition-kollar}
it suffices to check that $A = \mathcal{O}_{\overline{\{x\}}, z}$ satisfies
the condition of the proposition on associated primes of its completion
if and only if $\dim(A) \geq 2$.
Observe that $A$ is universally catenary (this is clear)
and that its formal fibres are $(S_1)$ as follows from
More on Algebra, Lemma \ref{more-algebra-lemma-formal-fibres-normal} and
Proposition \ref{more-algebra-proposition-finite-type-over-P-ring}.
Let $\mathfrak p' \subset A^\wedge$ be an associated prime.
As $A \to A^\wedge$ is flat,
by Algebra, Lemma \ref{algebra-lemma-bourbaki},
we find that $\mathfrak p'$ lies over $(0) \subset A$.
Since the formal fibre $A^\wedge \otimes_A f.f.(A)$
is $(S_1)$ we see that $\mathfrak p'$ is a minimal prime, see
Algebra, Lemma \ref{algebra-lemma-criterion-no-embedded-primes}.
Since $A$ is universally catenary it is formally catenary
by More on Algebra, Proposition \ref{more-algebra-proposition-ratliff}.
Hence $\dim(A^\wedge/\mathfrak p') = \dim(A)$ which
proves the equivalence.
\end{proof}






\section{Depth and dimension}
\label{section-dept-dimension}

\noindent
Some helper lemmas.

\begin{lemma}
\label{lemma-ideal-depth-function}
Let $A$ be a Noetherian ring. Let $I \subset A$ be an ideal.
Let $M$ be a finite $A$-module. Let $\mathfrak p \in V(I)$
be a prime ideal. Assume
$e = \text{depth}_{IA_\mathfrak p}(M_\mathfrak p) < \infty$.
Then there exists a nonempty open $U \subset V(\mathfrak p)$
such that $\text{depth}_{IA_\mathfrak q}(M_\mathfrak q) \geq e$
for all $\mathfrak q \in U$.
\end{lemma}

\begin{proof}
By definition of depth we have $IM_\mathfrak p \not = M_\mathfrak p$
and there exists an $M_\mathfrak p$-regular sequence
$f_1, \ldots, f_e \in IA_\mathfrak p$. After replacing $A$ by
a principal localization we may assume $f_1, \ldots, f_e \in I$
form an $M$-regular sequence, see
Algebra, Lemma \ref{algebra-lemma-regular-sequence-in-neighbourhood}.
Consider the module $M' = M/IM$. Since $\mathfrak p \in \text{Supp}(M')$
and since the support of a finite module is closed, we find
$V(\mathfrak p) \subset \text{Supp}(M')$. Thus
for $\mathfrak q \in V(\mathfrak p)$ we get
$IM_\mathfrak q \not = M_\mathfrak q$. Hence, using that
localization is exact, we see that
$\text{depth}_{IA_\mathfrak q}(M_\mathfrak q) \geq e$
for any $\mathfrak q \in V(I)$ by definition of depth.
\end{proof}

\begin{lemma}
\label{lemma-depth-function}
Let $A$ be a Noetherian ring. Let $M$ be a finite $A$-module.
Let $\mathfrak p$ be a prime ideal. Assume
$e = \text{depth}_{A_\mathfrak p}(M_\mathfrak p) < \infty$.
Then there exists a nonempty open $U \subset V(\mathfrak p)$
such that $\text{depth}_{A_\mathfrak q}(M_\mathfrak q) \geq e$
for all $\mathfrak q \in U$ and
for all but finitely many $\mathfrak q \in U$ we have
$\text{depth}_{A_\mathfrak q}(M_\mathfrak q) > e$.
\end{lemma}

\begin{proof}
By definition of depth we have $\mathfrak p M_\mathfrak p \not = M_\mathfrak p$
and there exists an $M_\mathfrak p$-regular sequence
$f_1, \ldots, f_e \in \mathfrak p A_\mathfrak p$. After replacing $A$ by
a principal localization we may assume $f_1, \ldots, f_e \in \mathfrak p$
form an $M$-regular sequence, see
Algebra, Lemma \ref{algebra-lemma-regular-sequence-in-neighbourhood}.
Consider the module $M' = M/(f_1, \ldots, f_e)M$.
Since $\mathfrak p \in \text{Supp}(M')$
and since the support of a finite module is closed, we find
$V(\mathfrak p) \subset \text{Supp}(M')$. Thus
for $\mathfrak q \in V(\mathfrak p)$ we get
$\mathfrak q M_\mathfrak q \not = M_\mathfrak q$. Hence, using that
localization is exact, we see that
$\text{depth}_{A_\mathfrak q}(M_\mathfrak q) \geq e$
for any $\mathfrak q \in V(I)$ by definition of depth.
Moreover, as soon as $\mathfrak q$ is not an associated
prime of the module $M'$, then the depth goes up.
Thus we see that the final statement holds by
Algebra, Lemma \ref{algebra-lemma-finite-ass}.
\end{proof}

\begin{lemma}
\label{lemma-sitting-in-degrees}
Let $(A, \mathfrak m)$ be a Noetherian local ring with
normalized dualizing complex $\omega_A^\bullet$.
Let $M$ be a finite $A$-module.
Set $E^i = \text{Ext}_A^{-i}(M, \omega_A^\bullet)$.
Then
\begin{enumerate}
\item $E^i$ is a finite $A$-module nonzero only for
$0 \leq i \leq \dim(\text{Supp}(M))$,
\item $\dim(\text{Supp}(E^i)) \leq i$,
\item $\text{depth}(M)$ is the smallest integer $\delta \geq 0$ such that
$E^\delta \not = 0$,
\item $\mathfrak p \in \text{Supp}(E^0 \oplus \ldots \oplus E^i)
\Leftrightarrow
\text{depth}_{A_\mathfrak p}(M_\mathfrak p) + \dim(A/\mathfrak p) \leq i$,
\item the annihilator of $E^i$ is equal to the annihilator
of $H^i_\mathfrak m(M)$.
\end{enumerate}
\end{lemma}

\begin{proof}
Parts (1), (2), and (3) are copies of the statements in
Dualizing Complexes, Lemma \ref{dualizing-lemma-sitting-in-degrees}.
For a prime $\mathfrak p$ of $A$ we have that
$(\omega_A^\bullet)_\mathfrak p[-\dim(A/\mathfrak p)]$
is a normalized dualzing complex for $A_\mathfrak p$.
See Dualizing Complexes, Lemma \ref{dualizing-lemma-dimension-function}.
Thus
$$
E^i_\mathfrak p =
\text{Ext}^{-i}_A(M, \omega_A^\bullet)_\mathfrak p =
\text{Ext}^{-i + \dim(A/\mathfrak p)}_{A_\mathfrak p}
(M_\mathfrak p, (\omega_A^\bullet)_\mathfrak p[-\dim(A/\mathfrak p)])
$$
is zero for
$i - \dim(A/\mathfrak p) < \text{depth}_{A_\mathfrak p}(M_\mathfrak p)$
and nonzero for
$i = \dim(A/\mathfrak p) + \text{depth}_{A_\mathfrak p}(M_\mathfrak p)$
by part (3) over $A_\mathfrak p$.
This proves part (4).
If $E$ is an injective hull of the residue field of $A$, then we have
$$
\Hom_A(H^i_\mathfrak m(M), E) =
\text{Ext}^{-i}_A(M, \omega_A^\bullet)^\wedge =
(E^i)^\wedge =
E^i \otimes_A A^\wedge
$$
by the local duality theorem (in the form of
Dualizing Complexes, Lemma \ref{dualizing-lemma-special-case-local-duality}).
Since $A \to A^\wedge$ is faithfully flat, we find (5) is true by
Matlis duality
(Dualizing Complexes, Proposition \ref{dualizing-proposition-matlis}).
\end{proof}

\begin{lemma}
\label{lemma-descending-chain}
Let $(A, \mathfrak m)$ be a Noetherian local ring.
\begin{enumerate}
\item Let $M$ be a finite $A$-module. Then the $A$-module
$H^i_\mathfrak m(M)$ satisfies the descending chain condition
for any $i$.
\item Let $U = \Spec(A) \setminus \{\mathfrak m\}$ be the
punctured spectrum of $A$.
Let $\mathcal{F}$ be a coherent $\mathcal{O}_U$-module.
Then the $A$-module $H^i(U, \mathcal{F})$
satisfies the descending chain condition for $i > 0$.
\end{enumerate}
\end{lemma}

\begin{proof}
Proof of (1). Let $A^\wedge$ be the completion of $A$. Observe that
$H^i_\mathfrak m(M) \otimes_A A^\wedge =
H^i_{\mathfrak mA^\wedge}(M \otimes_A A^\wedge)$ by
Dualizing Complexes, Lemma \ref{dualizing-lemma-torsion-change-rings}.
Thus we may assume $A$ is complete; some details omitted.
If $A$ is complete, then $A$ has a normalized dualizing complex
$\omega_A^\bullet$ and we find that $H^i_\mathfrak m(M)$ is
Matlis dual to the finite $A$-module $\text{Ext}^{-i}_A(M, \omega_A^\bullet)$
by the local duality theorem (Dualizing Complexes, Lemma
\ref{dualizing-lemma-special-case-local-duality}).
We find (1) is true by Matlis duality
(Dualizing Complexes, Proposition \ref{dualizing-proposition-matlis}).
Part (2) follows from (1) via
Lemma \ref{lemma-finiteness-pushforwards-and-H1-local}.
\end{proof}

\begin{lemma}
\label{lemma-ML-local}
Let $(A, \mathfrak m)$ be a Noetherian local ring.
\begin{enumerate}
\item Let $(M_n)$ be an inverse system of finite $A$-modules. Then the
inverse system $H^i_\mathfrak m(M_n)$ satisfies the Mittag-Leffler
condition for any $i$.
\item Let $U = \Spec(A) \setminus \{\mathfrak m\}$ be the
punctured spectrum of $A$.
Let $\mathcal{F}_n$ be an inverse system of
coherent $\mathcal{O}_U$-modules.
Then the inverse system $H^i(U, \mathcal{F}_n)$
satisfies the Mittag-Leffler condition for $i > 0$.
\end{enumerate}
\end{lemma}

\begin{proof}
Follows immediately from Lemma \ref{lemma-descending-chain}.
\end{proof}






\section{Improving coherent modules}
\label{section-improve}

\noindent
Similar constructions can be found in \cite{EGA} and more recently in
\cite{Kollar-local-global-hulls} and \cite{Kollar-variants}.

\begin{lemma}
\label{lemma-get-depth-1-along-Z}
Let $X$ be a Noetherian scheme. Let $Z \subset X$ be a closed subscheme.
Let $\mathcal{F}$ be a coherent $\mathcal{O}_X$-module.
Then there is a canonical surjection $\mathcal{F} \to \mathcal{F}'$
of coherent $\mathcal{O}_X$-modules such that
\begin{enumerate}
\item $\mathcal{F}|_{X \setminus Z} \to \mathcal{F}'|_{X \setminus Z}$
is an isomorphism,
\item for $z \in Z$ we have
$\text{depth}_{\mathcal{O}_{X, z}}(\mathcal{F}'_z) \geq 1$.
\end{enumerate}
If $f : Y \to X$ is a flat morphism with $Y$ Noetherian, then
$f^*\mathcal{F} \to f^*\mathcal{F}'$ is the corresponding
quotient for $f^{-1}(Z) \subset Y$ and $f^*\mathcal{F}$.
\end{lemma}

\begin{proof}
Condition (2) on $\mathcal{F}'$ just means that $\mathcal{F}'$
has no associated points in $Z$. For example if
$Z = X$, then $\mathcal{F}' = 0$ is the solution.
The statement on pullbacks follows from
Divisors, Lemma \ref{divisors-lemma-bourbaki}.

\medskip\noindent
Existence of $\mathcal{F} \to \mathcal{F}'$.
Let $\mathcal{G} \subset \mathcal{F}$
be the quasi-coherent subsheaf of sections supported in $Z$, see
Properties, Definition
\ref{properties-definition-subsheaf-sections-supported-on-closed}.
Set $\mathcal{F}' = \mathcal{F}/\mathcal{G}$.
Since $\mathcal{F}'$ does not have any nonzero section
whose support is contained in $Z$ we see that
$\text{Ass}(\mathcal{F}') \cap Z = \emptyset$
and the proof is complete.
\end{proof}

\begin{lemma}
\label{lemma-get-depth-2-along-Z}
Let $X$ be a Noetherian scheme. Let $Z \subset X$ be a closed subscheme.
Let $\mathcal{F}$ be a coherent $\mathcal{O}_X$-module. Denote
$j : X \setminus Z \to X$ the inclusion morphism.
Assume $\mathcal{F}' = j_*(\mathcal{F}|_{X \setminus Z})$
is coherent (Proposition \ref{proposition-kollar} and
Lemma \ref{lemma-finiteness-pushforward-general}).
Then $\mathcal{F} \to \mathcal{F}'$ is the unique map
of coherent $\mathcal{O}_X$-modules such that
\begin{enumerate}
\item $\mathcal{F}|_{X \setminus Z} \to \mathcal{F}'|_{X \setminus Z}$
is an isomorphism,
\item for $z \in Z$ we have
$\text{depth}_{\mathcal{O}_{X, z}}(\mathcal{F}'_z) \geq 2$.
\end{enumerate}
If $f : Y \to X$ is a flat morphism with $Y$ Noetherian, then
$f^*\mathcal{F} \to f^*\mathcal{F}'$ is the corresponding
map for $f^{-1}(Z) \subset Y$ and $f^*\mathcal{F}$.
\end{lemma}

\begin{proof}
Let us show that $\text{depth}_{\mathcal{O}_{X, z}}(\mathcal{F}'_z) \geq 2$
for $z \in Z$. Namely, let $U$ be the punctured spectrum of
$\mathcal{O}_{X, z}$. Then $U$ contains the inverse image of
$X \setminus Z$ along $\Spec(\mathcal{O}_{X, z}) \to X$.
Since $\mathcal{F}' = j_*(\mathcal{F}|_{X \setminus Z}) =
j_*(\mathcal{F}'|_{X \setminus Z})$, the same is true after
base change by the flat morphism $\Spec(\mathcal{O}_{X, z}) \to X$
(Cohomology of Schemes, Lemma
\ref{coherent-lemma-flat-base-change-cohomology}).
A fortiori, the canonical map
$\mathcal{F}'_z \to H^0(U, \mathcal{F}'|_U)$
is an isomorphism. This means that $H^i_{\mathfrak m_z}(\mathcal{F}'_z)$
is zero for $i = 0, 1$, see
Lemma \ref{lemma-finiteness-pushforwards-and-H1-local}.
Thus the depth is at least $2$.
We omit the proof of the other statements.
\end{proof}

\begin{lemma}
\label{lemma-make-S2-along-Z}
Let $X$ be a Noetherian scheme. Let $Z \subset X$ be a closed subscheme.
Let $\mathcal{F}$ be a coherent $\mathcal{O}_X$-module. Assume
$X$ is universally catenary and the formal fibres of
local rings have $(S_1)$.
Then there exists a canonical map $\mathcal{F} \to \mathcal{F}'$
of coherent $\mathcal{O}_X$-modules such that
\begin{enumerate}
\item $\mathcal{F}|_{X \setminus Z} \to \mathcal{F}'|_{X \setminus Z}$
is an isomorphism,
\item for $z \in Z$ we have either
\begin{enumerate}
\item $\text{depth}_{\mathcal{O}_{X, z}}(\mathcal{F}'_z) \geq 2$, or
\item there is an $x \in \text{Ass}(\mathcal{F}|_{X \setminus Z})$
with $z \in \overline{\{x\}}$ and
$\dim(\mathcal{O}_{\overline{\{x\}}, z}) = 1$
and in this case $\mathcal{F}_z \to \mathcal{F}'_z$ is the unique
surjection with $\text{depth}_{\mathcal{O}_{X, z}}(\mathcal{F}'_z) = 1$.
\end{enumerate}
\end{enumerate}
If $f : Y \to X$ is a Cohen-Macaulay morphism with $Y$ Noetherian,
then $f^*\mathcal{F} \to f^*\mathcal{F}'$ satisfies the same properties
with respect to $f^{-1}(Z) \subset Y$ and $f^*\mathcal{F}$.
\end{lemma}

\begin{proof}
We first replace $\mathcal{F}$ by the quotient of it constructed in
Lemma \ref{lemma-get-depth-1-along-Z}. Recall that
$\text{Ass}(\mathcal{F}) = \{x_1, \ldots, x_n\}$
is finite (and $x_i \not \in Z$ by our choice of $\mathcal{F}$).
Let $Y_i$ be the closure of $\{x_i\}$. Let
$Z_{i, j}$ be the irreducible components of $Z \cap Y_i$.
Observe that $\text{Supp}(\mathcal{F}) \cap Z = \bigcup Z_{i, j}$.
Let $z_{i, j} \in Z_{i, j}$ be the generic point.
Let
$$
d_{i, j} = \dim(\mathcal{O}_{\overline{\{x_i\}}, z_{i, j}})
$$
If $d_{i, j} = 1$, then condition (2)(b) holds for
$\mathcal{F}$ at $z_{i, j}$. Thus we do not need to modify
$\mathcal{F}$ at these points. Furthermore, still assuming
$d_{i, j} = 1$, using Lemma \ref{lemma-depth-function}
we can find an open neighbourhood
$z_{i, j} \in V_{i, j} \subset X$ such that
$\text{depth}_{\mathcal{O}_{X, z}}(\mathcal{F}_z) \geq 2$
for $z \in Z_{i, j} \cap V_{i, j}$, $z \not = z_{i, j}$.
Set
$$
Z' = X \setminus
\left(
X \setminus Z \cup \bigcup\nolimits_{d_{i, j} = 1} V_{i, j})
\right)
$$
Denote $j' : X \setminus Z' \to X$. By our choice of $Z'$
the assumptions of Lemma \ref{lemma-finiteness-pushforward-general}
are satisfied.
We conclude by setting $\mathcal{F}' = j'_*(\mathcal{F}|_{X \setminus Z'})$
and applying Lemma \ref{lemma-get-depth-2-along-Z}.

\medskip\noindent
The final statement follows from the formula for the change in
depth along a flat local homomorphism, see
Algebra, Lemma \ref{algebra-lemma-apply-grothendieck-module}
and the assumption on the fibres of $f$ inherent in $f$ being
Cohen-Macaulay. Details omitted.
\end{proof}






\section{Finiteness of local cohomology, II}
\label{section-finiteness-II}

\noindent
We continue the discussion of finiteness of local cohomology
started in Section \ref{section-finiteness}.
Let $A$ be a Noetherian ring and let $I \subset A$ be an ideal.
Set $X = \Spec(A)$ and $Z = V(I) \subset X$. Let $M$ be a finite $A$-module.
We define
\begin{equation}
\label{equation-cutoff}
s_{A, I}(M) =
\min \{
\text{depth}_{A_\mathfrak p}(M_\mathfrak p) + \dim((A/\mathfrak p)_\mathfrak q)
\mid
\mathfrak p \in X \setminus Z, \mathfrak q \in Z,
\mathfrak p \subset \mathfrak q
\}
\end{equation}
Our conventions on depth are that the depth of $0$ is $\infty$
thus we only need to consider primes $\mathfrak p$ in the support
of $M$. It will turn out that $s_{A, I}(M)$ is an important invariant of
the situation.

\begin{lemma}
\label{lemma-cutoff}
Let $A \to B$ be a finite homomorphism of Noetherian rings.
Let $I \subset A$ be an ideal and set $J = IB$. Let $M$ be
a finite $B$-module. If $A$ is universally catenary, then
$s_{B, J}(M) = s_{A, I}(M)$.
\end{lemma}

\begin{proof}
Let $\mathfrak p \subset \mathfrak q \subset A$ be primes with
$I \subset \mathfrak q$ and $I \not \subset \mathfrak p$.
Since $A \to B$ is finite there are finitely many primes
$\mathfrak p_i$ lying over $\mathfrak p$. By
Algebra, Lemma \ref{algebra-lemma-depth-goes-down-finite}
we have
$$
\text{depth}(M_\mathfrak p) = \min \text{depth}(M_{\mathfrak p_i})
$$
Let $\mathfrak p_i \subset \mathfrak q_{ij}$ be primes lying
over $\mathfrak q$. By going up for $A \to B$
(Algebra, Lemma \ref{algebra-lemma-integral-going-up})
there is at least one $\mathfrak q_{ij}$ for each $i$.
Then we see that
$$
\dim((B/\mathfrak p_i)_{\mathfrak q_{ij}}) =
\dim((A/\mathfrak p)_\mathfrak q)
$$
by the dimension formula, see
Algebra, Lemma \ref{algebra-lemma-dimension-formula}.
This implies that the minimum of the quantities
used to define $s_{B, J}(M)$
for the pairs $(\mathfrak p_i, \mathfrak q_{ij})$
is equal to the quantity for the pair $(\mathfrak p, \mathfrak q)$.
This proves the lemma.
\end{proof}

\begin{lemma}
\label{lemma-cutoff-completion}
Let $A$ be a universally catenary Noetherian local ring.
Let $I \subset A$ be an ideal. Let $M$ be
a finite $A$-module. Then
$$
s_{A, I}(M) \geq s_{A^\wedge, I^\wedge}(M^\wedge)
$$
If the formal fibres of $A$ are $(S_n)$, then
$\min(n + 1, s_{A, I}(M)) \leq s_{A^\wedge, I^\wedge}(M^\wedge)$.
\end{lemma}

\begin{proof}
Write $X = \Spec(A)$, $X^\wedge = \Spec(A^\wedge)$, $Z = V(I) \subset X$, and
$Z^\wedge = V(I^\wedge)$.
Let $\mathfrak p' \subset \mathfrak q' \subset A^\wedge$
be primes with $\mathfrak p' \not \in Z^\wedge$ and
$\mathfrak q' \in Z^\wedge$. Let $\mathfrak p \subset \mathfrak q$
be the corresponding primes of $A$. Then $\mathfrak p \not \in Z$
and $\mathfrak q \in Z$. Picture
$$
\xymatrix{
\mathfrak p' \ar[r] & \mathfrak q' \ar[r] & A^\wedge \\
\mathfrak p \ar[r] \ar@{-}[u] &
\mathfrak q \ar[r] \ar@{-}[u] & A \ar[u]
}
$$
Let us write
\begin{align*}
a & = \dim(A/\mathfrak p) = \dim(A^\wedge/\mathfrak pA^\wedge),\\
b & = \dim(A/\mathfrak q) = \dim(A^\wedge/\mathfrak qA^\wedge),\\
a' & = \dim(A^\wedge/\mathfrak p'),\\
b' & = \dim(A^\wedge/\mathfrak q')
\end{align*}
Equalities by
More on Algebra, Lemma \ref{more-algebra-lemma-completion-dimension}.
We also write
\begin{align*}
p & = \dim(A^\wedge_{\mathfrak p'}/\mathfrak p A^\wedge_{\mathfrak p'}) =
\dim((A^\wedge/\mathfrak p A^\wedge)_{\mathfrak p'}) \\
q & = \dim(A^\wedge_{\mathfrak q'}/\mathfrak p A^\wedge_{\mathfrak q'}) =
\dim((A^\wedge/\mathfrak q A^\wedge)_{\mathfrak q'})
\end{align*}
Since $A$ is universally catenary we see that
$A^\wedge/\mathfrak pA^\wedge = (A/\mathfrak p)^\wedge$
is equidimensional of dimension $a$
(More on Algebra, Proposition \ref{more-algebra-proposition-ratliff}).
Hence $a = a' + p$. Similarly $b = b' + q$.
By Algebra, Lemma \ref{algebra-lemma-apply-grothendieck-module}
applied to the flat local ring map
$A_\mathfrak p \to A^\wedge_{\mathfrak p'}$
we have
$$
\text{depth}(M^\wedge_{\mathfrak p'})
=
\text{depth}(M_\mathfrak p) +
\text{depth}(A^\wedge_{\mathfrak p'} / \mathfrak p A^\wedge_{\mathfrak p'})
$$
The quantity we are minimizing for $s_{A, I}(M)$ is
$$
s(\mathfrak p, \mathfrak q) =
\text{depth}(M_\mathfrak p) + \dim((A/\mathfrak p)_\mathfrak q) =
\text{depth}(M_\mathfrak p) + a - b
$$
(last equality as $A$ is catenary). The quantity we are minimizing
for $s_{A^\wedge, I^\wedge}(M^\wedge)$
is
$$
s(\mathfrak p', \mathfrak q') =
\text{depth}(M^\wedge_{\mathfrak p'}) +
\dim((A^\wedge/\mathfrak p')_{\mathfrak q'}) =
\text{depth}(M^\wedge_{\mathfrak p'}) + a' - b'
$$
(last equality as $A^\wedge$ is catenary).
Now we have enough notation in place to start the proof.

\medskip\noindent
Let $\mathfrak p \subset \mathfrak q \subset A$ be primes
with $\mathfrak p \not \in Z$ and $\mathfrak q \in Z$ such that
$s_{A, I}(M) = s(\mathfrak p, \mathfrak q)$.
Then we can pick $\mathfrak q'$ minimal over $\mathfrak q A^\wedge$
and $\mathfrak p' \subset \mathfrak q'$ minimal over
$\mathfrak p A^\wedge$ (using going down for $A \to A^\wedge$).
Then we have four primes as above with $p = 0$ and $q = 0$.
Moreover, we have
$\text{depth}(A^\wedge_{\mathfrak p'} / \mathfrak p A^\wedge_{\mathfrak p'})=0$
also because $p = 0$. This means that
$s(\mathfrak p', \mathfrak q') = s(\mathfrak p, \mathfrak q)$.
Thus we get the first inequality.

\medskip\noindent
Assume that the formal fibres of $A$ are $(S_n)$. Then
$\text{depth}(A^\wedge_{\mathfrak p'} / \mathfrak p A^\wedge_{\mathfrak p'})
\geq \min(n, p)$.
Hence
$$
s(\mathfrak p', \mathfrak q') \geq
s(\mathfrak p, \mathfrak q) + q + \min(n, p) - p \geq
s_{A, I}(M) + q + \min(n, p) - p
$$
Thus the only way we can get in trouble is if $p > n$. If this happens
then
\begin{align*}
s(\mathfrak p', \mathfrak q')
& =
\text{depth}(M^\wedge_{\mathfrak p'}) +
\dim((A^\wedge/\mathfrak p')_{\mathfrak q'}) \\
& =
\text{depth}(M_\mathfrak p) +
\text{depth}(A^\wedge_{\mathfrak p'} / \mathfrak p A^\wedge_{\mathfrak p'}) +
\dim((A^\wedge/\mathfrak p')_{\mathfrak q'}) \\
& \geq
0 + n + 1
\end{align*}
because $(A^\wedge/\mathfrak p')_{\mathfrak q'}$ has at least two primes.
This proves the second inequality.
\end{proof}

\noindent
The method of proof of the following lemma works more generally,
but the stronger results one gets will be subsumed in
Theorem \ref{theorem-finiteness} below.

\begin{lemma}
\label{lemma-local-annihilator}
\begin{reference}
This is a special case of
\cite[Satz 1]{Faltings-annulators}.
\end{reference}
Let $A$ be a Gorenstein Noetherian local ring. Let $I \subset A$
be an ideal and set $Z = V(I) \subset \Spec(A)$.
Let $M$ be a finite $A$-module. Let $s = s_{A, I}(M)$ as in
(\ref{equation-cutoff}). Then $H^i_Z(M)$ is finite for $i < s$,
but $H^s_Z(M)$ is not finite.
\end{lemma}

\begin{proof}
An important role will be played by the finite $A$-modules
$$
E^i = \Ext_A^i(M, A)
$$
For $\mathfrak p \subset A$ we will write $H^i_\mathfrak p$ to denote the
local cohomology of a $A_\mathfrak p$-module. Then we see that
the $\mathfrak pA_\mathfrak p$-adic completion of
$$
(E^i)_\mathfrak p = \Ext^i_{A_\mathfrak p}(M_\mathfrak p, A_\mathfrak p)
$$
is Matlis dual to
$$
H^{\dim(A_\mathfrak p) - i}_{\mathfrak p}(M_\mathfrak p)
$$
by Dualizing Complexes, Lemma \ref{dualizing-lemma-special-case-local-duality}
and the fact that
$A_\mathfrak p$ is Gorenstein. In particular we deduce from this the
following fact: an ideal $J \subset A$ annihilates
$(E^i)_\mathfrak p$ if and only if $J$ annihilates
$H^{\dim(A_\mathfrak p) - i}_{\mathfrak p}(M_\mathfrak p)$.
Set $Z_n = \{\mathfrak p \in Z \mid \dim(A/\mathfrak p) \leq n\}$.
Observe that $Z_{-1} = \emptyset$ and $Z_n = Z$ for $n = \dim(Z)$.

\medskip\noindent
Proof of finiteness for $i < s$. We will use a double induction to
do this. For $i < s$ consider the induction hypothesis $IH_i$:
$H^a_Z(M)$ is finite for $0 \leq a \leq i$. The case $IH_0$ is trivial
because $H^0_Z(M)$ is a submodule of $M$ and hence finite.

\medskip\noindent
Induction step. Assume $IH_{i - 1}$ holds for some $0 < i < s$.
For $0 \leq a \leq i - 1$ let $J_a$ be the annihilator of
$H^a_Z(M)$. Observe that $V(J_a) \subset Z$ as the support
of the finite $A$-module $H^a_Z(M)$ is contained in $Z$.
We will show by descending induction on $n$ that there
exists an ideal $J$ with $V(J) \subset Z$ such that the
associated primes of $J H^i_Z(M)$ are in $Z_n$.
For $n = -1$ this implies $JH^i_Z(M) = 0$ 
(Algebra, Lemma \ref{algebra-lemma-ass-zero})
and hence the finiteness of $H^i_Z(M)$ by
Lemma \ref{lemma-check-finiteness-local-cohomology-by-annihilator}.
The base case $n = \dim(Z)$ is trivial.

\medskip\noindent
Thus we assume given $J$ with the property for $n$. Let $\mathfrak q \in Z_n$.
With $Z_\mathfrak q = V(IA_\mathfrak q)$ we have
$H^j_Z(M)_\mathfrak q = H^j_{Z_\mathfrak q}(M_\mathfrak q)$
by Dualizing Complexes, Lemma \ref{dualizing-lemma-torsion-change-rings}.
Consider the spectral sequence
$$
H_\mathfrak q^p(H^q_Z(M)_\mathfrak q) \Rightarrow
H^{p + q}_\mathfrak q(M_\mathfrak q)
$$
of Dualizing Complexes, Lemma \ref{dualizing-lemma-local-cohomology-ss}
for the ideals
$IA_\mathfrak q \subset \mathfrak qA_\mathfrak q \subset A_\mathfrak q$.
Below we will find an ideal $J' \subset A$ with $V(J') \subset Z$
such that $H^i_\mathfrak q(M_\mathfrak q)$ is annihilated by $J'$ for all
$\mathfrak q \in Z_n \setminus Z_{n - 1}$.
Claim: $JJ'J_0 \ldots J_{i - 1}$ will work for $n - 1$.
Namely, let $\mathfrak q \in Z_n \setminus Z_{n - 1}$.
The spectral sequence above defines a filtration
$$
E_\infty^{0, i} = E_{i + 2}^{0, i} \subset \ldots \subset E_3^{0, i} \subset
E_2^{0, i} = H^0_\mathfrak q(H^i_Z(M)_\mathfrak q)
$$
The module $E_\infty^{0, i}$ is annihilated by $J'$. The subquotients
$E_j^{0, i}/E_{j + 1}^{0, i}$ are annihilated by $J_{i - j + 1}$
because the target of $d_j^{0, i}$ is a subquotient of
$H^j_\mathfrak q(H^{i - j + 1}_Z(M))$.
Finally, by our choice of $J$ we have
$J H^i_Z(M)_\mathfrak q \subset H^0_\mathfrak q(H^i_Z(M)_\mathfrak q)$.
Thus $\mathfrak q$ cannot be an associated prime of
$JJ'J_0 \ldots J_{i - 1}H^i_Z(M)$ as desired.

\medskip\noindent
By our initial remarks we see that $J'$ should annihilate
$$
(E^{\dim(A_\mathfrak q) - i})_\mathfrak q =
(E^{\dim(A) - n - i})_\mathfrak q
$$
for all $\mathfrak q \in Z_n \setminus Z_{n - 1}$.
But if $J'$ works for one $\mathfrak q$, then it works for all
$\mathfrak q$ in an open neighbourhood of $\mathfrak q$
as the modules $E^{\dim(A) - n - i}$ are finite.
Since every subset of $X$ is Noetherian with the induced
topology (Topology, Lemma \ref{topology-lemma-Noetherian}),
we conclude that it suffices
to prove the existence of $J'$ for one $\mathfrak q$.

\medskip\noindent
Since the ext modules are finite the existence of $J'$ is
equivalent to
$$
\text{Supp}(E^{\dim(A) - n - i}) \cap \Spec(A_\mathfrak q) \subset Z.
$$
This is equivalent to showing the localization at every
$\mathfrak p \subset \mathfrak q$, $\mathfrak p \not \in Z$
is zero. Using local duality over $A_\mathfrak p$ we find that we need
to prove that
$$
H^{\dim(A_\mathfrak p) - \dim(A) + n + i}_\mathfrak p(M_\mathfrak p) =
H^{i - \dim((A/\mathfrak p)_\mathfrak q)}_\mathfrak p(M_\mathfrak p)
$$
is zero (this uses that $A$ is catenary). This vanishes exactly by
our definition of $s(M)$ and
Dualizing Complexes, Lemma \ref{dualizing-lemma-depth}.
This finishes the proof of finiteness for $i < s$.

\medskip\noindent
To prove $H^s_Z(M)$ is not finite we work
backwards through the arguments above. First, we pick a
$\mathfrak q \in Z$, $\mathfrak p \subset \mathfrak q$
with $\mathfrak p \not \in Z$ such that
$s = \text{depth}_{A_\mathfrak p}(M_\mathfrak p) +
\dim((A/\mathfrak p)_\mathfrak q)$. Then
$H^{i - \dim((A/\mathfrak p)_\mathfrak q)}_\mathfrak p(M_\mathfrak p)$
is nonzero by the nonvanishing in
Dualizing Complexes, Lemma \ref{dualizing-lemma-depth}.
Set $n = \dim(A/\mathfrak q)$. Then
there does not exist an ideal $J \subset A$ with $V(J) \subset Z$
such that $J(E^{\dim(A) - n - s})_\mathfrak q = 0$.
Thus $H^s_\mathfrak q(M_\mathfrak q)$ is not
annihilated by an ideal $J \subset A$ with $V(J) \subset Z$.
It follows from the spectral sequence displayed above
that at least one of the modules $H^i_Z(M)_\mathfrak q$,
$0 \leq i \leq s$ is not annihilated by an ideal $J \subset A$
with $V(J) \subset Z$. Since $H^i_Z(M)$ is finite for $i < s$
and hence are annihilated by such ideals,
we conclude that $H^s_Z(M)$ is not finite.
\end{proof}

\noindent
Observe that the hypotheses of the following theorem are satisfied
by excellent Noetherian rings (by definition),
by Noetherian rings which have a dualizing complex
(Dualizing Complexes, Lemma \ref{dualizing-lemma-universally-catenary} and
Dualizing Complexes, Lemma
\ref{dualizing-lemma-dualizing-gorenstein-formal-fibres}), and
by quotients of regular Noetherian rings.

\begin{theorem}
\label{theorem-finiteness}
\begin{reference}
This is a special case of \cite[Satz 2]{Faltings-finiteness}.
\end{reference}
Let $A$ be a Noetherian ring and let $I \subset A$ be an ideal.
Set $Z = V(I) \subset \Spec(A)$. Let $M$ be a finite $A$-module.
Set $s = s_{A, I}(M)$ as in (\ref{equation-cutoff}).
Assume that
\begin{enumerate}
\item $A$ is universally catenary,
\item the formal fibres of the local rings of $A$ are Cohen-Macaulay.
\end{enumerate}
Then $H^i_Z(M)$ is finite for $0 \leq i < s$ and
$H^s_Z(M)$ is not finite.
\end{theorem}

\begin{proof}
By Lemma \ref{lemma-check-finiteness-local-cohomology-locally}
we may assume that $A$ is a local ring.

\medskip\noindent
If $A$ is a Noetherian complete local ring, then we can write $A$
as the quotient of a regular complete local ring $B$ by
Cohen's structure theorem
(Algebra, Theorem \ref{algebra-theorem-cohen-structure-theorem}).
Using Lemma \ref{lemma-cutoff} and
Dualizing Complexes, Lemma
\ref{dualizing-lemma-local-cohomology-and-restriction}
we reduce to the case
of a regular local ring which is a consequence of
Lemma \ref{lemma-local-annihilator}
because a regular local ring is Gorenstein
(Dualizing Complexes, Lemma \ref{dualizing-lemma-regular-gorenstein}).

\medskip\noindent
Let $A$ be a Noetherian local ring. Let $\mathfrak m$ be the maximal ideal.
We may assume $I \subset \mathfrak m$, otherwise the lemma is trivial.
Let $A^\wedge$ be the completion of $A$, let $Z^\wedge = V(IA^\wedge)$, and
let $M^\wedge = M \otimes_A A^\wedge$ be the completion of $M$
(Algebra, Lemma \ref{algebra-lemma-completion-tensor}).
Then $H^i_Z(M) \otimes_A A^\wedge = H^i_{Z^\wedge}(M^\wedge)$ by
Dualizing Complexes, Lemma \ref{dualizing-lemma-torsion-change-rings}
and flatness of $A \to A^\wedge$
(Algebra, Lemma \ref{algebra-lemma-completion-flat}).
Hence it suffices to show that $H^i_{Z^\wedge}(M^\wedge)$ is
finite for $i < s$ and not finite for $i = s$, see
Algebra, Lemma \ref{algebra-lemma-descend-properties-modules}.
Since we know the result is true for $A^\wedge$ it suffices
to show that $s_{A, I}(M) = s_{A^\wedge, I^\wedge}(M^\wedge)$.
This follows from Lemma \ref{lemma-cutoff-completion}.
\end{proof}

\begin{remark}
\label{remark-astute-reader}
The astute reader will have realized that we can get away with a
slightly weaker condition on the formal fibres of the local rings
of $A$. Namely, in the situation of Theorem \ref{theorem-finiteness}
assume $A$ is universally catenary but make no assumptions on
the formal fibres. Suppose we have an $n$ and we want to prove that
$H^i_Z(M)$ are finite for $i \leq n$. Then the exact same proof
shows that it suffices that $s_{A, I}(M) > n$ and that
the formal fibres of local rings of $A$ are $(S_n)$.
On the other hand, if we want to show that $H^s_Z(M)$
is not finite where $s = s_{A, I}(M)$, then our arguments prove
this if the formal fibres are $(S_{s - 1})$.
\end{remark}







\section{Finiteness of pushforwards, II}
\label{section-finiteness-pushforward-II}

\noindent
This section is the continuation of
Section \ref{section-finiteness-pushforward}.
In this section we reap the fruits of the labor done in
Section \ref{section-finiteness-II}.

\begin{lemma}
\label{lemma-finiteness-Rjstar}
Let $X$ be a locally Noetherian scheme. Let $j : U \to X$ be the inclusion
of an open subscheme with complement $Z$. Let $\mathcal{F}$ be a coherent
$\mathcal{O}_U$-module. Let $n \geq 0$ be an integer. Assume
\begin{enumerate}
\item $X$ is universally catenary,
\item for every $z \in Z$ the formal fibres of
$\mathcal{O}_{X, z}$ are $(S_n)$.
\end{enumerate}
In this situation the following are equivalent
\begin{enumerate}
\item[(a)] for $x \in \text{Supp}(\mathcal{F})$ and
$z \in Z \cap \overline{\{x\}}$ we have
$\text{depth}_{\mathcal{O}_{X, x}}(\mathcal{F}_x) +
\dim(\mathcal{O}_{\overline{\{x\}}, z}) > n$,
\item[(b)] $R^pj_*\mathcal{F}$ is coherent for $0 \leq p < n$.
\end{enumerate}
\end{lemma}

\begin{proof}
The statement is local on $X$, hence we may assume $X$ is affine.
Say $X = \Spec(A)$ and $Z = V(I)$. Let $M$ be a finite $A$-module
whose associated coherent $\mathcal{O}_X$-module restricts
to $\mathcal{F}$ over $U$, see
Lemma \ref{lemma-finiteness-pushforwards-and-H1-local}.
This lemma also tells us that $R^pj_*\mathcal{F}$ is coherent
if and only if $H^{p + 1}_Z(M)$ is a finite $A$-module.
Observe that the minimum of the expressions
$\text{depth}_{\mathcal{O}_{X, x}}(\mathcal{F}_x) +
\dim(\mathcal{O}_{\overline{\{x\}}, z})$
is the number $s_{A, I}(M)$ of (\ref{equation-cutoff}).
Having said this the lemma follows from
Theorem \ref{theorem-finiteness}
as elucidated by Remark \ref{remark-astute-reader}.
\end{proof}

\begin{lemma}
\label{lemma-finiteness-for-finite-locally-free}
Let $X$ be a locally Noetherian scheme. Let $j : U \to X$ be the inclusion
of an open subscheme with complement $Z$. Let $n \geq 0$ be an integer.
If $R^pj_*\mathcal{O}_U$ is coherent for $0 \leq p < n$, then
the same is true for $R^pj_*\mathcal{F}$, $0 \leq p < n$
for any finite locally free $\mathcal{O}_U$-module $\mathcal{F}$.
\end{lemma}

\begin{proof}
The question is local on $X$, hence we may assume $X$ is affine.
Say $X = \Spec(A)$ and $Z = V(I)$. Via
Lemma \ref{lemma-finiteness-pushforwards-and-H1-local}
our lemma follows from
Lemma \ref{lemma-local-finiteness-for-finite-locally-free}.
\end{proof}

\begin{lemma}
\label{lemma-annihilate-Hp}
\begin{reference}
\cite[Lemma 1.9]{Bhatt-local}
\end{reference}
Let $A$ be a ring and let $J \subset I \subset A$ be finitely generated ideals.
Let $p \geq 0$ be an integer. Set $U = \Spec(A) \setminus V(I)$. If
$H^p(U, \mathcal{O}_U)$ is annihilated by $J^n$ for some $n$, then
$H^p(U, \mathcal{F})$ annihilated by $J^m$ for some $m = m(\mathcal{F})$
for every finite locally free $\mathcal{O}_U$-module $\mathcal{F}$.
\end{lemma}

\begin{proof}
Consider the annihilator $\mathfrak a$ of $H^p(U, \mathcal{F})$.
Let $u \in U$. There exists an open neighbourhood $u \in U' \subset U$
and an isomorphism
$\varphi : \mathcal{O}_{U'}^{\oplus r} \to \mathcal{F}|_{U'}$.
Pick $f \in A$ such that $u \in D(f) \subset U'$.
There exist maps
$$
a : \mathcal{O}_U^{\oplus r} \longrightarrow \mathcal{F}
\quad\text{and}\quad
b : \mathcal{F} \longrightarrow \mathcal{O}_U^{\oplus r}
$$
whose restriction to $D(f)$ are equal to $f^N \varphi$
and $f^N \varphi^{-1}$ for some $N$. Moreover we may assume that
$a \circ b$ and $b \circ a$ are equal to multiplication by $f^{2N}$.
This follows from Properties, Lemma
\ref{properties-lemma-section-maps-backwards}
since $U$ is quasi-compact ($I$ is finitely generated), separated, and
$\mathcal{F}$ and $\mathcal{O}_U^{\oplus r}$ are finitely presented.
Thus we see that $H^p(U, \mathcal{F})$ is annihilated by
$f^{2N}J^n$, i.e., $f^{2N}J^n \subset \mathfrak a$.

\medskip\noindent
As $U$ is quasi-compact we can find finitely many $f_1, \ldots, f_t$
and $N_1, \ldots, N_t$ such that $U = \bigcup D(f_i)$ and
$f_i^{2N_i}J^n \subset \mathfrak a$. Then $V(I) = V(f_1, \ldots, f_t)$
and since $I$ is finitely generated we conclude
$I^M \subset (f_1, \ldots, f_t)$ for some $M$.
All in all we see that $J^m \subset \mathfrak a$ for
$m \gg 0$, for example $m = M (2N_1 + \ldots + 2N_t) n$  will do.
\end{proof}




\section{Cohomological dimension}
\label{section-cd}

\noindent
A quick section about cohomological dimension.

\begin{lemma}
\label{lemma-cd}
Let $I \subset A$ be a finitely generated ideal of a ring $A$.
Set $Y = V(I) \subset X = \Spec(A)$. Let $d \geq -1$ be an integer.
The following are equivalent
\begin{enumerate}
\item $H^i_Y(A) = 0$ for $i > d$,
\item $H^i_Y(M) = 0$ for $i > d$ for every $A$-module $M$, and
\item if $d = -1$, then $Y = \emptyset$, if $d = 0$, then
$Y$ is open and closed in $X$, and if $d > 0$ then
$H^i(X \setminus Y, \mathcal{F}) = 0$ for $i \geq d$
for every quasi-coherent $\mathcal{O}_{X \setminus Y}$-module $\mathcal{F}$.
\end{enumerate}
\end{lemma}

\begin{proof}
Observe that $R\Gamma_Y(-)$ has finite cohomological dimension by
Dualizing Complexes, Lemma \ref{dualizing-lemma-local-cohomology-adjoint}
for example. Hence we can choose a large integer $N$ such that
$H^i_Y(M) = 0$ for all $A$-modules $M$.

\medskip\noindent
Let us prove that (1) and (2) are equivalent. It is immediate that
(2) implies (1). Assume (1). Choose any $A$-module $M$ and fit it into
a short exact sequence $0 \to N \to F \to M \to 0$ where $F$ is a
free $A$-module. Since $R\Gamma_Y$ is a right adjoint, we see that
$H^i_Y(-)$ commutes with direct sums. Hence $H^i_Y(F) = 0$
for $i > d$ by assumption (1). Then we see that
$H^i_Y(M) = H^{i + 1}_Y(N)$ for all $i > d$.
Thus if we've shown the vanishing of $H^j_Y(N)$ for some
$j > d + 1$ and all $A$-modules $N$, then we obtain the
vanishing of $H^{j - 1}_Y(M)$ for all $A$-modules $M$.
By induction we find that (2) is true.

\medskip\noindent
Assume $d = -1$ and (2) holds. Then $0 = H^0_Y(A/I) = A/I \Rightarrow A = I
\Rightarrow Y = \emptyset$. Thus (3) holds. We omit the proof of the converse.

\medskip\noindent
Assume $d = 0$ and (2) holds. Set
$J = H^0_I(A) = \{x \in A \mid I^nx = 0 \text{ for some }n > 0\}$.
Then
$$
H^1_Y(A) = \Coker(A \to \Gamma(X \setminus Y, \mathcal{O}_{X \setminus Y}))
\quad\text{and}\quad
H^1_Y(I) = \Coker(I \to \Gamma(X \setminus Y, \mathcal{O}_{X \setminus Y}))
$$
and the kernel of the first map is equal to $J$. See
Lemma \ref{lemma-local-cohomology}.
We conclude from (2) that $I(A/J) = A/J$.
Thus we may pick $f \in I$
mapping to $1$ in $A/J$. Then $1 - f \in J$ so $I^n(1 - f) = 0$ for some
$n > 0$. Hence $f^n = f^{n + 1}$. Then $e = f^n \in I$ is an idempotent.
Consider the complementary idempotent $e' = 1 - f^n \in J$.
For any element $g \in I$ we have $g^m e' = 0$ for some $m > 0$.
Thus $I$ is contained in the radical of ideal $(e) \subset I$.
This means $Y = V(I) = V(e)$ is open and closed in $X$ as predicted in (3).
Conversely, if $Y = V(I)$ is open and closed, then the functor
$H^0_Y(-)$ is exact and has vanshing higher derived functors.

\medskip\noindent
If $d > 0$, then we see immediately from
Lemma \ref{lemma-local-cohomology} that (2) is equivalent to (3).
\end{proof}

\begin{definition}
\label{definition-cd}
Let $I \subset A$ be a finitely generated ideal of a ring $A$.
The smallest integer $d \geq -1$ satisfying the equivalent conditions
of Lemma \ref{lemma-cd} is called the
{\it cohomological dimension of $I$ in $A$} and is
denoted $\text{cd}(A, I)$.
\end{definition}

\noindent
Thus we have $\text{cd}(A, I) = -1$ if
$I = A$ and $\text{cd}(A, I) = 0$ if $I$ is locally nilpotent
or generated by an idempotent.
Observe that $\text{cd}(A, I)$ exists by the following lemma.

\begin{lemma}
\label{lemma-bound-cd}
Let $I \subset A$ be a finitely generated ideal of a ring $A$.
Then
\begin{enumerate}
\item $\text{cd}(A, I)$ is at most equal to the number of
generators of $I$,
\item $\text{cd}(A, I) \leq r$ if there exist $f_1, \ldots, f_r \in A$
such that $V(f_1, \ldots, f_r) = V(I)$,
\item $\text{cd}(A, I) \leq c$ if $\Spec(A) \setminus V(I)$
can be covered by $c$ affine opens.
\end{enumerate}
\end{lemma}

\begin{proof}
The explicit description for $R\Gamma_Y(-)$ given in
Dualizing Complexes, Lemma \ref{dualizing-lemma-local-cohomology-adjoint}
shows that (1) is true. We can deduce (2) from (1) using the
fact that $R\Gamma_Z$ depends only on the closed subset
$Z$ and not on the choice of the finitely generated ideal
$I \subset A$ with $V(I) = Z$. This follows either from the
construction of local cohomology in
Dualizing Complexes, Section \ref{dualizing-section-local-cohomology}
combined with
More on Algebra, Lemma \ref{more-algebra-lemma-local-cohomology-closed}.
or it follows from Lemma \ref{lemma-local-cohomology-is-local-cohomology}.
To see (3) we use Lemma \ref{lemma-cd}
and the vanishing result of Cohomology of Schemes, Lemma
\ref{coherent-lemma-vanishing-nr-affines}.
\end{proof}

\begin{lemma}
\label{lemma-cd-change-rings}
Let $A \to B$ be a ring map. Let $I \subset A$ be a finitely generated ideal.
Then $\text{cd}(B, IB) \leq \text{cd}(A, I)$. If $A \to B$ is faithfully
flat, then equality holds.
\end{lemma}

\begin{proof}
Use the definition and
Dualizing Complexes, Lemma \ref{dualizing-lemma-torsion-change-rings}.
\end{proof}

\begin{lemma}
\label{lemma-cd-local}
Let $I \subset A$ be a finitely generated ideal of a ring $A$.
Then $\text{cd}(A, I) = \max \text{cd}(A_\mathfrak p, I_\mathfrak p)$.
\end{lemma}

\begin{proof}
Let $Y = V(I)$ and $Y' = V(I_\mathfrak p) \subset \Spec(A_\mathfrak p)$.
Recall that
$R\Gamma_Y(A) \otimes_A A_\mathfrak p = R\Gamma_{Y'}(A_\mathfrak p)$
by Dualizing Complexes, Lemma \ref{dualizing-lemma-torsion-change-rings}.
Thus we conclude by Algebra, Lemma \ref{algebra-lemma-characterize-zero-local}.
\end{proof}

\begin{lemma}
\label{lemma-cd-dimension}
Let $I \subset A$ be a finitely generated ideal of a ring $A$.
Then $\text{cd}(A, I) \leq \dim(A)$.
\end{lemma}

\begin{proof}
Recall that $\dim(A)$ denotes the Krull dimension. By
Lemma \ref{lemma-cd-local} we may assume $A$ is local.
If $V(I) = \emptyset$, then the result is true.
If $V(I) \not = \emptyset$, then
$\dim(\Spec(A) \setminus V(I)) < \dim(A)$ because
the closed point is missing. Observe that
$U = \Spec(A) \setminus V(I)$ is a quasi-compact
open of the spectral space $\Spec(A)$, hence a spectral space itself.
See Algebra, Lemma \ref{algebra-lemma-spec-spectral} and
Topology, Lemma \ref{topology-lemma-spectral-sub}.
Thus Cohomology, Proposition
\ref{cohomology-proposition-cohomological-dimension-spectral}
implies $H^i(U, \mathcal{F}) = 0$ for $i \geq \dim(A)$
which implies what we want by Lemma \ref{lemma-cd}.
In the Noetherian case we can use Grothendieck's Cohomology, Proposition
\ref{cohomology-proposition-vanishing-Noetherian}.
\end{proof}

\begin{lemma}
\label{lemma-cd-is-one}
Let $I \subset A$ be a finitely generated ideal of a ring $A$. If
$\text{cd}(A, I) = 1$ then $\Spec(A) \setminus V(I)$ is nonempty affine.
\end{lemma}

\begin{proof}
This follows from Lemma \ref{lemma-cd} and
Cohomology of Schemes, Lemma
\ref{coherent-lemma-quasi-compact-h1-zero-covering}.
\end{proof}

\begin{lemma}
\label{lemma-cd-maximal}
Let $(A, \mathfrak m)$ be a Noetherian local ring of dimension $d$.
Then $H^d_\mathfrak m(A)$ is nonzero and $\text{cd}(A, \mathfrak m) = d$.
\end{lemma}

\begin{proof}
By one of the characterizations of dimension, there exists
an ideal of definition for $A$ generated by $d$ elements, see
Algebra, Proposition \ref{algebra-proposition-dimension}.
Hence $\text{cd}(A, \mathfrak m) \leq d$ by
Lemma \ref{lemma-bound-cd}. Thus $H^d_\mathfrak m(A)$ is
nonzero if and only if $\text{cd}(A, I) = d$ if and only if
$\text{cd}(A, I) \geq d$.

\medskip\noindent
Let $A \to A^\wedge$ be the map from $A$ to its completion.
Observe that $A^\wedge$ is a Noetherian local ring of the
same dimension as $A$ with maximal ideal $\mathfrak m A^\wedge$.
See Algebra, Lemmas
\ref{algebra-lemma-completion-Noetherian-Noetherian},
\ref{algebra-lemma-completion-complete}, and
\ref{algebra-lemma-completion-faithfully-flat} and
More on Algebra, Lemma \ref{more-algebra-lemma-completion-dimension}.
By Lemma \ref{lemma-cd-change-rings}
it suffices to prove the lemma for $A^\wedge$.

\medskip\noindent
By the previous paragraph we may assume that $A$ is
a complete local ring. Then $A$ has a normalized dualizing complex
$\omega_A^\bullet$ (Dualizing Complexes, Lemma
\ref{dualizing-lemma-ubiquity-dualizing}).
The local duality theorem (in the form of
Dualizing Complexes, Lemma \ref{dualizing-lemma-special-case-local-duality})
tells us $H^d_\mathfrak m(A)$ is Matlis dual to
$\text{Ext}^{-d}(A, \omega_A^\bullet) = H^{-d}(\omega_A^\bullet)$
which is nonzero for example by
Dualizing Complexes, Lemma
\ref{dualizing-lemma-nonvanishing-generically-local}.
\end{proof}

\begin{lemma}
\label{lemma-cd-bound-dim-local}
Let $(A, \mathfrak m)$ be a Noetherian local ring.
Let $I \subset A$ be a proper ideal.
Let $\mathfrak p \subset A$ be a prime ideal
such that $V(\mathfrak p) \cap V(I) = \{\mathfrak m\}$.
Then $\dim(A/\mathfrak p) \leq \text{cd}(A, I)$.
\end{lemma}

\begin{proof}
By Lemma \ref{lemma-cd-change-rings} we have
$\text{cd}(A, I) \geq \text{cd}(A/\mathfrak p, I(A/\mathfrak p))$.
Since $V(I) \cap V(\mathfrak p) = \{\mathfrak m\}$ we have
$\text{cd}(A/\mathfrak p, I(A/\mathfrak p)) =
\text{cd}(A/\mathfrak p, \mathfrak m/\mathfrak p)$.
By Lemma \ref{lemma-cd-maximal} this is equal to $\dim(A/\mathfrak p)$.
\end{proof}








\section{Formal functions for a principal ideal}
\label{section-formal-functions-principal}

\noindent
In this section we ask if completion and taking cohomology commute
for sheaves of modules on schemes over an affine base $A$ when completion
is with respect to a principal ideal in $A$. Of course, we have already
discussed the theorem on formal functions in
Cohomology of Schemes, Section \ref{coherent-section-theorem-formal-functions}.
Moreover, we will see in Section \ref{section-formal-functions}
that derived completion commutes with derived cohomology in great generality.
In this section we just collect a few simple special cases of this material
that will help us with future developments.

\begin{lemma}
\label{lemma-limit-finite}
Let $A$ be a Noetherian ring complete with respect to a principal ideal $(f)$.
Let $X$ be a scheme over $\Spec(A)$. Let
$$
\ldots \to \mathcal{F}_2 \to \mathcal{F}_1 \to \mathcal{F}_0
$$
be an inverse system of $\mathcal{O}_X$-modules. Assume
\begin{enumerate}
\item $\Gamma(X, \mathcal{F}_0)$ is a finite $A$-module,
\item multiplication by $f$ on $\mathcal{F}_{n + 1}$ factors
through $\mathcal{F}_{n + 1} \to \mathcal{F}_n$ to give a
short exact sequence
$0 \to \mathcal{F}_n \to \mathcal{F}_{n + 1} \to \mathcal{F}_0 \to 0$
\end{enumerate}
Then
$$
M = \lim \Gamma(X, \mathcal{F}_n)
$$
is a finite $A$-module, $f$ is a nonzerodivisor on $M$, and
$M/fM$ is the image of $M$ in $\Gamma(X, \mathcal{F}_0)$.
\end{lemma}

\begin{proof}
Assumption (2) implies that $\mathcal{F}_0$ is annihilated by $f$
and then by induction that $\mathcal{F}_n$ is annihilated by $f^{n + 1}$.
Set $M_n = \Gamma(X, \mathcal{F}_n)$. Since $f^{n + 1}$ annihilates
$M_n$ we see that $\bigcap f^nM = 0$. Since the kernel of
$f : M_{n + 1} \to M_{n + 1}$ dies in $M_n$ by (2) we see that
$f : M \to M$ is injective. The cokernel of $f : M \to M$
is the image of $M \to M_0$. Namely, if $m = (m_n)$ is an element
of $M$ with $m_0 = 0$, then each $m_{n + 1}$ is in the image of
$M_n \to M_{n + 1}$ by assumption (2).
If $m'_n \in M_n$ maps to $m_{n + 1}$ then $f(m'_n) = (m_n)$ in $M$.
Since $A$ is Noetherian and $M_0$ is finite, we see that
$M/fM \subset M_0$ is a finite module. By
Algebra, Lemma \ref{algebra-lemma-finite-over-complete-ring}
we conclude that $M$ is finite over $A$.
\end{proof}

\begin{lemma}
\label{lemma-ML}
Let $A$ be a ring. Let $f \in A$. Let $X$ be a scheme over $\Spec(A)$. Let
$$
\ldots \to \mathcal{F}_2 \to \mathcal{F}_1 \to \mathcal{F}_0
$$
be an inverse system of $\mathcal{O}_X$-modules. Assume
\begin{enumerate}
\item $H^1(X, \mathcal{F}_0)$ is an $A$-module of finite length,
\item multiplication by $f$ on $\mathcal{F}_{n + 1}$ factors
through $\mathcal{F}_{n + 1} \to \mathcal{F}_n$ to give a
short exact sequence
$0 \to \mathcal{F}_n \to \mathcal{F}_{n + 1} \to \mathcal{F}_0 \to 0$,
\end{enumerate}
Then the system $M_n = \Gamma(X, \mathcal{F}_n)$ satisfies the
Mittag-Leffler condition.
\end{lemma}

\begin{proof}
By the short exact sequences and induction we see that
$H^1_n = H^1(X, \mathcal{F}_n)$ is an $A$-module of finite
length for all $n$. Fix $n$. Our goal is to show that
$$
Q_m = \Coker(M_m \to M_n),\quad m \geq n
$$
stabilizes for $m \gg n$. Note that $Q_m \subset H^1_{m - n}$ has finite length
and that we have surjective maps $Q_{m + 1} \to Q_m$ for all $m \geq n$.
Applying cohomology to the short exact sequence
$$
0 \to \mathcal{F}_{m - n} \to \mathcal{F}_m \to \mathcal{F}_n \to 0
$$
we get an exact sequence
$$
0 \to Q_m \to H^1_{m - n} \to H^1_m \to H^1_n
$$
of finite length modules.
Set $q_m = \text{length}_A(Q_m)$ and $l_m = \text{length}_A(H^1_m)$.
Then we conclude that
$$
l_m \leq l_{m - n} - q_m + l_n
$$
Above we have seen that $q_{m + 1} \geq q_m$ for all $n$. If the sequence
does not stabilize then for some $m_0$ we have $q_m > l_n$ for all
$m \geq m_0$. Then we would get
$$
l_m \leq l_{m - n} - q_m + l_n \leq l_{m - n} - 1
$$
provided $m \geq m_0$. This would imply that the sequence
$l_{m_0}, l_{m_0 + n}, l_{m_0 + 2n}, \ldots$ is strictly decreasing
contradicting the fact that $l_m > q_m$ and the sequence $q_m$
is nondecreasing. Thus the sequence stabilizes.
\end{proof}

\begin{lemma}
\label{lemma-ML-better}
Let $A$ be a ring. Let $f \in A$. Let $X$ be a scheme over $\Spec(A)$. Let
$$
\ldots \to \mathcal{F}_2 \to \mathcal{F}_1 \to \mathcal{F}_0
$$
be an inverse system of $\mathcal{O}_X$-modules. Assume
\begin{enumerate}
\item for every $n$ there is an $m > n$ such that the image of
$H^1(X, \mathcal{F}_m) \to H^1(X, \mathcal{F}_n)$
is an $A$-module of finite length,
\item multiplication by $f$ on $\mathcal{F}_{n + 1}$ factors
through $\mathcal{F}_{n + 1} \to \mathcal{F}_n$ to give a
short exact sequence
$0 \to \mathcal{F}_n \to \mathcal{F}_{n + 1} \to \mathcal{F}_0 \to 0$,
\end{enumerate}
Then the system $M_n = \Gamma(X, \mathcal{F}_n)$ satisfies the
Mittag-Leffler condition.
\end{lemma}

\begin{proof}
Observe that condition (1) implies that the system
$H^1(X, \mathcal{F}_n)$ has the Mittag-Leffler condition.
Denote $H^1_n \subset H^1(X, \mathcal{F}_n)$ the stable image
which is a finite length $A$-module by condition (1).
For any $m' > m > n$ we have a map of short exact sequences
$$
\xymatrix{
0 \ar[r] &
\mathcal{F}_{m' - n} \ar[r] \ar[d] &
\mathcal{F}_{m'} \ar[r] \ar[d] &
\mathcal{F}_n \ar[r] \ar@{=}[d] & 0 \\
0 \ar[r] &
\mathcal{F}_{m - n} \ar[r] &
\mathcal{F}_m \ar[r] &
\mathcal{F}_n \ar[r] & 0
}
$$
In particular, the boundary maps
$\delta : M_n \to H^1(X, \mathcal{F}_{m - n})$
have image in $H^1_{m - n}$.
Consider the six term sequence
$$
0 \to M_{m - n} \to M_m \to M_n \to H^1_{m - n} \to H^1_m \to H^1_n
$$
This is exact, except possibly at $H^1_m$. However, it is easy to show
exactness there as well: let $\xi \in H^1_m$ map to zero in $H^1_n$.
Then choose a very large $m' > m$ and a lift $\xi' \in H^1_{m'}$
mapping to $\xi$. Then $\xi'$ maps to zero in
$H^1_n \subset H^1(X, \mathcal{F}_n)$. Thus we can lift $\xi'$
to an element $\xi'' \in H^1(X, \mathcal{F}_{m' - n})$.
Since $m'$ was chosen large enough, the image of $\xi''$
in $H^1(X, \mathcal{F}_{m - n})$ lies in $H^1_{m - n}$
and maps to $\xi$ as desired.

\medskip\noindent
To finish the proof argue exactly as in the proof of
Lemma \ref{lemma-ML}.
\end{proof}

\begin{lemma}
\label{lemma-formal-functions-principal}
\begin{reference}
\cite[Lemma 1.6]{Bhatt-local}
\end{reference}
Let $A$ be a ring and $f \in A$. Let $X$ be a scheme over $A$.
Let $\mathcal{F}$ be a quasi-coherent $\mathcal{O}_X$-module.
Assume that $\mathcal{F}[f^n] = \Ker(f^n : \mathcal{F} \to \mathcal{F})$
stabilizes. Then
$$
R\Gamma(X, \lim \mathcal{F}/f^n\mathcal{F}) =
R\Gamma(X, \mathcal{F})^\wedge
$$
where the right hand side indicates the derived completion
with respect to the ideal $(f) \subset A$. Let $H^p$ be the
$p$th cohomology group of this complex. Then there are short
exact sequences
$$
0 \to R^1\lim H^{p - 1}(X, \mathcal{F}/f^n\mathcal{F})
\to H^p \to \lim H^p(X, \mathcal{F}/f^n\mathcal{F}) \to 0
$$
and
$$
0 \to H^0(H^p(X, \mathcal{F})^\wedge) \to H^p \to
T_f(H^{p + 1}(X, \mathcal{F})) \to 0
$$
where $T_f(-)$ denote the $f$-adic Tate module as in
More on Algebra, Example
\ref{more-algebra-example-spectral-sequence-principal}.
\end{lemma}

\begin{proof}
We start with the canonical identifications
\begin{align*}
R\Gamma(X, \mathcal{F})^\wedge
& =
R\lim R\Gamma(X, \mathcal{F}) \otimes_A^\mathbf{L} (A \xrightarrow{f^n} A) \\
& =
R\lim R\Gamma(X, \mathcal{F} \xrightarrow{f^n} \mathcal{F}) \\
& =
R\Gamma(X, R\lim (\mathcal{F} \xrightarrow{f^n} \mathcal{F}))
\end{align*}
The first equality holds by
More on Algebra, Lemma \ref{more-algebra-lemma-derived-completion-koszul}.
The second by the projection formula, see 
Cohomology, Lemma \ref{cohomology-lemma-projection-formula-perfect}.
The third by Cohomology, Lemma
\ref{cohomology-lemma-Rf-commutes-with-Rlim}.
Note that by
Derived Categories of Schemes, Lemma \ref{perfect-lemma-Rlim-quasi-coherent}
we have
$\lim \mathcal{F}/f^n\mathcal{F} = R\lim \mathcal{F}/f^n \mathcal{F}$.
Thus to finish the proof of the first statement of the lemma it suffices to
show that the pro-objects $(f^n : \mathcal{F} \to \mathcal{F})$
and $(\mathcal{F}/f^n \mathcal{F})$ are isomorphic. There is clearly
a map from the first system to the second. Suppose that
$\mathcal{F}[f^c] = \mathcal{F}[f^{c + 1}] = \mathcal{F}[f^{c + 2}] = \ldots$.
Then we can define an arrow of systems in $D(\mathcal{O}_X)$
in the other direction by the diagrams
$$
\xymatrix{
\mathcal{F}/\mathcal{F}[f^c] \ar[r]_-{f^{n + c}} \ar[d]_{f^c} &
\mathcal{F} \ar[d]^1 \\
\mathcal{F} \ar[r]^{f^n} & \mathcal{F}
}
$$
Since the top horizontal arrow is injective the complex
in the top row is quasi-isomorphic to $\mathcal{F}/f^{n + c}\mathcal{F}$.
Some details omitted.

\medskip\noindent
Since $R\Gamma(X, -)$ commutes with derived limits
(Injectives, Lemma \ref{injectives-lemma-RF-commutes-with-Rlim})
we see that
$$
R\Gamma(X, \lim \mathcal{F}/f^n\mathcal{F}) =
R\Gamma(X, R\lim \mathcal{F}/f^n\mathcal{F}) =
R\lim R\Gamma(X, \mathcal{F}/f^n\mathcal{F})
$$
(for first equality see first paragraph of proof).
By More on Algebra, Remark \ref{more-algebra-remark-compare-derived-limit}
we obtain exact sequences
$$
0 \to
R^1\lim H^{p - 1}(X, \mathcal{F}/f^n\mathcal{F}) \to
H^p(X, \lim \mathcal{F}/I^n\mathcal{F}) \to
\lim H^p(X, \mathcal{F}/I^n\mathcal{F}) \to 0
$$
of $A$-modules. The second set of short exact sequences follow immediately
from the discussion in More on Algebra, Example
\ref{more-algebra-example-spectral-sequence-principal}.
\end{proof}








\section{Generalities on derived completion}
\label{section-derived-completion}

\noindent
We urge the reader to skip this section on a first reading.

\medskip\noindent
The algebra version of this material can be found in
More on Algebra, Section \ref{more-algebra-section-derived-completion}.
Let $\mathcal{O}$ be a sheaf of rings on a site $\mathcal{C}$.
Let $f$ be a global section of $\mathcal{O}$. We denote
$\mathcal{O}_f$ the sheaf associated to the presheaf of localizations
$U \mapsto \mathcal{O}(U)_f$.

\begin{lemma}
\label{lemma-map-twice-localize}
Let $(\mathcal{C}, \mathcal{O})$ be a ringed site. Let $f$ be a global
section of $\mathcal{O}$.
\begin{enumerate}
\item For $L, N \in D(\mathcal{O}_f)$ we have
$R\SheafHom_\mathcal{O}(L, N) = R\SheafHom_{\mathcal{O}_f}(L, N)$.
In particular the two $\mathcal{O}_f$-structures on
$R\SheafHom_\mathcal{O}(L, N)$ agree.
\item For $K \in D(\mathcal{O})$ and
$L \in D(\mathcal{O}_f)$ we have
$$
R\SheafHom_\mathcal{O}(L, K) =
R\SheafHom_{\mathcal{O}_f}(L, R\SheafHom_\mathcal{O}(\mathcal{O}_f, K))
$$
In particular
$R\SheafHom_\mathcal{O}(\mathcal{O}_f,
R\SheafHom_\mathcal{O}(\mathcal{O}_f, K)) =
R\SheafHom_\mathcal{O}(\mathcal{O}_f, K)$.
\item If $g$ is a second global
section of $\mathcal{O}$, then
$$
R\SheafHom_\mathcal{O}(\mathcal{O}_f, R\SheafHom_\mathcal{O}(\mathcal{O}_g, K))
= R\SheafHom_\mathcal{O}(\mathcal{O}_{gf}, K).
$$
\end{enumerate}
\end{lemma}

\begin{proof}
Proof of (1). Let $\mathcal{J}^\bullet$ be a K-injective complex of
$\mathcal{O}_f$-modules representing $N$. By Cohomology on Sites, Lemma
\ref{sites-cohomology-lemma-K-injective-flat} it follows that
$\mathcal{J}^\bullet$ is a K-injective complex of
$\mathcal{O}$-modules as well. Let $\mathcal{F}^\bullet$ be a complex of
$\mathcal{O}_f$-modules representing $L$. Then
$$
R\SheafHom_\mathcal{O}(L, N) =
R\SheafHom_\mathcal{O}(\mathcal{F}^\bullet, \mathcal{J}^\bullet) =
R\SheafHom_{\mathcal{O}_f}(\mathcal{F}^\bullet, \mathcal{J}^\bullet)
$$
by
Modules on Sites, Lemma \ref{sites-modules-lemma-epimorphism-modules}
because $\mathcal{J}^\bullet$ is a K-injective complex of $\mathcal{O}$
and of $\mathcal{O}_f$-modules.

\medskip\noindent
Proof of (2). Let $\mathcal{I}^\bullet$ be a K-injective complex of
$\mathcal{O}$-modules representing $K$.
Then $R\SheafHom_\mathcal{O}(\mathcal{O}_f, K)$ is represented by
$\SheafHom_\mathcal{O}(\mathcal{O}_f, \mathcal{I}^\bullet)$ which is
a K-injective complex of $\mathcal{O}_f$-modules and of
$\mathcal{O}$-modules by
Cohomology on Sites, Lemmas \ref{sites-cohomology-lemma-hom-K-injective} and
\ref{sites-cohomology-lemma-K-injective-flat}.
Let $\mathcal{F}^\bullet$ be a complex of $\mathcal{O}_f$-modules
representing $L$. Then
$$
R\SheafHom_\mathcal{O}(L, K) =
R\SheafHom_\mathcal{O}(\mathcal{F}^\bullet, \mathcal{I}^\bullet) =
R\SheafHom_{\mathcal{O}_f}(\mathcal{F}^\bullet,
\SheafHom_\mathcal{O}(\mathcal{O}_f, \mathcal{I}^\bullet))
$$
by Modules on Sites, Lemma \ref{sites-modules-lemma-adjoint-hom-restrict}
and because $\SheafHom_\mathcal{O}(\mathcal{O}_f, \mathcal{I}^\bullet)$ is a
K-injective complex of $\mathcal{O}_f$-modules.

\medskip\noindent
Proof of (3). This follows from the fact that
$R\SheafHom_\mathcal{O}(\mathcal{O}_g, \mathcal{I}^\bullet)$
is K-injective as a complex of $\mathcal{O}$-modules and the fact that
$\SheafHom_\mathcal{O}(\mathcal{O}_f,
\SheafHom_\mathcal{O}(\mathcal{O}_g, \mathcal{H})) = 
\SheafHom_\mathcal{O}(\mathcal{O}_{gf}, \mathcal{H})$
for all sheaves of $\mathcal{O}$-modules $\mathcal{H}$.
\end{proof}

\noindent
Let $K \in D(\mathcal{O})$. We denote
$T(K, f)$ a derived limit (Derived Categories, Definition
\ref{derived-definition-derived-limit}) of the system
$$
\ldots \to K \xrightarrow{f} K \xrightarrow{f} K
$$
in $D(\mathcal{O})$.

\begin{lemma}
\label{lemma-hom-from-Af}
Let $(\mathcal{C}, \mathcal{O})$ be a ringed site. Let $f$ be a global
section of $\mathcal{O}$. Let $K \in D(\mathcal{O})$.
The following are equivalent
\begin{enumerate}
\item $R\SheafHom_\mathcal{O}(\mathcal{O}_f, K) = 0$,
\item $R\SheafHom_\mathcal{O}(L, K) = 0$ for all $L$ in $D(\mathcal{O}_f)$,
\item $T(K, f) = 0$.
\end{enumerate}
\end{lemma}

\begin{proof}
It is clear that (2) implies (1). The implication (1) $\Rightarrow$ (2)
follows from Lemma \ref{lemma-map-twice-localize}.
A free resolution of the $\mathcal{O}$-module $\mathcal{O}_f$ is given by
$$
0 \to \bigoplus\nolimits_{n \in \mathbf{N}} \mathcal{O} \to
\bigoplus\nolimits_{n \in \mathbf{N}} \mathcal{O}
\to \mathcal{O}_f \to 0
$$
where the first map sends a local section $(x_0, x_1, \ldots)$ to
$(fx_0 - x_1, fx_1 - x_2, \ldots)$ and the second map sends
$(x_0, x_1, \ldots)$ to $x_0 + x_1/f + x_2/f^2 + \ldots$.
Applying $\SheafHom_\mathcal{O}(-, \mathcal{I}^\bullet)$
where $\mathcal{I}^\bullet$ is a K-injective complex of $\mathcal{O}$-modules
representing $K$ we get a short exact sequence of complexes
$$
0 \to \SheafHom_\mathcal{O}(\mathcal{O}_f, \mathcal{I}^\bullet) \to
\prod \mathcal{I}^\bullet \to \prod \mathcal{I}^\bullet \to 0
$$
because $\mathcal{I}^n$ is an injective $\mathcal{O}$-module.
The products are products in $D(\mathcal{O})$, see
Injectives, Lemma \ref{injectives-lemma-derived-products}.
This means that the object $T(K, f)$ is a representative of
$R\SheafHom_\mathcal{O}(\mathcal{O}_f, K)$ in $D(\mathcal{O})$.
Thus the equivalence of (1) and (3).
\end{proof}

\begin{lemma}
\label{lemma-ideal-of-elements-complete-wrt}
Let $(\mathcal{C}, \mathcal{O})$ be a ringed site. Let $K \in D(\mathcal{O})$.
The rule which associates to $U$ the set $\mathcal{I}(U)$
of sections $f \in \mathcal{O}(U)$ such that $T(K|_U, f) = 0$
is a sheaf of ideals in $\mathcal{O}$.
\end{lemma}

\begin{proof}
We will use the results of Lemma \ref{lemma-hom-from-Af} without further
mention. If $f \in \mathcal{I}(U)$, and $g \in \mathcal{O}(U)$, then
$\mathcal{O}_{U, gf}$ is an $\mathcal{O}_{U, f}$-module
hence $R\SheafHom_\mathcal{O}(\mathcal{O}_{U, gf}, K|_U) = 0$, hence
$gf \in \mathcal{I}(U)$. Suppose $f, g \in \mathcal{O}(U)$.
Then there is a short exact sequence
$$
0 \to \mathcal{O}_{U, f + g} \to
\mathcal{O}_{U, f(f + g)} \oplus \mathcal{O}_{U, g(f + g)} \to
\mathcal{O}_{U, gf(f + g)} \to 0
$$
because $f, g$ generate the unit ideal in $\mathcal{O}(U)_{f + g}$.
This follows from
Algebra, Lemma \ref{algebra-lemma-standard-covering}
and the easy fact that the last arrow is surjective.
Because $R\SheafHom_\mathcal{O}( - , K|_U)$ is an exact functor
of triangulated categories the vanishing of
$R\SheafHom_{\mathcal{O}_U}(\mathcal{O}_{U, f(f + g)}, K|_U)$,
$R\SheafHom_{\mathcal{O}_U}(\mathcal{O}_{U, g(f + g)}, K|_U)$, and
$R\SheafHom_{\mathcal{O}_U}(\mathcal{O}_{U, gf(f + g)}, K|_U)$,
implies the vanishing of 
$R\SheafHom_{\mathcal{O}_U}(\mathcal{O}_{U, f + g}, K|_U)$.
We omit the verification of the sheaf condition.
\end{proof}

\noindent
We can make the following definition for any ringed site.

\begin{definition}
\label{definition-derived-complete}
Let $(\mathcal{C}, \mathcal{O})$ be a ringed site.
Let $\mathcal{I} \subset \mathcal{O}$ be a sheaf of ideals.
Let $K \in D(\mathcal{O})$. We say that $K$ is
{\it derived complete with respect to $\mathcal{I}$}
if for every object $U$ of $\mathcal{C}$ and $f \in \mathcal{I}(U)$
the object $T(K|_U, f)$ of $D(\mathcal{O}_U)$ is zero.
\end{definition}

\noindent
It is clear that the full subcategory
$D_{comp}(\mathcal{O}) = D_{comp}(\mathcal{O}, \mathcal{I}) \subset
D(\mathcal{O})$ consisting of derived complete objects
is a saturated triangulated subcategory, see
Derived Categories, Definitions
\ref{derived-definition-triangulated-subcategory} and
\ref{derived-definition-saturated}. This subcategory is preserved
under products and homotopy limits in $D(\mathcal{O})$.
But it is not preserved under countable direct sums in general.

\begin{lemma}
\label{lemma-derived-complete-internal-hom}
Let $(\mathcal{C}, \mathcal{O})$ be a ringed site.
Let $\mathcal{I} \subset \mathcal{O}$ be a sheaf of ideals.
If $K \in D(\mathcal{O})$ and $L \in D_{comp}(\mathcal{O})$, then
$R\SheafHom_\mathcal{O}(K, L) \in D_{comp}(\mathcal{O})$.
\end{lemma}

\begin{proof}
Let $U$ be an object of $\mathcal{C}$ and let $f \in \mathcal{I}(U)$.
Recall that
$$
\Hom_{D(\mathcal{O}_U)}(\mathcal{O}_{U, f}, R\SheafHom_\mathcal{O}(K, L)|_U)
=
\Hom_{D(\mathcal{O}_U)}(
K|_U \otimes_{\mathcal{O}_U}^\mathbf{L} \mathcal{O}_{U, f}, L|_U)
$$
by Cohomology on Sites, Lemma \ref{sites-cohomology-lemma-internal-hom}.
The right hand side is zero by Lemma \ref{lemma-hom-from-Af}
and the relationship between internal hom and actual hom, see
Cohomology on Sites, Lemma \ref{sites-cohomology-lemma-section-RHom-over-U}.
The same vanishing holds for all $U'/U$. Thus the object
$R\SheafHom_{\mathcal{O}_U}(\mathcal{O}_{U, f},
R\SheafHom_\mathcal{O}(K, L)|_U)$ of $D(\mathcal{O}_U)$ has vanishing
$0$th cohomology sheaf (by locus citatus). Similarly for the other
cohomology sheaves, i.e., $R\SheafHom_{\mathcal{O}_U}(\mathcal{O}_{U, f},
R\SheafHom_\mathcal{O}(K, L)|_U)$ is zero in $D(\mathcal{O}_U)$.
By Lemma \ref{lemma-hom-from-Af} we conclude.
\end{proof}

\begin{lemma}
\label{lemma-restriction-derived-complete}
Let $\mathcal{C}$ be a site. Let $\mathcal{O} \to \mathcal{O}'$
be a homomorphism of sheaves of rings. Let $\mathcal{I} \subset \mathcal{O}$
be a sheaf of ideals. The inverse image of $D_{comp}(\mathcal{O}, \mathcal{I})$
under the restriction functor $D(\mathcal{O}') \to D(\mathcal{O})$ is
$D_{comp}(\mathcal{O}', \mathcal{I}\mathcal{O}')$.
\end{lemma}

\begin{proof}
Using Lemma \ref{lemma-ideal-of-elements-complete-wrt}
we see that $K' \in D(\mathcal{O}')$ is in
$D_{comp}(\mathcal{O}', \mathcal{I}\mathcal{O}')$
if and only if $T(K'|_U, f)$ is zero for every local section
$f \in \mathcal{I}(U)$. Observe that the cohomology sheaves of
$T(K'|_U, f)$ are computed in the category of abelian sheaves,
so it doesn't matter whether we think of $f$ as a section of
$\mathcal{O}$ or take the image of $f$ as a section of $\mathcal{O}'$.
The lemma follows immediately from this and the
definition of derived complete objects.
\end{proof}

\begin{lemma}
\label{lemma-pushforward-derived-complete}
Let $f : (\Sh(\mathcal{D}), \mathcal{O}') \to (\Sh(\mathcal{C}), \mathcal{O})$
be a morphism of ringed topoi. Let $\mathcal{I} \subset \mathcal{O}$
and $\mathcal{I}' \subset \mathcal{O}'$ be sheaves of ideals such
that $f^\sharp$ sends $f^{-1}\mathcal{I}$ into $\mathcal{I}'$.
Then $Rf_*$ sends $D_{comp}(\mathcal{O}', \mathcal{I}')$
into $D_{comp}(\mathcal{O}, \mathcal{I})$.
\end{lemma}

\begin{proof}
We may assume $f$ is given by a morphism of ringed sites corresponding
to a continuous functor $\mathcal{C} \to \mathcal{D}$
(Modules on Sites, Lemma
\ref{sites-modules-lemma-morphism-ringed-topoi-comes-from-morphism-ringed-sites}
).
Let $U$ be an object of $\mathcal{C}$ and let $g$ be a section of
$\mathcal{I}$ over $U$. We have to show that
$\Hom_{D(\mathcal{O}_U)}(\mathcal{O}_{U, g}, Rf_*K|_U) = 0$
whenever $K$ is derived complete with respect to $\mathcal{I}'$.
Namely, by Cohomology on Sites, Lemma
\ref{sites-cohomology-lemma-section-RHom-over-U}
this, applied to all objects over $U$ and all shifts of $K$,
will imply that $R\SheafHom_{\mathcal{O}_U}(\mathcal{O}_{U, g}, Rf_*K|_U)$
is zero, which implies that $T(Rf_*K|_U, g)$ is zero
(Lemma \ref{lemma-hom-from-Af}) which is what we have to show
(Definition \ref{definition-derived-complete}).
Let $V$ in $\mathcal{D}$ be the image of $U$. Then
$$
\Hom_{D(\mathcal{O}_U)}(\mathcal{O}_{U, g}, Rf_*K|_U) =
\Hom_{D(\mathcal{O}'_V)}(\mathcal{O}'_{V, g'}, K|_V) = 0
$$
where $g' = f^\sharp(g) \in \mathcal{I}'(V)$. The second equality
because $K$ is derived complete and the first equality because
the derived pullback of $\mathcal{O}_{U, g}$ is $\mathcal{O}'_{V, g'}$
and
Cohomology on Sites, Lemma \ref{sites-cohomology-lemma-adjoint}.
\end{proof}

\noindent
The following lemma is the simplest case where one has derived completion.

\begin{lemma}
\label{lemma-derived-completion}
Let $(\mathcal{C}, \mathcal{O})$ be a ringed on a site. Let $f_1, \ldots, f_r$
be global sections of $\mathcal{O}$. Let $\mathcal{I} \subset \mathcal{O}$ be
the ideal sheaf generated by $f_1, \ldots, f_r$.
Then the inclusion functor $D_{comp}(\mathcal{O}) \to D(\mathcal{O})$
has a left adjoint, i.e., given any object $K$ of $D(\mathcal{O})$
there exists a map $K \to K^\wedge$ with $K^\wedge$ in $D_{comp}(\mathcal{O})$
such that the map
$$
\Hom_{D(\mathcal{O})}(K^\wedge, E) \longrightarrow \Hom_{D(\mathcal{O})}(K, E)
$$
is bijective whenever $E$ is in $D_{comp}(\mathcal{O})$. In fact
we have
$$
K^\wedge =
R\SheafHom_\mathcal{O}
(\mathcal{O} \to \prod\nolimits_{i_0} \mathcal{O}_{f_{i_0}} \to
\prod\nolimits_{i_0 < i_1} \mathcal{O}_{f_{i_0}f_{i_1}} \to
\ldots \to \mathcal{O}_{f_1\ldots f_r}, K)
$$
functorially in $K$.
\end{lemma}

\begin{proof}
Define $K^\wedge$ by the last displayed formula of the lemma.
There is a map of complexes
$$
(\mathcal{O} \to \prod\nolimits_{i_0} \mathcal{O}_{f_{i_0}} \to
\prod\nolimits_{i_0 < i_1} \mathcal{O}_{f_{i_0}f_{i_1}} \to
\ldots \to \mathcal{O}_{f_1\ldots f_r}) \longrightarrow \mathcal{O}
$$
which induces a map $K \to K^\wedge$. It suffices to prove that
$K^\wedge$ is derived complete and that $K \to K^\wedge$ is an
isomorphism if $K$ is derived complete.

\medskip\noindent
Let $f$ be a global section of $\mathcal{O}$.
By Lemma \ref{lemma-map-twice-localize} the object
$R\SheafHom_\mathcal{O}(\mathcal{O}_f, K^\wedge)$
is equal to
$$
R\SheafHom_\mathcal{O}(
(\mathcal{O}_f \to \prod\nolimits_{i_0} \mathcal{O}_{ff_{i_0}} \to
\prod\nolimits_{i_0 < i_1} \mathcal{O}_{ff_{i_0}f_{i_1}} \to
\ldots \to \mathcal{O}_{ff_1\ldots f_r}), K)
$$
If $f = f_i$ for some $i$, then $f_1, \ldots, f_r$ generate the
unit ideal in $\mathcal{O}_f$, hence the extended alternating
{\v C}ech complex
$$
\mathcal{O}_f \to \prod\nolimits_{i_0} \mathcal{O}_{ff_{i_0}} \to
\prod\nolimits_{i_0 < i_1} \mathcal{O}_{ff_{i_0}f_{i_1}} \to
\ldots \to \mathcal{O}_{ff_1\ldots f_r}
$$
is zero (even homotopic to zero). In this way we see that $K^\wedge$
is derived complete.

\medskip\noindent
If $K$ is derived complete, then $R\SheafHom_\mathcal{O}(\mathcal{O}_f, K)$
is zero for all $f = f_{i_0} \ldots f_{i_p}$, $p \geq 0$. Thus
$K \to K^\wedge$ is an isomorphism in $D(\mathcal{O})$.
\end{proof}

\noindent
Next we explain why derived completion is a completion.

\begin{lemma}
\label{lemma-derived-completion-koszul}
Let $(\mathcal{C}, \mathcal{O})$ be a ringed on a site. Let $f_1, \ldots, f_r$
be global sections of $\mathcal{O}$. Let $\mathcal{I} \subset \mathcal{O}$ be
the ideal sheaf generated by $f_1, \ldots, f_r$. Let $K \in D(\mathcal{O})$.
The derived completion $K^\wedge$ of Lemma \ref{lemma-derived-completion}
is given by the formula
$$
K^\wedge = R\lim K \otimes^\mathbf{L}_\mathcal{O} K_n
$$
where $K_n = K(\mathcal{O}, f_1^n, \ldots, f_r^n)$
is the Koszul complex on $f_1^n, \ldots, f_r^n$ over $\mathcal{O}$.
\end{lemma}

\begin{proof}
In More on Algebra, Lemma
\ref{more-algebra-lemma-extended-alternating-Cech-is-colimit-koszul}
we have seen that the extended alternating {\v C}ech complex
$$
\mathcal{O} \to \prod\nolimits_{i_0} \mathcal{O}_{f_{i_0}} \to
\prod\nolimits_{i_0 < i_1} \mathcal{O}_{f_{i_0}f_{i_1}} \to
\ldots \to \mathcal{O}_{f_1\ldots f_r}
$$
is a colimit of the Koszul complexes
$K^n = K(\mathcal{O}, f_1^n, \ldots, f_r^n)$ sitting in
degrees $0, \ldots, r$. Note that $K^n$ is a finite chain complex
of finite free $\mathcal{O}$-modules with dual
$\SheafHom_\mathcal{O}(K^n, \mathcal{O}) = K_n$ where $K_n$ is the Koszul
cochain complex sitting in degrees $-r, \ldots, 0$ (as usual). By
Lemma \ref{lemma-derived-completion}
the functor $E \mapsto E^\wedge$ is gotten by taking
$R\SheafHom$ from the extended alternating {\v C}ech complex into $E$:
$$
E^\wedge = R\SheafHom(\colim K^n, E)
$$
This is equal to $R\lim (E \otimes_\mathcal{O}^\mathbf{L} K_n)$
by
Cohomology on Sites, Lemma \ref{sites-cohomology-lemma-colim-and-lim-of-duals}.
\end{proof}

\begin{lemma}
\label{lemma-all-rings}
There exist a way to construct
\begin{enumerate}
\item for every pair $(A, I)$ consisting of a ring $A$ and a finitely
generated ideal $I \subset A$ a complex $K(A, I)$ of $A$-modules,
\item a map $K(A, I) \to A$ of complexes of $A$-modules,
\item for every ring map $A \to B$ and finitely generated ideal $I \subset A$
a map of complexes $K(A, I) \to K(B, IB)$,
\end{enumerate}
such that
\begin{enumerate}
\item[(a)] for $A \to B$ and $I \subset A$ finitely generated the diagram
$$
\xymatrix{
K(A, I) \ar[r] \ar[d] & A \ar[d] \\
K(B, IB) \ar[r] & B
}
$$
commutes,
\item[(b)] for $A \to B \to C$ and $I \subset A$ finitely generated
the composition of the maps
$K(A, I) \to K(B, IB) \to K(C, IC)$ is the map $K(A, I) \to K(C, IC)$.
\item[(c)] for $A \to B$ and a finitely generated ideal $I \subset A$
the induced map $K(A, I) \otimes_A^\mathbf{L} B \to K(B, IB)$
is an isomorphism in $D(B)$, and
\item[(d)] if $I = (f_1, \ldots, f_r) \subset A$ then there is a commutative
diagram
$$
\xymatrix{
(A \to \prod\nolimits_{i_0} A_{f_{i_0}} \to
\prod\nolimits_{i_0 < i_1} A_{f_{i_0}f_{i_1}} \to
\ldots \to A_{f_1\ldots f_r}) \ar[r] \ar[d] &  K(A, I) \ar[d] \\
A \ar[r]^1 & A
}
$$
in $D(A)$ whose horizontal arrows are isomorphisms.
\end{enumerate}
\end{lemma}

\begin{proof}
Let $S$ be the set of rings $A_0$ of the form
$A_0 = \mathbf{Z}[x_1, \ldots, x_n]/J$.
Every finite type $\mathbf{Z}$-algebra is isomorphic to
an element of $S$. Let $\mathcal{A}_0$ be the category whose objects
are pairs $(A_0, I_0)$ where $A_0 \in S$ and $I_0 \subset A_0$
is an ideal and whose morphisms $(A_0, I_0) \to (B_0, J_0)$ are
ring maps $\varphi : A_0 \to B_0$ such that $J_0 = \varphi(I_0)B_0$.

\medskip\noindent
Suppose we can construct $K(A_0, I_0) \to A_0$ functorially for
objects of $\mathcal{A}_0$ having properties (a), (b), (c), and (d).
Then we take
$$
K(A, I) = \colim_{\varphi : (A_0, I_0) \to (A, I)} K(A_0, I_0)
$$
where the colimit is over ring maps $\varphi : A_0 \to A$ such
that $\varphi(I_0)A = I$ with $(A_0, I_0)$ in $\mathcal{A}_0$.
A morphism between $(A_0, I_0) \to (A, I)$ and $(A_0', I_0') \to (A, I)$
are given by maps $(A_0, I_0) \to (A_0', I_0')$ in $\mathcal{A}_0$
commuting with maps to $A$.
The category of these $(A_0, I_0) \to (A, I)$ is filtered
(details omitted). Moreover, $\colim_{\varphi : (A_0, I_0) \to (A, I)} A_0 = A$
so that $K(A, I)$ is a complex of $A$-modules.
Finally, given $\varphi : A \to B$ and $I \subset A$
for every $(A_0, I_0) \to (A, I)$ in the colimit, the composition
$(A_0, I_0) \to (B, IB)$ lives in the colimit for $(B, IB)$.
In this way we get a map on colimits. Properties (a), (b), (c), and (d)
follow readily from this and the corresponding
properties of the complexes $K(A_0, I_0)$.

\medskip\noindent
Endow $\mathcal{C}_0 = \mathcal{A}_0^{opp}$ with the chaotic topology.
We equip $\mathcal{C}_0$ with the sheaf of rings
$\mathcal{O} : (A, I) \mapsto A$. The ideals $I$ fit together to give a
sheaf of ideals $\mathcal{I} \subset \mathcal{O}$.
Choose an injective resolution $\mathcal{O} \to \mathcal{J}^\bullet$.
Consider the object
$$
\mathcal{F}^\bullet = \bigcup\nolimits_n \mathcal{J}^\bullet[\mathcal{I}^n]
$$
Let $U = (A, I) \in \Ob(\mathcal{C}_0)$.
Since the topology in $\mathcal{C}_0$ is chaotic, the value
$\mathcal{J}^\bullet(U)$ is a resolution of $A$ by injective
$A$-modules. Hence the value $\mathcal{F}^\bullet(U)$ is an
object of $D(A)$ representing the image of $R\Gamma_I(A)$ in $D(A)$, see
Dualizing Complexes, Section \ref{dualizing-section-local-cohomology}.
Choose a complex of $\mathcal{O}$-modules $\mathcal{K}^\bullet$
and a commutative diagram
$$
\xymatrix{
\mathcal{O} \ar[r] & \mathcal{J}^\bullet \\
\mathcal{K}^\bullet \ar[r] \ar[u] & \mathcal{F}^\bullet \ar[u]
}
$$
where the horizontal arrows are quasi-isomorphisms. This is possible
by the construction of the derived category $D(\mathcal{O})$.
Set $K(A, I) = \mathcal{K}^\bullet(U)$ where $U = (A, I)$.
Properties (a) and (b) are clear and properties (c) and (d)
follow from Dualizing Complexes, Lemmas
\ref{dualizing-lemma-compute-local-cohomology-noetherian} and
\ref{dualizing-lemma-local-cohomology-change-rings}.
\end{proof}

\begin{lemma}
\label{lemma-global-extended-cech-complex}
Let $(\mathcal{C}, \mathcal{O})$ be a ringed site. Let
$\mathcal{I} \subset \mathcal{O}$ be a finite type sheaf of ideals.
There exists a map $K \to \mathcal{O}$ in $D(\mathcal{O})$
such that for every $U \in \Ob(\mathcal{C})$ such that
$\mathcal{I}|_U$ is generated by $f_1, \ldots, f_r \in \mathcal{I}(U)$
there is an isomorphism
$$
(\mathcal{O}_U \to \prod\nolimits_{i_0} \mathcal{O}_{U, f_{i_0}} \to
\prod\nolimits_{i_0 < i_1} \mathcal{O}_{U, f_{i_0}f_{i_1}} \to
\ldots \to \mathcal{O}_{U, f_1\ldots f_r}) \longrightarrow K|_U
$$
compatible with maps to $\mathcal{O}_U$.
\end{lemma}

\begin{proof}
Let $\mathcal{C}' \subset \mathcal{C}$ be the full subcategory
of objects $U$ such that $\mathcal{I}|_U$ is generated by
finitely many sections. Then $\mathcal{C}' \to \mathcal{C}$
is a special cocontinuous functor
(Sites, Definition \ref{sites-definition-special-cocontinuous-functor}).
Hence it suffices to work with $\mathcal{C}'$, see
Sites, Lemma \ref{sites-lemma-equivalence}.
In other words we may assume that for every
object $U$ of $\mathcal{C}$ there exists a finitely generated
ideal $I \subset \mathcal{I}(U)$ such that
$\mathcal{I}|_U = \Im(I \otimes \mathcal{O}_U \to \mathcal{O}_U)$.
We will say that $I$ generates $\mathcal{I}|_U$.
Warning: We do not know that $\mathcal{I}(U)$ is a finitely generated
ideal in $\mathcal{O}(U)$.

\medskip\noindent
Let $U$ be an object and $I \subset \mathcal{O}(U)$ a finitely
generated ideal which generates $\mathcal{I}|_U$.
On the category $\mathcal{C}/U$ consider the complex of presheaves
$$
K_{U, I}^\bullet : U'/U \longmapsto K(\mathcal{O}(U'), I\mathcal{O}(U'))
$$
with $K(-, -)$ as in Lemma \ref{lemma-all-rings}.
We claim that the sheafification of this is independent of
the choice of $I$. Indeed, if $I' \subset \mathcal{O}(U)$
is a finitely generated ideal which also generates $\mathcal{I}|_U$, then
there exists a covering $\{U_j \to U\}$ such that
$I\mathcal{O}(U_j) = I'\mathcal{O}(U_j)$. (Hint: this works because
both $I$ and $I'$ are finitely generated and generate $\mathcal{I}|_U$.)
Hence $K_{U, I}^\bullet$ and $K_{U, I'}^\bullet$ are the {\it same}
for any object lying over one of the $U_j$. The statement
on sheafifications follows. Denote $K_U^\bullet$ the common value.

\medskip\noindent
The independence of choice of $I$ also shows that
$K_U^\bullet|_{\mathcal{C}/U'} = K_{U'}^\bullet$
whenever we are given a morphism
$U' \to U$ and hence a localization morphism
$\mathcal{C}/U' \to \mathcal{C}/U$. Thus the complexes
$K_U^\bullet$ glue to give a single well defined complex $K^\bullet$
of $\mathcal{O}$-modules. The existence of the map $K^\bullet \to \mathcal{O}$
and the quasi-isomorphism of the lemma follow immediately from
the corresponding properties of the complexes $K(-, -)$ in
Lemma \ref{lemma-all-rings}.
\end{proof}

\begin{proposition}
\label{proposition-derived-completion}
Let $(\mathcal{C}, \mathcal{O})$ be a ringed site.
Let $\mathcal{I} \subset \mathcal{O}$ be a finite type sheaf of
ideals. There exists a left adjoint to the inclusion
functor $D_{comp}(\mathcal{O}) \to D(\mathcal{O})$.
\end{proposition}

\begin{proof}
Let $K \to \mathcal{O}$ in $D(\mathcal{O})$ be as constructed in
Lemma \ref{lemma-global-extended-cech-complex}. Let $E \in D(\mathcal{O})$.
Then $E^\wedge = R\SheafHom(K, E)$ together with the map $E \to E^\wedge$
will do the job. Namely, locally on the site $\mathcal{C}$ we
recover the adjoint of Lemma \ref{lemma-derived-completion}.
This shows that $E^\wedge$ is always derived complete and that
$E \to E^\wedge$ is an isomorphism if $E$ is derived complete.
\end{proof}

\begin{remark}[Comparison with completion]
\label{remark-compare-with-completion}
Let $(\mathcal{C}, \mathcal{O})$ be a ringed site.
Let $\mathcal{I} \subset \mathcal{O}$ be a finite type sheaf of
ideals. Let $K \mapsto K^\wedge$ be the derived completion functor
of Proposition \ref{proposition-derived-completion}.
For any $n \geq 1$ the object
$K \otimes_\mathcal{O}^\mathbf{L} \mathcal{O}/\mathcal{I}^n$
is derived complete as it is annihilated by powers of
local sections of $\mathcal{I}$. Hence there is a canonical factorization
$$
K \to K^\wedge \to K \otimes_\mathcal{O}^\mathbf{L} \mathcal{O}/\mathcal{I}^n
$$
of the canonical map
$K \to K \otimes_\mathcal{O}^\mathbf{L} \mathcal{O}/\mathcal{I}^n$.
These maps are compatible for varying $n$ and we obtain a comparison map
$$
K^\wedge
\longrightarrow
R\lim \left(K \otimes_\mathcal{O}^\mathbf{L} \mathcal{O}/\mathcal{I}^n\right)
$$
The right hand side is more recognizable as a kind of completion.
In general this comparison map is not an isomorphism.
\end{remark}

\begin{remark}[Localization and derived completion]
\label{remark-localization-and-completion}
Let $(\mathcal{C}, \mathcal{O})$ be a ringed site.
Let $\mathcal{I} \subset \mathcal{O}$ be a finite type sheaf of
ideals. Let $K \mapsto K^\wedge$ be the derived completion functor
of Proposition \ref{proposition-derived-completion}. It follows
from the construction in the proof of the proposition that $K^\wedge|_U$
is the derived completion of $K|_U$ for any $U \in \Ob(\mathcal{C})$.
But we can also prove this as follows. From the definition
of derived complete objects it follows that $K^\wedge|_U$ is derived complete.
Thus we obtain a canonical map $a : (K|_U)^\wedge \to K^\wedge|_U$.
On the other hand, if $E$ is a derived complete object of
$D(\mathcal{O}_U)$, then $Rj_*E$ is a derived complete object of
$D(\mathcal{O})$ by Lemma \ref{lemma-pushforward-derived-complete}.
Here $j$ is the localization morphism
(Modules on Sites, Section \ref{sites-modules-section-localize}).
Hence we also obtain a canonical
map $b : K^\wedge \to Rj_*((K|_U)^\wedge)$. We omit the (formal) verification
that the adjoint of $b$ is the inverse of $a$.
\end{remark}

\begin{remark}[Completed tensor product]
\label{remark-completed-tensor-product}
Let $(\mathcal{C}, \mathcal{O})$ be a ringed site. Let
$\mathcal{I} \subset \mathcal{O}$ be a finite type sheaf of ideals. 
Denote $K \mapsto K^\wedge$ the adjoint of
Proposition \ref{proposition-derived-completion}.
Then we set
$$
K \otimes^\wedge_\mathcal{O} L = (K \otimes_\mathcal{O}^\mathbf{L} L)^\wedge
$$
This {\it completed tensor product} defines a functor
$D_{comp}(\mathcal{O}) \times D_{comp}(\mathcal{O}) \to D_{comp}(\mathcal{O})$
such that we have
$$
\Hom_{D_{comp}(\mathcal{O})}(K, R\SheafHom_\mathcal{O}(L, M))
=
\Hom_{D_{comp}(\mathcal{O})}(K \otimes_\mathcal{O}^\wedge L, M)
$$
for $K, L, M \in D_{comp}(\mathcal{O})$. Note that
$R\SheafHom_\mathcal{O}(L, M) \in D_{comp}(\mathcal{O})$ by
Lemma \ref{lemma-derived-complete-internal-hom}.
\end{remark}

\begin{lemma}
\label{lemma-map-identifies-koszul-and-cech-complexes}
Let $\mathcal{C}$ be a site.
Assume $\varphi : \mathcal{O} \to \mathcal{O}'$ is a flat homomorphism
of sheaves of rings. Let $f_1, \ldots, f_r$ be global sections
of $\mathcal{O}$ such that
$\mathcal{O}/(f_1, \ldots, f_r) \cong \mathcal{O}'/(f_1, \ldots, f_r)$.
Then the map of extended alternating {\v C}ech complexes
$$
\xymatrix{
\mathcal{O} \to
\prod_{i_0} \mathcal{O}_{f_{i_0}} \to
\prod_{i_0 < i_1} \mathcal{O}_{f_{i_0}f_{i_1}} \to \ldots \to
\mathcal{O}_{f_1\ldots f_r} \ar[d] \\
\mathcal{O}' \to
\prod_{i_0} \mathcal{O}'_{f_{i_0}} \to
\prod_{i_0 < i_1} \mathcal{O}'_{f_{i_0}f_{i_1}} \to \ldots \to
\mathcal{O}'_{f_1\ldots f_r}
}
$$
is a quasi-isomorphism.
\end{lemma}

\begin{proof}
Observe that the second complex is the tensor product of the first
complex with $\mathcal{O}'$. We can write the first extended
alternating {\v C}ech complex as a colimit of the Koszul complexes
$K_n = K(\mathcal{O}, f_1^n, \ldots, f_r^n)$, see
More on Algebra, Lemma
\ref{more-algebra-lemma-extended-alternating-Cech-is-colimit-koszul}.
Hence it suffices to prove $K_n \to K_n \otimes_\mathcal{O} \mathcal{O}'$
is a quasi-isomorphism. Since $\mathcal{O} \to \mathcal{O}'$ is flat
it suffices to show that $H^i \to H^i \otimes_\mathcal{O} \mathcal{O}'$
is an isomorphism where $H^i$ is the $i$th cohomology sheaf
$H^i = H^i(K_n)$. These sheaves are annihilated by $f_1^n, \ldots, f_r^n$, see
More on Algebra, Lemma \ref{more-algebra-lemma-homotopy-koszul}.
Thus it suffices to show that
$\mathcal{O}/(f_1^n, \ldots, f_r^n) \to \mathcal{O}'/(f_1^n, \ldots, f_r^n)$
is an isomorphism. Equivalently, we will show that
$\mathcal{O}/(f_1, \ldots, f_r)^n \to \mathcal{O}'/(f_1, \ldots, f_r)^n$
is an isomorphism for all $n$. This holds for $n = 1$ by assumption.
It follows for all $n$ by induction using
Modules on Sites, Lemma \ref{sites-modules-lemma-flat-over-thickening}
applied to the ring map
$\mathcal{O}/(f_1, \ldots, f_r)^{n + 1} \to \mathcal{O}/(f_1, \ldots, f_r)^n$
and the module $\mathcal{O}'/(f_1, \ldots, f_r)^{n + 1}$.
\end{proof}

\begin{lemma}
\label{lemma-restriction-derived-complete-equivalence}
Let $\mathcal{C}$ be a site. Let $\mathcal{O} \to \mathcal{O}'$ be a
homomorphism of sheaves of rings. Let $\mathcal{I} \subset \mathcal{O}$
be a finite type sheaf of ideals.
If $\mathcal{O} \to \mathcal{O}'$ is flat and
$\mathcal{O}/\mathcal{I} \cong \mathcal{O}'/\mathcal{I}\mathcal{O}'$,
then the restriction functor $D(\mathcal{O}') \to D(\mathcal{O})$
induces an equivalence
$D_{comp}(\mathcal{O}', \mathcal{I}\mathcal{O}') \to
D_{comp}(\mathcal{O}, \mathcal{I})$.
\end{lemma}

\begin{proof}
Lemma \ref{lemma-pushforward-derived-complete} implies
restriction $r : D(\mathcal{O}') \to D(\mathcal{O})$
sends $D_{comp}(\mathcal{O}', \mathcal{I}\mathcal{O}')$
into $D_{comp}(\mathcal{O}, \mathcal{I})$. We will construct a
quasi-inverse $E \mapsto E'$.

\medskip\noindent
Let $K \to \mathcal{O}$ be the morphism of $D(\mathcal{O})$
constructed in Lemma \ref{lemma-global-extended-cech-complex}. 
Set $K' = K \otimes_\mathcal{O}^\mathbf{L} \mathcal{O}'$ in $D(\mathcal{O}')$.
Then $K' \to \mathcal{O}'$ is a map in $D(\mathcal{O}')$ which
satisfies the conclusions of Lemma \ref{lemma-global-extended-cech-complex}
with respect to $\mathcal{I}' = \mathcal{I}\mathcal{O}'$.
The map $K \to r(K')$ is a quasi-isomorphism by
Lemma \ref{lemma-map-identifies-koszul-and-cech-complexes}.
Now, for $E \in D_{comp}(\mathcal{O}, \mathcal{I})$ we set
$$
E' = R\SheafHom_\mathcal{O}(r(K'), E)
$$
viewed as an object in $D(\mathcal{O}')$ using the $\mathcal{O}'$-module
structure on $K'$. Since $E$ is derived complete
we have $E = R\SheafHom_\mathcal{O}(K, E)$, see
proof of Proposition \ref{proposition-derived-completion}.
On the other hand, since $K \to r(K')$ is an isomorphism in
we see that there is an isomorphism
$E \to r(E')$ in $D(\mathcal{O})$. To finish the proof we
have to show that, if $E = r(M')$ for an object $M'$ of
$D_{comp}(\mathcal{O}', \mathcal{I}')$, then
$E' \cong M'$. To get a map we use
$$
M' = R\SheafHom_{\mathcal{O}'}(\mathcal{O}', M') \to
R\SheafHom_\mathcal{O}(r(\mathcal{O}'), r(M')) \to
R\SheafHom_\mathcal{O}(r(K'), r(M')) = E'
$$
where the second arrow uses the map $K' \to \mathcal{O}'$.
To see that this is an isomorphism, one shows that $r$ applied
to this arrow is the same as the isomorphism $E \to r(E')$ above.
Details omitted.
\end{proof}

\begin{lemma}
\label{lemma-pushforward-derived-complete-adjoint}
Let $f : (\Sh(\mathcal{D}), \mathcal{O}') \to (\Sh(\mathcal{C}), \mathcal{O})$
be a morphism of ringed topoi. Let $\mathcal{I} \subset \mathcal{O}$
and $\mathcal{I}' \subset \mathcal{O}'$ 
be finite type sheaves of ideals such that $f^\sharp$ sends
$f^{-1}\mathcal{I}$ into $\mathcal{I}'$.
Then $Rf_*$ sends $D_{comp}(\mathcal{O}', \mathcal{I}')$
into $D_{comp}(\mathcal{O}, \mathcal{I})$ and has a left adjoint
$Lf_{comp}^*$ which is $Lf^*$ followed by derived completion.
\end{lemma}

\begin{proof}
The first statement we have seen in
Lemma \ref{lemma-pushforward-derived-complete}.
Note that the second statement makes sense as we have a derived
completion functor $D(\mathcal{O}') \to D_{comp}(\mathcal{O}', \mathcal{I}')$
by Proposition \ref{proposition-derived-completion}.
OK, so now let $K \in D_{comp}(\mathcal{O}, \mathcal{I})$
and $M \in D_{comp}(\mathcal{O}', \mathcal{I}')$. Then we have
$$
\Hom(K, Rf_*M) = \Hom(Lf^*K, M) = \Hom(Lf_{comp}^*K, M)
$$
by the universal property of derived completion.
\end{proof}

\begin{lemma}
\label{lemma-pushforward-commutes-with-derived-completion}
\begin{reference}
Generalization of \cite[Lemma 6.5.9 (2)]{BS}. Compare with
\cite[Theorem 6.5]{HL-P} in the setting of quasi-coherent modules
and morphisms of (derived) algebraic stacks.
\end{reference}
Let $f : (\Sh(\mathcal{D}), \mathcal{O}') \to (\Sh(\mathcal{C}), \mathcal{O})$
be a morphism of ringed topoi. Let $\mathcal{I} \subset \mathcal{O}$
be a finite type sheaf of ideals. Let $\mathcal{I}' \subset \mathcal{O}'$
be the ideal generated by $f^\sharp(f^{-1}\mathcal{I})$.
Then $Rf_*$ commutes with derived completion, i.e.,
$Rf_*(K^\wedge) = (Rf_*K)^\wedge$.
\end{lemma}

\begin{proof}
By Proposition \ref{proposition-derived-completion} the derived completion
functors exist. By Lemma \ref{lemma-pushforward-derived-complete} the object
$Rf_*(K^\wedge)$ is derived complete, and hence we obtain a canonical map
$(Rf_*K)^\wedge \to Rf_*(K^\wedge)$ by the universal property of derived
completion. We may check this map is an isomorphism locally on $\mathcal{C}$.
Thus, since derived completion commutes with localization
(Remark \ref{remark-localization-and-completion}) we may assume
that $\mathcal{I}$ is generated by global sections $f_1, \ldots, f_r$.
Then $\mathcal{I}'$ is generated by $g_i = f^\sharp(f_i)$. By
Lemma \ref{lemma-derived-completion-koszul}
we have to prove that
$$
R\lim \left(
Rf_*K \otimes^\mathbf{L}_\mathcal{O} K(\mathcal{O}, f_1^n, \ldots, f_r^n)
\right)
=
Rf_*\left(
R\lim
K \otimes^\mathbf{L}_{\mathcal{O}'} K(\mathcal{O}', g_1^n, \ldots, g_r^n)
\right)
$$
Because $Rf_*$ commutes with $R\lim$
(Cohomology on Sites, Lemma
\ref{sites-cohomology-lemma-Rf-commutes-with-Rlim})
it suffices to prove that
$$
Rf_*K \otimes^\mathbf{L}_\mathcal{O} K(\mathcal{O}, f_1^n, \ldots, f_r^n) =
Rf_*\left(
K \otimes^\mathbf{L}_{\mathcal{O}'} K(\mathcal{O}', g_1^n, \ldots, g_r^n)
\right)
$$
This follows from the projection formula (Cohomology on Sites, Lemma
\ref{sites-cohomology-lemma-projection-formula}) and the fact that
$Lf^*K(\mathcal{O}, f_1^n, \ldots, f_r^n) =
K(\mathcal{O}', g_1^n, \ldots, g_r^n)$.
\end{proof}

\begin{lemma}
\label{lemma-formal-functions-general}
Let $A$ be a ring and let $I \subset A$ be a finitely generated ideal.
Let $\mathcal{C}$ be a site and let $\mathcal{O}$ be a sheaf
of $A$-algebras. Let $\mathcal{F}$ be a sheaf of $\mathcal{O}$-modules.
Then we have
$$
R\Gamma(\mathcal{C}, \mathcal{F})^\wedge =
R\Gamma(\mathcal{C}, \mathcal{F}^\wedge)
$$
in $D(A)$ where $\mathcal{F}^\wedge$ is the derived
completion of $\mathcal{F}$ with respect to $I\mathcal{O}$ and on the
left hand wide we have the derived completion with respect to $I$.
This produces two spectral sequences
$$
E_2^{i, j} = H^i(H^j(\mathcal{C}, \mathcal{F})^\wedge)
\quad\text{and}\quad
E_2^{p, q} = H^p(\mathcal{C}, H^q(\mathcal{F}^\wedge))
$$
both converging to
$H^*(R\Gamma(\mathcal{C}, \mathcal{F})^\wedge) =
H^*(\mathcal{C}, \mathcal{F}^\wedge)$
\end{lemma}

\begin{proof}
Apply Lemma \ref{lemma-pushforward-commutes-with-derived-completion}
to the morphism of ringed topoi $(\mathcal{C}, \mathcal{O}) \to (pt, A)$
and take cohomology to get the first statement. The second spectral sequence
is just the Leray spectral sequence for this morphism, see
Cohomology on Sites, Lemma \ref{sites-cohomology-lemma-Leray}.
The first spectral sequence is the spectral sequence of
More on Algebra, Example
\ref{more-algebra-example-derived-completion-spectral-sequence}
applied to $R\Gamma(\mathcal{C}, \mathcal{F})^\wedge$.
\end{proof}

\begin{remark}
\label{remark-local-calculation-derived-completion}
Let $(\mathcal{C}, \mathcal{O})$ be a ringed site.
Let $\mathcal{I} \subset \mathcal{O}$ be a finite type sheaf of
ideals. Let $K \mapsto K^\wedge$ be the derived completion of
Proposition \ref{proposition-derived-completion}.
Let $U \in \Ob(\mathcal{C})$ be an object such that $\mathcal{I}$
is generated as an ideal sheaf by $f_1, \ldots, f_r \in \mathcal{I}(U)$.
Set $A = \mathcal{O}(U)$ and $I = (f_1, \ldots, f_r) \subset A$.
Warning: it may not be the case that $I = \mathcal{I}(U)$.
Then we have
$$
R\Gamma(U, K^\wedge) = R\Gamma(U, K)^\wedge
$$
where the right hand side is the derived completion of
the object $R\Gamma(U, K)$ of $D(A)$ with respect to $I$.
This is true because derived completion commutes with localization
(Remark \ref{remark-localization-and-completion}) and
Lemma \ref{lemma-formal-functions-general}.
\end{remark}








\section{Application to theorem on formal functions}
\label{section-formal-functions}

\noindent
We interrupt the flow of the exposition to talk a little bit about
derived completion in the setting of quasi-coherent modules on schemes
and to use this to give a somewhat different proof of the theorem on
formal functions. We give some pointers to the literature in
Remark \ref{remark-references}.

\medskip\noindent
Lemma \ref{lemma-pushforward-commutes-with-derived-completion} is a
(very formal) derived version of the theorem on formal functions
(Cohomology of Schemes, Theorem \ref{coherent-theorem-formal-functions}).
To make this more explicit, suppose $f : X \to S$ is a morphism of schemes,
$\mathcal{I} \subset \mathcal{O}_S$ is a quasi-coherent sheaf of ideals
of finite type,
and $\mathcal{F}$ is a quasi-coherent sheaf on $X$. Then the lemma says that
\begin{equation}
\label{equation-formal-functions}
Rf_*(\mathcal{F}^\wedge) = (Rf_*\mathcal{F})^\wedge
\end{equation}
where $\mathcal{F}^\wedge$ is the derived completion of $\mathcal{F}$
with respect to $f^{-1}\mathcal{I} \cdot \mathcal{O}_X$ and the right
hand side is the derived completion of $\mathcal{F}$
with respect to $\mathcal{I}$. To see that this gives back the theorem
on formal functions we have to do a bit of work.

\begin{lemma}
\label{lemma-sections-derived-completion-pseudo-coherent}
Let $X$ be a locally Noetherian scheme. Let $\mathcal{I} \subset \mathcal{O}_X$
be a quasi-coherent sheaf of ideals. Let $K$ be a
pseudo-coherent object of $D(\mathcal{O}_X)$ with derived completion
$K^\wedge$. Then
$$
H^p(U, K^\wedge) = \lim H^p(U, K)/I^nH^p(U, K) =
H^p(U, K)^\wedge
$$
for any affine open $U \subset X$
where $I = \mathcal{I}(U)$ and where on the right we have the derived
completion with respect to $I$.
\end{lemma}

\begin{proof}
Write $U = \Spec(A)$. The ring $A$ is Noetherian
and hence $I \subset A$ is finitely generated. Then we have
$$
R\Gamma(U, K^\wedge) = R\Gamma(U, K)^\wedge
$$
by Remark \ref{remark-local-calculation-derived-completion}.
Now $R\Gamma(U, K)$ is a pseudo-coherent complex of $A$-modules
(Derived Categories of Schemes, Lemma
\ref{perfect-lemma-pseudo-coherent-affine}).
By More on Algebra, Lemma
\ref{more-algebra-lemma-derived-completion-pseudo-coherent}
we conclude that the $p$th cohomology module of $R\Gamma(U, K^\wedge)$
is equal to the $I$-adic completion of $H^p(U, K)$.
This proves the first equality. The second (less important) equality
follows immediately from a second application of the lemma just used.
\end{proof}

\begin{lemma}
\label{lemma-derived-completion-pseudo-coherent}
Let $X$ be a locally Noetherian scheme. Let $\mathcal{I} \subset \mathcal{O}_X$
be a quasi-coherent sheaf of ideals.
Let $K$ be an object of $D(\mathcal{O}_X)$. Then
\begin{enumerate}
\item the derived completion $K^\wedge$ is equal to
$R\lim (K \otimes_{\mathcal{O}_X}^\mathbf{L} \mathcal{O}_X/\mathcal{I}^n)$.
\end{enumerate}
Let $K$ is a pseudo-coherent object of $D(\mathcal{O}_X)$. Then
\begin{enumerate}
\item[(2)] the cohomology sheaf $H^q(K^\wedge)$ is equal to
$\lim H^q(K)/\mathcal{I}^nH^q(K)$.
\end{enumerate}
Let $\mathcal{F}$ be a coherent $\mathcal{O}_X$-module\footnote{For example
$H^q(K)$ for $K$ pseudo-coherent on our locally Noetherian $X$.}. Then
\begin{enumerate}
\item[(3)] the derived completion $\mathcal{F}^\wedge$ is equal to
$\lim \mathcal{F}/\mathcal{I}^n\mathcal{F}$,
\item[(4)]
$\lim \mathcal{F}/I^n \mathcal{F} = R\lim \mathcal{F}/I^n \mathcal{F}$,
\item[(5)] $H^p(U, \mathcal{F}^\wedge) = 0$ for $p \not = 0$ for all
affine opens $U \subset X$.
\end{enumerate}
\end{lemma}

\begin{proof}
Proof of (1). There is a canonical map
$$
K \longrightarrow
R\lim (K \otimes_{\mathcal{O}_X}^\mathbf{L} \mathcal{O}_X/\mathcal{I}^n),
$$
see Remark \ref{remark-compare-with-completion}.
Derived completion commutes with passing to open subschemes
(Remark \ref{remark-localization-and-completion}).
Formation of $R\lim$ commutes with passsing to open subschemes.
It follows that to check our map is an isomorphism, we may work locally.
Thus we may assume $X = U = \Spec(A)$. Say
Say $I = (f_1, \ldots, f_r)$. Let
$K_n = K(A, f_1^n, \ldots, f_r^n)$ be the Koszul complex.
By More on Algebra, Lemma \ref{more-algebra-lemma-sequence-Koszul-complexes}
we have seen that the pro-systems $\{K_n\}$ and
$\{A/I^n\}$ of $D(A)$ are isomorphic.
Using the equivalence $D(A) = D_{\QCoh}(\mathcal{O}_X)$
of Derived Categories of Schemes, Lemma
\ref{perfect-lemma-affine-compare-bounded}
we see that the pro-systems $\{K(\mathcal{O}_X, f_1^n, \ldots, f_r^n)\}$
and $\{\mathcal{O}_X/\mathcal{I}^n\}$ are isomorphic in
$D(\mathcal{O}_X)$. This proves the second equality in
$$
K^\wedge = R\lim \left(
K \otimes_{\mathcal{O}_X}^\mathbf{L} K(\mathcal{O}_X, f_1^n, \ldots, f_r^n)
\right) =
R\lim (K \otimes_{\mathcal{O}_X}^\mathbf{L} \mathcal{O}_X/\mathcal{I}^n)
$$
The first equality is
Lemma \ref{lemma-derived-completion-koszul}.

\medskip\noindent
Assume $K$ is pseudo-coherent. For $U \subset X$ affine open
we have $H^q(U, K^\wedge) = \lim H^q(U, K)/\mathcal{I}^n(U)H^q(U, K)$
by Lemma \ref{lemma-sections-derived-completion-pseudo-coherent}.
As this is true for every $U$ we see that
$H^q(K^\wedge) = \lim H^q(K)/\mathcal{I}^nH^q(K)$ as sheaves.
This proves (2).

\medskip\noindent
Part (3) is a special case of (2).
Parts (4) and (5) follow from
Derived Categories of Schemes, Lemma
\ref{perfect-lemma-Rlim-quasi-coherent}.
\end{proof}

\begin{lemma}
\label{lemma-formal-functions}
Let $A$ be a Noetherian ring and let $I \subset A$ be an ideal. Let $X$ be a
Noetherian scheme over $A$. Let $\mathcal{F}$ be a coherent
$\mathcal{O}_X$-module. Assume that $H^p(X, \mathcal{F})$ is
a finite $A$-module for all $p$. Then there are short exact sequences
$$
0 \to R^1\lim H^{p - 1}(X, \mathcal{F}/I^n\mathcal{F}) \to
H^p(X, \mathcal{F})^\wedge \to \lim H^p(X, \mathcal{F}/I^n\mathcal{F}) \to 0
$$
of $A$-modules where $H^p(X, \mathcal{F})^\wedge$ is the usual $I$-adic
completion. If $f$ is proper, then the $R^1\lim$ term is zero.
\end{lemma}

\begin{proof}
Consider the two spectral sequences of
Lemma \ref{lemma-formal-functions-general}.
The first degenerates by More on Algebra, Lemma
\ref{more-algebra-lemma-derived-completion-pseudo-coherent}.
We obtain $H^p(X, \mathcal{F})^\wedge$ in degree $p$.
This is where we use the assumption that $H^p(X, \mathcal{F})$ is
a finite $A$-module. The second degenerates because
$$
\mathcal{F}^\wedge = \lim \mathcal{F}/I^n\mathcal{F} =
R\lim \mathcal{F}/I^n\mathcal{F}
$$
is a sheaf by Lemma \ref{lemma-derived-completion-pseudo-coherent}.
We obtain $H^p(X, \lim \mathcal{F}/I^n\mathcal{F})$ in degree $p$.
Since $R\Gamma(X, -)$ commutes with derived limits
(Injectives, Lemma \ref{injectives-lemma-RF-commutes-with-Rlim})
we also get
$$
R\Gamma(X, \lim \mathcal{F}/I^n\mathcal{F}) =
R\Gamma(X, R\lim \mathcal{F}/I^n\mathcal{F}) =
R\lim R\Gamma(X, \mathcal{F}/I^n\mathcal{F})
$$
By More on Algebra, Remark
\ref{more-algebra-remark-how-unique}
we obtain exact sequences
$$
0 \to
R^1\lim H^{p - 1}(X, \mathcal{F}/I^n\mathcal{F}) \to
H^p(X, \lim \mathcal{F}/I^n\mathcal{F}) \to
\lim H^p(X, \mathcal{F}/I^n\mathcal{F}) \to 0
$$
of $A$-modules. Combining the above we get the first statement of the lemma.
The vanishing of the $R^1\lim$ term follows from
Cohomology of Schemes, Lemma \ref{coherent-lemma-ML-cohomology-powers-ideal}.
\end{proof}

\begin{remark}
\label{remark-references}
Here are some references to discussions of related material the literature.
It seems that a ``derived formal functions theorem'' for proper maps
goes back to \cite[Theorem 6.3.1]{lurie-thesis}.
There is the discussion in \cite{dag12}, especially
Chapter 4 which discusses the affine story, see
More on Algebra, Section \ref{more-algebra-section-derived-completion}.
In \cite[Section 2.9]{G-R} one finds a discussion of proper base change and
derived completion using (ind) coherent modules.
An analogue of (\ref{equation-formal-functions})
for complexes of quasi-coherent modules can be found as
\cite[Theorem 6.5]{HL-P}
\end{remark}






\section{Algebraization of formal sections}
\label{section-algebraization-sections}

\noindent
Let $(A, \mathfrak m)$ be a Noetherian local ring.
Let $I \subset A$ be an ideal. Let
$$
X = \Spec(A) \supset U = \Spec(A) \setminus \{\mathfrak m\}
$$
and denote $Y = V(I)$ the closed subscheme corresponding to $I$.
Let $\mathcal{F}$ be a coherent $\mathcal{O}_U$-module.
In this section we consider the limits
$$
\lim_n H^i(U, \mathcal{F}/I^n\mathcal{F})
$$
This is closely related to the cohomology of the pullback
of $\mathcal{F}$ to the formal completion of $U$ along $Y$;
however, since we have not yet introduced formal schemes,
we cannot use this terminology here.

\medskip\noindent
It turns out the modules we are interested in compute the cohomology
of the derived completion of the cohomology of $\mathcal{F}$ over $U$.

\begin{lemma}
\label{lemma-compare-with-derived-completion}
In the situation above the inverse systems $H^i(U, \mathcal{F}/I^n\mathcal{F})$
satisfy the Mittag-Leffler condition for all $i$ and moreover
$$
H^i(R\Gamma(U, \mathcal{F})^\wedge) =
\lim H^i(U, \mathcal{F}/I^n\mathcal{F})
$$
for all $i$ where $R\Gamma(U, \mathcal{F})^\wedge$ denotes
the derived $I$-adic completion.
\end{lemma}

\begin{proof}
Choose a finite $A$-module $M$ such that $\mathcal{F}$ is equal to
the restriction of $\widetilde{M}$ to $U$. Then we find short exact sequences
$$
0 \to H^0_\mathfrak m(M/I^nM) \to M/I^nM \to
H^0(U, \mathcal{F}/I^n\mathcal{F}) \to H^1_\mathfrak m(M/I^nM) \to 0
$$
and isomorphisms
$H^i(U, \mathcal{F}/I^n\mathcal{F}) = H^{i + 1}_\mathfrak m(M/I^nM)$
for $i \geq 1$. See Lemma \ref{lemma-finiteness-pushforwards-and-H1-local}.
We have the Mittag-Leffler conditions for
$H^i_\mathfrak m(M/I^nM)$ by Lemma \ref{lemma-ML-local}
and trivially for $M/I^nM$. Thus we get the Mittag-Leffler condition for
$H^i(U, \mathcal{F}/I^n\mathcal{F})$ by
More on Algebra, Lemma \ref{more-algebra-lemma-Mittag-Leffler}.
By Lemmas \ref{lemma-formal-functions-general} and
\ref{lemma-derived-completion-pseudo-coherent} we have
$$
R\Gamma(U, \mathcal{F})^\wedge =
R\Gamma(U, \mathcal{F}^\wedge) =
R\Gamma(U, R\lim \mathcal{F}/I^n\mathcal{F})
$$
Thus we obtain short exact sequences
$$
0 \to R^1\lim H^{i - 1}(U, \mathcal{F}/I^n\mathcal{F}) \to
H^i(R\Gamma(U, \mathcal{F})^\wedge) \to
\lim H^i(U, \mathcal{F}/I^n\mathcal{F}) \to 0
$$
by Cohomology, Lemma \ref{cohomology-lemma-RGamma-commutes-with-Rlim}.
The $R^1\lim$ terms vanish by the Mittag-Leffler condition
we just established.
\end{proof}

\begin{lemma}
\label{lemma-local-cohomology-derived-completion}
Let $(A, \mathfrak m)$ be a Noetherian local ring complete with
respect to an ideal $I \subset A$. Let $M$ be a finite $A$-module.
Then
$$
H^i(R\Gamma_\mathfrak m(M)^\wedge) = \lim H^i_\mathfrak m(M/I^nM)
$$
for all $i$ where $R\Gamma_\mathfrak m(M)^\wedge$ denotes
the derived $I$-adic completion.
\end{lemma}

\begin{proof}
Set $U = \Spec(A) \setminus \{\mathfrak m\}$ and denote
$\mathcal{F}$ the coherent $\mathcal{O}_U$-module corresponding to $M$.
We will use the results of Lemma \ref{lemma-compare-with-derived-completion}
without further mention. The distinguished triangle
$$
R\Gamma_\mathfrak m(M) \to M \to R\Gamma(U, \mathcal{F}) \to
R\Gamma_\mathfrak m(M)[1]
$$
is transformed into another distinguished triangle by
the exact functor of derived completion.
Since $A$ is $I$-adically complete, so is $M$, and hence
the derived completion $M^\wedge$ is equal to $M$.
Thus we find a short exact sequence
$$
0 \to H^0(R\Gamma_\mathfrak m(M)^\wedge) \to M \to
\lim H^0(U, \mathcal{F}/I^n\mathcal{F}) \to
H^1(R\Gamma_\mathfrak m(M)^\wedge) \to 0
$$
and isomorphisms
$\lim H^{i - 1}(U, \mathcal{F}/I^n\mathcal{F}) =
H^i(R\Gamma_\mathfrak m(M)^\wedge)$ for $i \geq 2$.
This proves the lemma holds for $i \geq 2$ since
we have $H^{i - 1}(U, \mathcal{F}/I^n\mathcal{F}) =
H^i_\mathfrak m(M)$ in this case.

\medskip\noindent
For $i = 0, 1$ we compare the sequence above with the exact sequences
$$
0 \to H^0_\mathfrak m(M/I^nM) \to M/I^nM \to
H^0(U, \mathcal{F}/I^n\mathcal{F}) \to
H^1_\mathfrak m(M/I^nM) \to 0
$$
where we have the Mittag-Leffler condition in each spot.
Since we have that the limit of the middle two systems
gives the middle two modules in the sequence of the previous
paragraph we conclude.
\end{proof}

\begin{lemma}
\label{lemma-kill-completion}
Let $(A, \mathfrak m)$ be a Noetherian local ring.
Let $I \subset A$ be an ideal. Let $M$ be a finite $A$-module and
let $\mathfrak p \subset A$ be a prime. Let $s$ and $d$ be integers. Assume
\begin{enumerate}
\item $A$ has a dualizing complex,
\item $\text{cd}(A, I) \leq d$, and
\item
$\text{depth}_{A_\mathfrak p}(M_\mathfrak p) + \dim(A/\mathfrak p) > d + s$.
\end{enumerate}
Then there exists an $f \in A \setminus \mathfrak p$ which annihilates
$H^i(R\Gamma_\mathfrak m(M)^\wedge)$ for $i \leq s$ where ${}^\wedge$
indicates $I$-adic completion.
\end{lemma}

\begin{proof}
Recall that
$$
R\Gamma_\mathfrak m(M)^\wedge = R\Hom_A(R\Gamma_I(A), R\Gamma_\mathfrak m(M))
$$
by the description of derived completion in
More on Algebra, Lemma \ref{more-algebra-lemma-derived-completion}
combined with the description of local cohomology in
Dualizing Complexes, Lemma
\ref{dualizing-lemma-compute-local-cohomology-noetherian}.
Assumption (2) means that $R\Gamma_I(A)$ has nonzero cohomology
only in degrees $\leq d$. Using the canonical truncations of
$R\Gamma_I(A)$ we find it suffices to show that
$$
\text{Ext}^i(N, R\Gamma_\mathfrak m(M))
$$
is annihilated by an $f \in A \setminus \mathfrak p$ for
$i \leq s + d$ and any $A$-module $N$.
In turn using the canonical truncations for
$R\Gamma_\mathfrak m(M)$ we see that it suffices to show
$H^i_\mathfrak m(M)$ is annihilated by an $f \in A \setminus \mathfrak p$
for $i \leq s + d$. This follows from Lemma \ref{lemma-sitting-in-degrees}.
\end{proof}

\begin{lemma}
\label{lemma-kill-local}
Let $(A, \mathfrak m)$ be a Noetherian local ring.
Let $I \subset A$ be an ideal. Let $M$ be a finite $A$-module.
Let $s$ be an integer. Assume
\begin{enumerate}
\item $A$ has a dualizing complex,
\item if $\mathfrak p \not \in V(I)$ and
$V(\mathfrak p) \cap V(I) \not = \{\mathfrak m\}$, then
$\text{depth}_{A_\mathfrak p}(M_\mathfrak p) + \dim(A/\mathfrak p) > s$.
\end{enumerate}
Then there exists an $n > 0$ and an ideal $J \subset A$
with $V(J) \cap V(I) = \{\mathfrak m\}$ such that $JI^n$ annihilates
$H^i_\mathfrak m(M)$ for $i \leq s$.
\end{lemma}

\begin{proof}
According to Lemma \ref{lemma-sitting-in-degrees}
we have to show this for the finite $A$-module
$E^i = \text{Ext}^{-i}_A(M, \omega_A^\bullet)$
for $i \leq s$. The support $Z$ of $E^0 \oplus \ldots \oplus E^s$
is closed in $\Spec(A)$ and does not contain any prime as in (4).
Hence it is contained in $V(JI^n)$ for some $J$ as in
the statement of the lemma.
\end{proof}

\begin{lemma}
\label{lemma-kill-colimit-weak}
Let $(A, \mathfrak m)$ be a Noetherian local ring.
Let $I \subset A$ be an ideal. Let $M$ be a finite $A$-module.
Let $s$ and $d$ be integers. Assume
\begin{enumerate}
\item $A$ has a dualizing complex,
\item if $\mathfrak p \in V(I) \setminus \{\mathfrak m\}$, then no condition,
\item if $\mathfrak p \not \in V(I)$ and
$V(\mathfrak p) \cap V(I) = \{\mathfrak m\}$, then
$\dim(A/\mathfrak p) \leq d$,
\item if $\mathfrak p \not \in V(I)$ and
$V(\mathfrak p) \cap V(I) \not = \{\mathfrak m\}$, then
$$
\text{depth}_{A_\mathfrak p}(M_\mathfrak p) \geq s
\quad\text{or}\quad
\text{depth}_{A_\mathfrak p}(M_\mathfrak p) + \dim(A/\mathfrak p) > d + s
$$
\end{enumerate}
Then there exists an ideal $J_0 \subset A$ with
$V(J_0) \cap V(I) = \{\mathfrak m\}$ such that for any $J \subset J_0$ with
$V(J) \cap V(I) = \{\mathfrak m\}$ the map
$$
R\Gamma_J(M) \longrightarrow R\Gamma_{J_0}(M)
$$
induces an isomorphism in cohomology in degrees $\leq s$
and moreover these modules are annihilated by a power of $J_0I$.
\end{lemma}

\begin{proof}
Let us consider the set
$$
B = \{\mathfrak p \not \in V(I) \text{ with }
V(\mathfrak p) \cap V(I) = \{\mathfrak m\} \text{ and }
\text{depth}(M_\mathfrak p) \leq s\}
$$
Let $V(J_0) \subset \Spec(A)$ be the closure of $B$, except
if $B = \emptyset$ we take $J_0 = \mathfrak m$.

\medskip\noindent
Claim I: $V(J_0) \cap V(I) = \{\mathfrak m\}$.

\medskip\noindent
Proof of Claim I. Let $\mathfrak p$ be a minimal prime of $V(J_0)$.
If $\mathfrak p \in B$, then $V(\mathfrak p) \cap V(I) = \{\mathfrak m\}$
as desired. If $\mathfrak p \not \in B$, then
$B \cap V(\mathfrak p)$ is dense, hence infinite, and we conclude that
$\text{depth}(M_\mathfrak p) < s$ by Lemma \ref{lemma-depth-function}.
On the other hand, every $\mathfrak p' \in B$ satisfies
$\text{depth}(M_{\mathfrak p'}) + \dim(A/\mathfrak p') \leq d + s$
by (3) and the definition of $B$. Since the set of such primes is
closed by Lemma \ref{lemma-sitting-in-degrees} we conclude that
$\text{depth}(M_\mathfrak p) + \dim(A/\mathfrak p) \leq d + s$.
Also $\mathfrak p \not \in V(I)$. Hence by (4) we see that
$V(\mathfrak p) \cap V(I) = \{\mathfrak m\}$. This finishes the
proof of Claim I.

\medskip\noindent
Claim II: $H^i_{J_0}(M) \to H^i_J(M)$ is an isomorphism for $i \leq s$
and $J \subset J_0$ with $V(J) \cap V(I) = \{\mathfrak m\}$.

\medskip\noindent
Proof of claim II. Choose $\mathfrak p \in V(J)$ not in $V(J_0)$.
It suffices to show that
$H^i_{\mathfrak pA_\mathfrak p}(M_\mathfrak p) = 0$, see
Lemma \ref{lemma-isomorphism}.
Since $\mathfrak p$ is not in $B$ we see that either
$\text{depth}(M_\mathfrak p) > s$ and then the group
vanishes or $\text{depth}(M_\mathfrak p) + \dim(A/\mathfrak p) > d + s$.
However, by condition (3) we find this also implies
$\text{depth}(M_\mathfrak p) > s$ and we conclude again.

\medskip\noindent
Claim III. $H^i_J(M)_f$ is a finite $A_f$-module for
$f \in I$ and $i \leq s$.

\medskip\noindent
Proof of Claim III. We will check the conditions of
Theorem \ref{theorem-finiteness} for $J_f \subset A_f$ and
the module $M_f$. Let $\mathfrak p \subset \mathfrak q$ be primes
such that $\mathfrak p \not \in V(J)$ and
$\mathfrak q \in V(J)$ but $f \not \in \mathfrak q$. We have to show that
$$
\text{depth}(M_\mathfrak p) + \dim((A/\mathfrak p)_\mathfrak q) > s
$$
Note that $\mathfrak p \not \in V(I)$ since $f \not \in \mathfrak q$.
If $V(\mathfrak p) \cap V(I) = \{\mathfrak m\}$
then $\text{depth}(M_\mathfrak p) > s$ because
$\mathfrak p \not \in B$. Hence the desired inequality.
If $\mathfrak p \not \in V(I)$ and
$V(\mathfrak p) \cap V(I) \not = \{\mathfrak m\}$, then
there are two cases: either $\text{depth}(M_\mathfrak p) \geq s$
and then the desired inequality follows as
$\dim((A/\mathfrak p)_\mathfrak q) \geq 1$ or
$\text{depth}_{A_\mathfrak p}(M_\mathfrak p) + \dim(A/\mathfrak p) > d + s$
which combined with $\dim(A/\mathfrak q) \leq d$ assumed in (3)
gives the desired inequality as $A$ is catenary:
$\dim((A/\mathfrak p)_\mathfrak q) = \dim(A/\mathfrak p) - \dim(A/\mathfrak q)
\geq \dim(A/\mathfrak p) - d$.

\medskip\noindent
Claim IV: The final statement of the lemma is true.

\medskip\noindent
Proof of Claim IV. Set $J = J_0$. Write $I = (f_1, \ldots, f_r)$.
By Claim III and
Lemma \ref{lemma-check-finiteness-local-cohomology-by-annihilator}
the modules $H^i_J(M)_{f_j}$ are finite and annihilated by
$J^t$ for some $t > 0$ (pick the same $t$ for all $j$).
Hence there exists a finite $A$-submodule $N^i \subset H^i_J(M)$
annihilated by $J^t$ inducing isomorphisms $N^i_{f_j} = H^i_J(M)_{f_j}$
for $j = 1, \ldots, r$. Then $H^i_J(M)/N^i$ is $I$-power torsion.
Hence $J^t H^i_J(M)$ is $I$-power torsion. To finish the proof
it suffices to show that $H^0_I(H^i_J(M))$ is annihilated by
$J' I^n$ for some $n > 0$ and some ideal $J' \subset J = J_0$ with
$V(J') \cap V(I) = \{\mathfrak m\}$. Namely, then the final
statement holds with $J_0$ replaced by $J'$. We will use without
further mention that the collection of ideals
$J' \subset J$ with $V(J') \cap V(I) = \{\mathfrak m\}$
ordered by inclusion is a directed set.
Consider the spectral sequence
$$
E_2^{p, q} = H^p_I(H^q_J(M)) \Rightarrow H^{a + b}_\mathfrak m(M)
$$
By Lemma \ref{lemma-kill-local} we find $J' \subset J$
with $V(J') \cap V(I) = \{\mathfrak m\}$ and an integer $n > 0$
such that a power of $J'I$ annihilates $H^i_\mathfrak m(M)$
for $i \leq s$. By induction on $i$ we may assume that a power
of $J' I$ annihilates $H^q_J(M)$ for $q < i$.
The spectral sequence shows there is a map
$$
H^i_\mathfrak m \to H^0_I(H^i_J(M))
$$
whose image the intersection of the kernels of the differentials
(each defined on the kernel of the previous one). These
differentials map into subquotients of
$H^2_I(H^{i - 1}_J(M)), H^3_I(H^{i - 2}_J(M)), \ldots$
which are killed by a power of $J''I$ for a suitable $J''$
by induction. Putting everything together we conclude.
\end{proof}

\begin{lemma}
\label{lemma-kill-colimit}
In Lemma \ref{lemma-kill-colimit-weak} if in stead of the empty
condition (2) we assume
\begin{enumerate}
\item[(2')] if $\mathfrak p \in V(I) \setminus \{\mathfrak m\}$, then
$\text{depth}_{A_\mathfrak p}(M_\mathfrak p) + \dim(A/\mathfrak p) > s$,
\end{enumerate}
then the conditions also imply that $H^i_{J_0}(M)$ is a finite
$A$-module for $i \leq s$.
\end{lemma}

\begin{proof}
We will use the construction of $J_0$ using the set $B$ from
the proof of Lemma \ref{lemma-kill-colimit-weak}.
We will check the conditions of Theorem \ref{theorem-finiteness}.
Let $\mathfrak p \subset \mathfrak q$ be primes
such that $\mathfrak p \not \in V(J_0)$ and
$\mathfrak q \in V(J_0)$. We have to show that
$$
\text{depth}(M_\mathfrak p) + \dim((A/\mathfrak p)_\mathfrak q) > s
$$
If $\mathfrak p \in V(I)$, then $\mathfrak q = \mathfrak m$ because
$V(J_0) \cap V(I) = \{\mathfrak m\}$. Hence we see that the desired
inequality follows from (2).
If $\mathfrak p \not \in V(I)$ and $V(\mathfrak p) \cap V(I) = \{\mathfrak m\}$
then $\text{depth}(M_\mathfrak p) > s$ because
$\mathfrak p \not \in B$. Hence the desired inequality.
Finally, if $\mathfrak p \not \in V(I)$ and
$V(\mathfrak p) \cap V(I) \not = \{\mathfrak m\}$, then
there are two cases: either $\text{depth}(M_\mathfrak p) \geq s$
and then the desired inequality follows as
$\dim((A/\mathfrak p)_\mathfrak q) \geq 1$ or
$\text{depth}_{A_\mathfrak p}(M_\mathfrak p) + \dim(A/\mathfrak p) > d + s$
which combined with $\dim(A/\mathfrak q) \leq d$ assumed in (3)
and gives the desired inequality as $A$ is catenary:
$\dim((A/\mathfrak p)_\mathfrak q) = \dim(A/\mathfrak p) - \dim(A/\mathfrak q)
\geq \dim(A/\mathfrak p) - d$.
\end{proof}

\begin{lemma}
\label{lemma-algebraize-local-cohomology}
Let $(A, \mathfrak m)$ be a Noetherian local ring.
Let $I \subset A$ be an ideal. Let $M$ be a finite $A$-module.
Let $s$ and $d$ be integers. Assume
\begin{enumerate}
\item $A$ is $I$-adically complete and has a dualizing complex,
\item if $\mathfrak p \in V(I) \setminus \{\mathfrak m\}$, no condition,
\item $\text{cd}(A, I) \leq d$,
\item if $\mathfrak p \not \in V(I)$ and
$V(\mathfrak p) \cap V(I) \not = \{\mathfrak m\}$ then
$$
\text{depth}_{A_\mathfrak p}(M_\mathfrak p) \geq s
\quad\text{or}\quad
\text{depth}_{A_\mathfrak p}(M_\mathfrak p) + \dim(A/\mathfrak p) > d + s
$$
\end{enumerate}
Then there exists an ideal $J_0 \subset A$ with
$V(J_0) \cap V(I) = \{\mathfrak m\}$ such that for any $J \subset J_0$ with
$V(J) \cap V(I) = \{\mathfrak m\}$ the map
$$
R\Gamma_J(M) \longrightarrow
R\Gamma_J(M)^\wedge = R\Gamma_\mathfrak m(M)^\wedge
$$
induces an isomorphism in cohomology in degrees $\leq s$.
Here ${}^\wedge$ denotes derived $I$-adic completion.
\end{lemma}

\begin{proof}
There is no difference between $R\Gamma_\mathfrak a$ and
$R\Gamma_{V(\mathfrak a)}$ in our current situation, see
Dualizing Complexes, Lemma \ref{dualizing-lemma-local-cohomology-noetherian}.
Next, we observe that
$$
R\Gamma_\mathfrak m(M)^\wedge =
R\Gamma_I(R\Gamma_J(M))^\wedge =
R\Gamma_J(M)^\wedge
$$
by Dualizing Complexes, Lemmas \ref{dualizing-lemma-local-cohomology-ss} and
\ref{dualizing-lemma-complete-and-local}
which explains the equality sign in the statement of the lemma.

\medskip\noindent
Suppose $M$ is a finite $A$-module with
$\text{Supp}(M) \subset V(J) \cup V(I)$ for some ideal $J$
with $V(J) \cap V(I) = \{\mathfrak m\}$.
Then we claim that $R\Gamma_J(M) \to R\Gamma_\mathfrak m(M)^\wedge$
is an isomorphism.
Namely, for any such module there is a short exact sequence
$0 \to M_1 \oplus M_2 \to M \to N \to 0$ with
$M_1$ annihilated by a power of $J$, with $M_2$ annihilated
by a power of $I$ and with $N$ annihilated by a power of $\mathfrak m$.
In the case of $M_1$ we see that $R\Gamma_J(M_1) = M_1$ and
since $M_1$ is a finite $A$-module and $I$-adically complete
we have $M_1^\wedge = M_1$.
In the case of $M_2$ we see that $H^i_J(M_2)$ is annihilated
by a power of $I$ and hence derived complete. Thus
$R\Gamma_J(M_2) = R\Gamma_J(M_2)^\wedge$ as desired.

\medskip\noindent
Next, let $M$ be as in the statement of the lemma.
Observe that the lemma holds for $s < 0$. This is not a trivial case because
it is not a priori clear that $H^s(R\Gamma_\mathfrak m(M)^\wedge)$
is zero for negative $s$. However, this vanishing was esthablished
in Lemma \ref{lemma-local-cohomology-derived-completion}.
We will prove the lemma by induction for $s \geq 0$.

\medskip\noindent
Let $M' \subset M$ be the submodule of elements whose
support is condained in $V(I) \cup V(J)$ for some
ideal $J$ with $V(J) \cap V(I) = \{\mathfrak m\}$.
Then $M'$ is finite and the result holds for $M'$ by
the second paragraph of the proof. Moreover, condition (4)
for $M$ is inherited by $M/M'$. After replacing $M$ by
$M/M'$ we may assume that $\text{Ass}(M)$ consists
of primes as in (4).

\medskip\noindent
The assumptions of Lemma \ref{lemma-kill-colimit-weak}
are satisfied by Lemma \ref{lemma-cd-bound-dim-local}.
Thus we may and do choose an ideal $J_0$ as in the lemma
and an integer $t > 0$ such that $(J_0I)^t$ annihilates $H^s_J(M)$.
The assumptions of Lemma \ref{lemma-kill-completion}
are satisfied for every $\mathfrak p \in \text{Ass}(M)$
(by our mangling of $M$ above).
Thus the annihilator $\mathfrak a \subset A$ of
$H^s(R\Gamma_\mathfrak m(M)^\wedge)$
is not contained in $\mathfrak p$ for $\mathfrak p \in \text{Ass}(M)$.
Thus we can find an $f \in \mathfrak a(J_0I)^t$
not in any associated prime of $M$ which is an annihilator
of both $H^s(R\Gamma_\mathfrak m(M)^\wedge)$ and $H^s_J(M)$.
Then $f$ is a nonzerodivisor on $M$ and we can consider the
short exact sequence
$$
0 \to M \xrightarrow{f} M \to M/fM \to 0
$$
Our choice of $f$ shows that we obtain
$$
\xymatrix{
H^{s - 1}_J(M) \ar[d] \ar[r] &
H^{s - 1}_J(M/fM) \ar[d] \ar[r] &
H^s_J(M) \ar[d] \ar[r] & 0 \\
H^{s - 1}(R\Gamma_\mathfrak m(M)^\wedge) \ar[r] &
H^{s - 1}(R\Gamma_\mathfrak m(M/fM)^\wedge) \ar[r] &
H^s(R\Gamma_\mathfrak m(M)^\wedge) \ar[r] & 0
}
$$
for any $J \subset J_0$ with $V(J) \cap V(I) = \{\mathfrak m\}$.
Thus if we choose $J$ such that it works for
$M$ and $M/fM$ and $s - 1$ (possible by induction hypothesis),
then we conclude that the lemma is true.
\end{proof}

\noindent
The lemma above is the main result of this section.
We can reformulate it in terms of cohomology of the
punctured spectrum as follows.

\begin{theorem}
\label{theorem-algebraization-formal-sections}
\begin{reference}
The method of proof follows roughly the method of
proof of \cite[Theorem 1]{Faltings-algebraisation}
and \cite[Satz 2]{Faltings-uber}.
The result is almost the same as
\cite[Theorem 1.1]{MRaynaud-paper} (affine complement case) and
\cite[Theorem 3.9]{MRaynaud-book} (complement is union of few affines).
\end{reference}
Let $(A, \mathfrak m)$ be a Noetherian local ring which has a
dualizing complex and is complete with respect to an ideal $I$.
Set $X = \Spec(A)$, $Y = V(I)$, and $U = X \setminus \{\mathfrak m\}$.
Let $\mathcal{F}$ be a coherent sheaf on $U$.
Assume
\begin{enumerate}
\item $\text{cd}(A, I) \leq d$, i.e.,
$H^i(X \setminus Y, \mathcal{G}) = 0$ for $i \geq d$ and
quasi-coherent $\mathcal{G}$ on $X$,
\item for any $x \in X \setminus Y$ whose closure $\overline{\{x\}}$
in $X$ meets $Y \cap U$ we have
$$
\text{depth}_{\mathcal{O}_{X, x}}(\mathcal{F}_x) \geq s
\quad\text{or}\quad
\text{depth}_{\mathcal{O}_{X, x}}(\mathcal{F}_x)
+ \dim(\overline{\{x\}}) > d + s
$$
\end{enumerate}
Then there exists an open $V_0 \subset U$ containing $Y \cap U$
such that for any open $V_0 \subset V \subset U$ containing $Y \cap U$
the map
$$
H^i(V, \mathcal{F}) \to \lim H^i(U, \mathcal{F}/I^n\mathcal{F})
$$
is an isomorphism for $i < s$. If in addition
$
\text{depth}_{\mathcal{O}_{X, x}}(\mathcal{F}_x) +
\dim(\overline{\{x\}}) > s
$
for all $x \in Y \cap U$, then these cohomology groups are finite $A$-modules.
\end{theorem}

\begin{proof}
Choose a finite $A$-module $M$ such that $\mathcal{F}$ is the
restriction to $U$ of the
coherent $\mathcal{O}_X$-module associated to $M$, see
Lemma \ref{lemma-finiteness-pushforwards-and-H1-local}.
Then the assumptions of
Lemma \ref{lemma-algebraize-local-cohomology}
are satisfied.
Pick $J_0$ as in that lemma and set $V_0 = X \setminus V(J_0)$.
Then opens $V_0 \subset V \subset U$ containing $Y \cap U$
correspond $1$-to-$1$ with ideals $J \subset J_0$ with
$V(J) \cap V(I) = \{\mathfrak m\}$.
Moreover, for such a choice we have a distinguished triangle
$$
R\Gamma_J(M) \to M \to R\Gamma(V, \mathcal{F}) \to
R\Gamma_J(M)[1]
$$
We similarly have a distinguished triangle
$$
R\Gamma_\mathfrak m(M)^\wedge \to
M \to
R\Gamma(U, \mathcal{F})^\wedge \to
R\Gamma_\mathfrak m(M)^\wedge[1]
$$
involving derived $I$-adic completions,
see proof of Lemma \ref{lemma-local-cohomology-derived-completion}.
The cohomology groups of $R\Gamma(U, \mathcal{F})^\wedge$ are
equal to the limits in the statement of the theorem by
Lemma \ref{lemma-compare-with-derived-completion}.
The canonical map between these triangles
and some easy arguments show that our
theorem follows from the main Lemma \ref{lemma-algebraize-local-cohomology}
(note that we have $i < s$ here whereas we have
$i \leq s$ in the lemma; this is because of the shift).
The finiteness of the cohomology groups
(under the additional assumption) follows from
Lemma \ref{lemma-kill-colimit}.
\end{proof}

\begin{lemma}
\label{lemma-application-theorem}
Let $(A, \mathfrak m)$ be a Noetherian local ring which has a
dualizing complex and is complete with respect to an ideal $I$.
Set $X = \Spec(A)$, $Y = V(I)$, and $U = X \setminus \{\mathfrak m\}$.
Let $\mathcal{F}$ be a coherent sheaf on $U$.
Assume
\begin{enumerate}
\item $\text{cd}(A, I) \leq d$,
\item for any $x \in U$ which is an associated point of $\mathcal{F}$
we have $\dim(\overline{\{x\}}) > d + 1$.
\end{enumerate}
Then the map
$$
\colim H^0(V, \mathcal{F})
\longrightarrow
\lim H^0(U, \mathcal{F}/I^n\mathcal{F})
$$
is an isomorphism of finite $A$-modules
where the colimit is over opens $V \subset U$
containing $Y \cap U$.
\end{lemma}

\begin{proof}
Apply Theorem \ref{theorem-algebraization-formal-sections} with $s = 1$
(we get finiteness too).
\end{proof}






\section{Algebraization of coherent formal modules}
\label{section-algebraization-modules}

\noindent
Let $(A, \mathfrak m)$ be a Noetherian local ring.
Let $I \subset A$ be an ideal. Let
$$
X = \Spec(A) \supset U = \Spec(A) \setminus \{\mathfrak m\}
$$
and denote $Y = V(I)$ the closed subscheme corresponding to $I$.
In this section we consider inverse systems of coherent
$\mathcal{O}_U$-modules $(\mathcal{F}_n)$ with $\mathcal{F}_n$
annihilated by $I^n$ such that the transition maps induce
isomorphisms $\mathcal{F}_{n + 1}/I^n\mathcal{F}_{n + 1} \to \mathcal{F}_n$.
The category of these systems was denoted
$$
\textit{Coh}(U, I\mathcal{O}_U)
$$
in Cohomology of Schemes, Section \ref{coherent-section-existence}.
This category is equivalent to the category of coherent modules
on the formal completion of $U$ along $Y$; however, since we have
not yet introduced formal schemes or coherent modules on them,
we cannot use this terminology here.

\begin{lemma}
\label{lemma-system-of-modules}
With $A, \mathfrak m, I, X, U$ as above.
Consider an inverse system $(M_n)$ of finite $A$-modules such
that $M_n$ is annihilated by $I^n$ and the kernel and cokernel of
$M_{n + 1}/I^nM_{n + 1} \to M_n$ have finite length.
Then $\widetilde{M}_n|_U$ is in $\textit{Coh}(U, I\mathcal{O}_U)$.
Conversely, every object of $\textit{Coh}(U, I\mathcal{O}_U)$
is of this form.
\end{lemma}

\begin{proof}
Omitted, but see Lemma \ref{lemma-finiteness-pushforwards-and-H1-local}.
\end{proof}

\noindent
If $A$ is $I$-adically complete, then an important question
is whether the completion functor Cohomology of Schemes,
Equation (\ref{coherent-equation-completion-functor})
$$
\textit{Coh}(\mathcal{O}_U)
\longrightarrow
\textit{Coh}(U, I\mathcal{O}_U),\quad
\mathcal{F} \longmapsto \mathcal{F}^\wedge
$$
is essentially surjective. Fully faithfullness of this functor
is often a consequence of the results in
Section \ref{section-algebraization-sections}
applied to suitable $\SheafHom$'s.
The essential surjectivity of the completion functor
was studied systematically in
\cite{SGA2}, \cite{MRaynaud-book}, and \cite{MRaynaud-paper}.
We will discuss this material (insert future reference here).
In this section we discuss only the case where the closed
subset $Y$ is cut out by a single nonzerodivisor and we only
deal with algebraization of formal vector bundles.

\begin{lemma}
\label{lemma-algebraization-principal}
With $A, \mathfrak m, I, X, U$ as above let
$(\mathcal{F}_n)$ be an object of $\textit{Coh}(U, I\mathcal{O}_U)$.
Assume
\begin{enumerate}
\item $A$ has a dualizing complex and is complete with respect to $I$,
\item $I = (f)$ is a principal ideal for a nonzerodivisor $f \in \mathfrak m$,
\item $\mathcal{F}_n$ is a finite locally free
$\mathcal{O}_U/f^n\mathcal{O}_U$-module,
\item if $\mathfrak p \in V(f) \setminus \{\mathfrak m\}$, then
$\text{depth}((A/f)_\mathfrak p) + \dim(A/\mathfrak p) > 1$, and
\item if $\mathfrak p \not \in V(f)$ and
$V(\mathfrak p) \cap V(f) \not = \{\mathfrak m\}$, then
$\text{depth}(A_\mathfrak p) + \dim(A/\mathfrak p) > 3$.
\end{enumerate}
Then there exists a coherent $\mathcal{O}_U$-module
$\mathcal{F}$ such that $(\mathcal{F}_n)$ is the completion of $\mathcal{F}$.
\end{lemma}

\begin{proof}
By induction on $n$ and the short exact sequences
$0 \to A/f^n \to A/f^{n + 1} \to A/f \to 0$ we see that
the associate primes of $A/f^nA$ agree with the associated
primes of $A/fA$. Since the associated points of $\mathcal{F}_n$
correspond to the associated primes of $A/f^nA$ not equal to $\mathfrak m$
by condition (3), we conclude that
$M_n = H^0(U, \mathcal{F}_n)$ is a finite $A$-module by
(4) and Proposition \ref{proposition-kollar}.

\medskip\noindent
We claim that for any $n > 0$ and $m \gg n$ the image of
$$
H^1(U, \mathcal{F}_m) \longrightarrow H^1(U, \mathcal{F}_n)
$$
has finite length as an $A$-module. The image is independent
of $m$ for $m$ large enough by Lemma \ref{lemma-ML-local}.
Let $\omega_A^\bullet$ be a normalized dualizing complex for $A$.
By the local duality theorem and Matlis duality
(Dualizing Complexes, Lemma \ref{dualizing-lemma-special-case-local-duality}
and Proposition \ref{dualizing-proposition-matlis})
our claim is equivalent to: the image of
$$
\text{Ext}^{-2}_A(M_n, \omega_A^\bullet) \to
\text{Ext}^{-2}_A(M_m, \omega_A^\bullet)
$$
has finite length for $m \gg n$. The modules in question are
finite $A$-modules supported at $V(f)$. Thus it suffices to show that this
map is zero after localization at a prime $\mathfrak q$
containing $f$ and different from $\mathfrak m$.
Let $\omega_{A_\mathfrak q}^\bullet$ be a normalized
dualizing complex on $A_\mathfrak q$ and recall that
$\omega_{A_\mathfrak q}^\bullet =
(\omega_A^\bullet)_\mathfrak q[\dim(A/\mathfrak q)]$ by
Dualizing Complexes, Lemma \ref{dualizing-lemma-dimension-function}.
Using the local structure of $\mathcal{F}_n$ given in (3)
we find that it suffices to show the vanishing of
$$
\text{Ext}^{-2 + \dim(A/\mathfrak q)}_{A_\mathfrak q}(
A_\mathfrak q/f^n, \omega_{A_\mathfrak q}^\bullet)
\to
\text{Ext}^{-2 + \dim(A/\mathfrak q)}_{A_\mathfrak q}(
A_\mathfrak q/f^m, \omega_{A_\mathfrak q}^\bullet)
$$
If $\dim(A/\mathfrak q) > 3$, then this is immediate from
Lemma \ref{lemma-sitting-in-degrees}. We will use the
long exact sequence
$$
\ldots
\xrightarrow{f^m}
H^{-1}(\omega_{A_\mathfrak q}^\bullet)
\to
\text{Ext}^{-1}_{A_\mathfrak q}(
A_\mathfrak q/f^m, \omega_{A_\mathfrak q}^\bullet) \to
H^0(\omega_{A_\mathfrak q}^\bullet)
\xrightarrow{f^m}
H^0(\omega_{A_\mathfrak q}^\bullet)
\to
\text{Ext}^0_{A_\mathfrak q}(
A_\mathfrak q/f^m, \omega_{A_\mathfrak q}^\bullet) \to 0
$$
If $\dim(A/\mathfrak q) = 2$, then
$H^0(\omega_{A_\mathfrak q}^\bullet) = 0$ as
the depth of $A_\mathfrak q$ is zero by dint of
$f$ being a nonzerodivisor.
Thus the long exact sequence shows the condition is that
$$
f^{m - n} :
H^{-1}(\omega_{A_\mathfrak q}^\bullet)/f^n \to
H^{-1}(\omega_{A_\mathfrak q}^\bullet)/f^m
$$
is zero. Now $H^{-1}(\omega^\bullet_\mathfrak q)$ is a finite
module supported in the primes $\mathfrak p \subset A_\mathfrak q$
such that $\text{depth}(A_\mathfrak p) + \dim((A/\mathfrak p)_\mathfrak q)
\leq 1$. By condition (5) all of these primes are contained in $V(f)$.
Thus the desired vanishing for $m$ large enough.
If $\dim(A/\mathfrak q) = 1$, then condition (4) combined
with the fact that $f$ is a nonzerodivisor
insures that $A_\mathfrak q$ has depth at least $2$. Hence
$H^0(\omega_{A_\mathfrak q}^\bullet) =
H^{-1}(\omega_{A_\mathfrak q}^\bullet) = 0$
and the long exact sequence shows the claim is
equivalent to the vanishing of
$$
f^{m - n} :
H^{-2}(\omega_{A_\mathfrak q}^\bullet)/f^n \to
H^{-2}(\omega_{A_\mathfrak q}^\bullet)/f^m
$$
Now $H^{-2}(\omega^\bullet_\mathfrak q)$ is a finite
module supported in the primes $\mathfrak p \subset A_\mathfrak q$
such that $\text{depth}(A_\mathfrak p) + \dim((A/\mathfrak p)_\mathfrak q)
\leq 2$. By condition (5) all of these primes are contained in $V(f)$.
Thus the desired vanishing for $m$ large enough proving the claim.

\medskip\noindent
By Lemmas \ref{lemma-limit-finite} and \ref{lemma-ML-better} 
the system of modules $(M_n)$ satisfies the Mittag-Leffler
condition, $M = \lim M_n$ is a finite $A$-module, $f$ is a
nonzerodivisor on $M$ and that $M/fM \subset M_1$. To finish the proof,
we will show that $M/f^nM \to M_n$ is an isomorphism after
localizing at any prime $\mathfrak q \in V(f)$,
$\mathfrak q \not = \mathfrak m$. Namely, by the Mittag-Leffler
condition, we know that $M/fM \subset M_1$ is the image of
$M_m \to M_1$ for some $m \gg 1$. Since the cokernel of
$M_m \to M_1$ is contained in $H^1(U, \mathcal{F}_{m -  1})$
which is $\mathfrak m$-power torsion, we conclude that
$M/fM \to M_1$ becomes an isomorphism after localizing at $\mathfrak q$.
Using induction and suitable short exact sequences the reader
conlcudes the same is true for $M/f^n M \to M_n$.
\end{proof}

\begin{remark}
\label{remark-interesting-case}
Let $(A, \mathfrak m)$ be a complete Noetherian normal local domain
of dimension $\geq 4$ and let $f \in \mathfrak m$ be nonzero.
Then assumptions (1), (2), (4), (5) of
Lemma \ref{lemma-algebraization-principal}
are satisfied. Thus vectorbundles
on the formal completion of $U$ along $U \cap V(f)$
can be algebraized.
\end{remark}

\begin{lemma}
\label{lemma-algebraization-principal-variant}
With $A, \mathfrak m, I, X, U$ as above let
$(\mathcal{F}_n)$ be an object of $\textit{Coh}(U, I\mathcal{O}_U)$.
Assume
\begin{enumerate}
\item $I = (f)$ is a principal ideal for a nonzerodivisor $f \in \mathfrak m$,
\item $A$ is complete with respect to $I = (f)$,
\item $\mathcal{F}_n$ is a finite locally free
$\mathcal{O}_U/f^n\mathcal{O}_U$-module,
\item $H^1_\mathfrak m(A/fA)$ and $H^2_\mathfrak m(A/fA)$
are finite $A$-modules.
\end{enumerate}
Then there exists a coherent $\mathcal{O}_U$-module
$\mathcal{F}$ such that $(\mathcal{F}_n)$ is the completion of $\mathcal{F}$.
\end{lemma}

\begin{proof}
This lemma is a variant of
Lemma \ref{lemma-algebraization-principal}
and if $A$ is a complete local ring, then it follows from that
lemma\footnote{Namely, the condition that
$H^1_\mathfrak m(A/fA)$ and $H^2_\mathfrak m(A/fA)$
are finite $A$-modules, is equivalent with
$\text{depth}((A/f)_\mathfrak q) + \dim(A/\mathfrak q) > 2$
for all $\mathfrak q \in V(f)$, $\mathfrak q \not = \mathfrak m$
by Theorem \ref{theorem-finiteness}. As $f$ is a nonzerodivisor
for such a prime
$\text{depth}(A_\mathfrak q) + \dim(A/\mathfrak q) > 3$. The locus
of these primes is open by Lemma \ref{lemma-sitting-in-degrees}.
Hence assumption (5) of Lemma \ref{lemma-algebraization-principal}
follows from condition (4) of this lemma.}.
We suggest the reader skip the proof.

\medskip\noindent
As $f$ is a nonzerodivisor we obtain short exact sequences
$$
0 \to A/f^nA \xrightarrow{f} A/f^{n + 1}A \to A/fA \to 0
$$
and we have corresponding short exact sequences
$0 \to \mathcal{F}_n \to \mathcal{F}_{n + 1} \to \mathcal{F}_1 \to 0$.
We will use Lemma \ref{lemma-finiteness-pushforwards-and-H1-local}
without further mention. Our assumptions imply that
$H^0(U, \mathcal{O}_U/f\mathcal{O}_U)$ and
$H^1(U, \mathcal{O}_U/f\mathcal{O}_U)$
are finite $A$-modules. Hence the same thing is true for $\mathcal{F}_1$, see
Lemma \ref{lemma-finiteness-for-finite-locally-free}.
Thus $H^0(U, \mathcal{F}_1)$ is a finite $A$-module
and $H^1(U, \mathcal{F}_1)$ has finite length
(as a finite $A$-module which is $\mathfrak m$-power torsion).
Thus Lemmas \ref{lemma-limit-finite} and
\ref{lemma-ML} apply to the system above. Setting
$M_n = \Gamma(U, \mathcal{F}_n)$ we find
the system of modules $(M_n)$ satisfies the Mittag-Leffler
condition, $M = \lim M_n$ is a finite $A$-module, $f$ is a
nonzerodivisor on $M$ and that $M/fM \subset M_1$. To finish the proof,
we will show that $M/f^nM \to M_n$ is an isomorphism after
localizing at any prime $\mathfrak q \in V(f)$,
$\mathfrak q \not = \mathfrak m$. Namely, by the Mittag-Leffler
condition, we know that $M/fM \subset M_1$ is the image of
$M_m \to M_1$ for some $m \gg 1$. Since the cokernel of
$M_m \to M_1$ is contained in $H^1(U, \mathcal{F}_{m -  1})$
which is $\mathfrak m$-power torsion, we conclude that
$M/fM \to M_1$ becomes an isomorphism after localizing at $\mathfrak q$.
Using induction and suitable short exact sequences the reader
conlcude the same is true for $M/f^n M \to M_n$.
\end{proof}









\input{chapters}

\bibliography{my}
\bibliographystyle{amsalpha}

\end{document}
