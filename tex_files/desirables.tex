\input{preamble}

% OK, start here.
%
\begin{document}

\title{Desirables}

\maketitle

\phantomsection
\label{section-phantom}


\tableofcontents

\section{Introduction}
\label{section-introduction}

\noindent
This is basically just a list of things that we want to put in the stacks
project. As we add material to the Stacks project continuously this is always
somewhat behind the current state of the Stacks project. In fact, it may have
been a mistake to try and list things we should add, because it seems
impossible to keep it up to date.

\medskip\noindent
Last updated: Thursday, August 31, 2017.


\section{Conventions}
\label{section-conventions}

\noindent
We should have a chapter with a short list of conventions used in the document.
This chapter already exists, see
Conventions, Section \ref{conventions-section-comments},
but a lot more could be added there. Especially useful would be to find
``hidden'' conventions and tacit assumptions and put those there.


\section{Sites and Topoi}
\label{section-sites}

\noindent
We have a chapter on sites and sheaves, see
Sites, Section \ref{sites-section-introduction}.
We have a chapter on ringed sites (and topoi) and modules on them, see
Modules on Sites, Section \ref{sites-modules-section-introduction}.
We have a chapter on cohomology in this setting, see
Cohomology on Sites, Section \ref{sites-cohomology-section-introduction}.
But a lot more could be added, especially in the chapter on cohomology.


\section{Stacks}
\label{section-stacks}

\noindent
We have a chapter on (abstract) stacks, see
Stacks, Section \ref{stacks-section-introduction}.
It would be nice if
\begin{enumerate}
\item improve the discussion on ``stackyfication'',
\item give examples of stackyfication,
\item more examples in general,
\item improve the discussion of gerbes.
\end{enumerate}
Example result which has not been added yet: Given a sheaf of abelian
groups $\mathcal{F}$
over $\mathcal{C}$ the set of equivalence classes of gerbes banded by
$\mathcal{F}$ is bijective to $H^2(\mathcal{C}, \mathcal{F})$.


\section{Simplicial methods}
\label{section-simplicial}

\noindent
We have a chapter on simplicial methods, see
Simplicial, Section \ref{simplicial-section-introduction}.
This has to be reviewed and improved. The discussion of
the relationship between simplicial homotopy (also known as
combinatorial homotopy) and Kan complexes should be improved upon.
There is a chapter on simplicial spaces, see
Simplicial Spaces, Section \ref{spaces-simplicial-section-introduction}.
This chapter briefly discusses
simplicial topological spaces, simplicial sites, and simplicial topoi.
We can further develop ``simplicial algebraic geometry'' to discuss
simplicial schemes (or simplicial algebraic spaces, or
simplicial algebraic stacks) and treat geometric questions, their cohomology,
etc.


\section{Cohomology of schemes}
\label{section-schemes-cohomology}

\noindent
There is already a chapter on cohomology of quasi-coherent sheaves, see
Cohomology of Schemes, Section \ref{coherent-section-introduction}.
We have a chapter discussing the derived category of
quasi-coherent sheaves on a scheme, see
Derived Categories of Schemes, Section \ref{perfect-section-introduction}.
We have a chapter discussing duality for Noetherian schemes
and relative duality for morphisms of schemes, see
Duality for Schemes, Section \ref{duality-section-introduction}.
We also have chapters on \'etale cohomology of schemes and on
crystalline cohomology of schemes. But most of the material in these
chapters is very basic and a lot more could/should be added there.


\section{Deformation theory \`a la Schlessinger}
\label{section-deformation-schlessinger}

\noindent
We have a chapter on this material, see
Formal Deformation Theory, Section \ref{formal-defos-section-introduction}.
We have a chapter discussing examples of the general theory, see
Deformation Problems, Section \ref{examples-defos-section-introduction}.
We have a chapter, see
Deformation Theory, Section \ref{defos-section-introduction}
which discusses deformations of rings (and modules),
deformations of ringed spaces (and sheaves of modules),
deformations of ringed topoi (and sheaves of modules).
In this chapter we use the naive cotangent complex
to describe obstructions, first order deformations, and
infinitesimal automorphisms. This material has found some
applications to algebraicity of moduli stacks in later chapters.
There is also a chapter discussing the full cotangent complex, see
Cotangent, Section \ref{cotangent-section-introduction}.


\section{Definition of algebraic stacks}
\label{section-definition-algebraic-stacks}

\noindent
An algebraic stack is a stack in groupoids over the category of schemes
with the fppf topology that has a diagonal representable by algebraic
spaces and is the target of a surjective smooth morphism from a scheme.
See Algebraic Stacks, Section \ref{algebraic-section-algebraic-stacks}.
A ``Deligne-Mumford stack'' is an algebraic stack for which there exists
a scheme and a surjective \'etale morphism from that scheme to it
as in the paper \cite{DM} of Deligne and Mumford, see
Algebraic Stacks, Definition \ref{algebraic-definition-deligne-mumford}.
We will reserve the term ``Artin stack'' for a stack such as in the papers by
Artin, see \cite{ArtinI}, \cite{ArtinII}, and \cite{ArtinVersal}.
A possible definition is that an Artin stack is an algebraic stack
$\mathcal{X}$ over a locally Noetherian scheme $S$ such that
$\mathcal{X} \to S$ is
locally of finite type\footnote{Namely, these are exactly the algebraic
stacks over $S$ satisfying Artin's axioms [-1], [0], [1], [2], [3], [4], [5]
of Artin's Axioms, Section \ref{artin-section-axioms}.}.


\section{Examples of schemes, algebraic spaces, algebraic stacks}
\label{section-examples-stacks}

\noindent
The Stacks project currently contains two chapters discussing
moduli stacks and their properties, see
Moduli Stacks, Section \ref{moduli-section-introduction} and
Moduli of Curves, Section \ref{moduli-curves-section-introduction}.
Over time we intend to add more, for example:
\begin{enumerate}
\item $\mathcal{A}_g$,
i.e., principally polarized abelian schemes of genus $g$,
\item $\mathcal{A}_1 = \mathcal{M}_{1, 1}$, i.e.,
$1$-pointed smooth projective genus $1$ curves,
\item $\mathcal{M}_{g, n}$, i.e., smooth projective genus $g$-curves
with $n$ pairwise distinct labeled points,
\item $\overline{\mathcal{M}}_{g, n}$, i.e.,
stable $n$-pointed nodal projective genus $g$-curves,
\item $\SheafHom_S(\mathcal{X}, \mathcal{Y})$, moduli of morphisms
(with suitable conditions on the stacks $\mathcal{X}$, $\mathcal{Y}$
and the base scheme $S$),
\item $\textit{Bun}_G(X) = \SheafHom_S(X, BG)$, the stack of $G$-bundles
of the geometric Langlands programme (with suitable conditions on the scheme
$X$, the group scheme $G$, and the base scheme $S$),
\item $\Picardstack_{\mathcal{X}/S}$, i.e., the Picard stack associated
to an algebraic stack over a base scheme (or space).
\end{enumerate}
More generally, the Stacks project is somewhat
lacking in geometrically meaningful examples.


\section{Properties of algebraic stacks}
\label{section-stacks-properties}

\noindent
This is perhaps one of the easier projects to work on, as most of the
basic theory is there now. Of course these things are really properties
of morphisms of stacks. We can define singularities (up to smooth factors)
etc. Prove that a connected normal stack is irreducible, etc.


\section{Lisse \'etale site of an algebraic stack}
\label{section-lisse-etale}

\noindent
This has been introduced in
Cohomology of Stacks, Section \ref{stacks-cohomology-section-lisse-etale}.
An example to show that it is not functorial with respect to $1$-morphisms
of algebraic stacks is discussed in
Examples, Section \ref{examples-section-lisse-etale-not-functorial}.
Of course a lot more could be said about this, but it turns out
to be very useful to prove things using the ``big'' \'etale site
as much as possible.



\section{Things you always wanted to know but were afraid to ask}
\label{section-stacks-fun-lemmas}

\noindent
There are going to be lots of lemmas that you use over and over again
that are useful but aren't really mentioned specifically in the literature,
or it isn't easy to find references for. Bag of tricks.

\medskip\noindent
Example: Given two groupoids in schemes $R\Rightarrow U$ and
$R' \Rightarrow U'$ what does it mean to have a $1$-morphism
$[U/R] \to [U'/R']$ purely in terms of groupoids in schemes.



\section{Quasi-coherent sheaves on stacks}
\label{section-quasi-coherent}

\noindent
These are defined and discussed in the chapter
Cohomology of Stacks, Section \ref{stacks-cohomology-section-introduction}.
Derived categories of modules are discussed in the chapter
Derived Categories of Stacks, Section \ref{stacks-perfect-section-introduction}.
A lot more could be added to these chapters.



\section{Flat and smooth}
\label{section-flat-smooth}

\noindent
Artin's theorem that having a flat surjection from a scheme is a replacement
for the smooth surjective condition. This is now available as
Criteria for Representability, Theorem \ref{criteria-theorem-bootstrap}.


\section{Artin's representability theorem}
\label{section-representability}

\noindent
This is discussed in the chapter
Artin's Axioms, Section \ref{artin-section-introduction}.
We also have an application, see
Quot, Theorem \ref{quot-theorem-coherent-algebraic}.
There should be a lot more applications and the chapter
itself has to be cleaned up as well.


\section{DM stacks are finitely covered by schemes}
\label{section-dm-finite-cover}

\noindent
We already have the corresponding result for algebraic spaces, see
Limits of Spaces, Section \ref{spaces-limits-section-finite-cover}.
What is missing is the result for DM and quasi-DM stacks.


\section{Martin Olsson's paper on properness}
\label{section-proper-parametrization}

\noindent
This proves two notions of proper are the same. The first part of this
is now available in the form of Chow's lemma for algebraic stacks, see
More on Morphisms of Stacks, Theorem
\ref{stacks-more-morphisms-theorem-chow-finite-type}.
As a consequence we show that it suffices to use DVR's
in checking the valuative criterion for properness for
algebraic stacks in certain cases, see
More on Morphisms of Stacks, Section
\ref{stacks-more-morphisms-section-Noetherian-valuative-criterion}.


\section{Proper pushforward of coherent sheaves}
\label{section-proper-pushforward}

\noindent
We can start working on this now that we have Chow's lemma for
algebraic stacks, see previous section.


\section{Keel and Mori}
\label{section-keel-mori}

\noindent
See \cite{K-M}. Their result has been added in
More on Morphisms of Stacks, Section
\ref{stacks-more-morphisms-section-Keel-Mori}.


\section{Add more here}
\label{section-add-more}

\noindent
Actually, no we should never have started this list as part of
the Stacks project itself! There is a todo list somewhere else
which is much easier to update.


\input{chapters}


\bibliography{my}
\bibliographystyle{amsalpha}

\end{document}
