\input{preamble}

% OK, start here.
%
\begin{document}

\title{Hypercoverings}


\maketitle

\phantomsection
\label{section-phantom}

\tableofcontents

\section{Introduction}
\label{section-introduction}

\noindent
Let $\mathcal{C}$ be a site, see Sites, Definition \ref{sites-definition-site}.
Let $X$ be an object of $\mathcal{C}$.
Given an abelian sheaf $\mathcal{F}$
on $\mathcal{C}$ we would like to compute
its cohomology groups
$$
H^i(X, \mathcal{F}).
$$
According to our general definitions (Cohomology on Sites, Section
\ref{sites-cohomology-section-cohomology-sheaves})
this cohomology group is computed by
choosing an injective resolution
$
0 \to \mathcal{F} \to \mathcal{I}^0 \to \mathcal{I}^1 \to \ldots
$
and setting
$$
H^i(X, \mathcal{F})
=
H^i(
\Gamma(X, \mathcal{I}^0) \to
\Gamma(X, \mathcal{I}^1) \to
\Gamma(X, \mathcal{I}^2)\to \ldots)
$$
The goal of this chapter is to show that we may also compute these
cohomology groups without choosing an injective resolution
(in the case that $\mathcal{C}$ has fibre products). To do this
we will use hypercoverings.

\medskip\noindent
A hypercovering in a site is a generalization of a covering, see
\cite[Expos\'e V, Sec. 7]{SGA4}. Given a hypercovering $K$ of an object
$X$, there is a {\v C}ech to cohomology spectral sequence
expressing the cohomology of an abelian sheaf $\mathcal{F}$
over $X$ in terms of the cohomology of the sheaf over the
components $K_n$ of $K$. It turns out that there are always
enough hypercoverings, so that taking the colimit over all hypercoverings,
the spectral sequence degenerates and the cohomology of $\mathcal{F}$
over $X$ is computed by the colimit of the {\v C}ech cohomology groups.

\medskip\noindent
A more general gadget one can consider is a simplicial augmentation where
one has cohomological descent, see \cite[Expos\'e Vbis]{SGA4}. A nice
manuscript on cohomological descent is the text by Brian Conrad, see
\url{http://math.stanford.edu/~conrad/papers/hypercover.pdf}.
We will come back to these issue in the chapter on simplicial spaces
where we will show, for example, that proper hypercoverings of
``locally compact'' topological spaces are of cohomological
descent (Simplicial Spaces, Section
\ref{spaces-simplicial-section-proper-hypercovering}).
Our method of attack will be to reduce this statement to the {\v C}ech to
cohomology spectral sequence constructed in this chapter.






























\section{Semi-representable objects}
\label{section-semi-representable}

\noindent
In order to start we make the following definition.
The letters ``SR'' stand for Semi-Representable.

\begin{definition}
\label{definition-SR}
Let $\mathcal{C}$ be a category. We denote $\text{SR}(\mathcal{C})$
the category of {\it semi-representable objects} defined as follows
\begin{enumerate}
\item objects are families of objects $\{U_i\}_{i \in I}$, and
\item morphisms $\{U_i\}_{i \in I} \to \{V_j\}_{j \in J}$ are given by
a map $\alpha : I \to J$ and for each $i \in I$
a morphism $f_i : U_i \to V_{\alpha(i)}$ of $\mathcal{C}$.
\end{enumerate}
Let $X \in \Ob(\mathcal{C})$ be an object of $\mathcal{C}$.
The category of {\it semi-representable objects over $X$}
is the category
$\text{SR}(\mathcal{C}, X) = \text{SR}(\mathcal{C}/X)$.
\end{definition}

\noindent
This definition is essentially equivalent to
\cite[Expos\'e V, Subsection 7.3.0]{SGA4}. Note that
this is a ``big'' category. We will later ``bound'' the size of the index
sets $I$ that we need for hypercoverings of $X$. We can then redefine
$\text{SR}(\mathcal{C}, X)$ to become a category. Let's spell out
the objects and morphisms $\text{SR}(\mathcal{C}, X)$:
\begin{enumerate}
\item objects are families of morphisms
$\{U_i \to X\}_{i \in I}$, and
\item morphisms $\{U_i \to X\}_{i \in I} \to
\{V_j \to X\}_{j \in J}$ are given by
a map $\alpha : I \to J$ and for each $i \in I$
a morphism $f_i : U_i \to V_{\alpha(i)}$ over $X$.
\end{enumerate}
There is a forgetful functor
$\text{SR}(\mathcal{C}, X) \to \text{SR}(\mathcal{C})$.

\begin{definition}
\label{definition-SR-F}
Let $\mathcal{C}$ be a category.
We denote $F$ the functor {\it which associates a presheaf to a
semi-representable object}. In a formula
\begin{eqnarray*}
F : \text{SR}(\mathcal{C}) & \longrightarrow & \textit{PSh}(\mathcal{C}) \\
\{U_i\}_{i \in I} & \longmapsto & \amalg_{i\in I} h_{U_i}
\end{eqnarray*}
where $h_U$ denotes the representable presheaf associated to
the object $U$.
\end{definition}

\noindent
Given a morphism $U \to X$ we obtain a morphism $h_U \to h_X$ of representable
presheaves. Thus we often think of $F$ on $\text{SR}(\mathcal{C}, X)$
as a functor into the category of presheaves of sets over $h_X$,
namely $\textit{PSh}(\mathcal{C})/h_X$. Here is a picture:
$$
\xymatrix{
\text{SR}(\mathcal{C}, X) \ar[r]_F \ar[d] &
\textit{PSh}(\mathcal{C})/h_X \ar[d] \\
\text{SR}(\mathcal{C}) \ar[r]^F &
\textit{PSh}(\mathcal{C})
}
$$
Next we discuss the existence of limits in the category of semi-representable
objects.

\begin{lemma}
\label{lemma-coprod-prod-SR}
Let $\mathcal{C}$ be a category.
\begin{enumerate}
\item the category $\text{SR}(\mathcal{C})$ has coproducts
and $F$ commutes with them,
\item the functor $F : \text{SR}(\mathcal{C}) \to \textit{PSh}(\mathcal{C})$
commutes with limits,
\item if $\mathcal{C}$ has fibre products, then $\text{SR}(\mathcal{C})$
has fibre products,
\item if $\mathcal{C}$ has products of pairs, then
$\text{SR}(\mathcal{C})$ has products of pairs,
\item if $\mathcal{C}$ has equalizers, so does $\text{SR}(\mathcal{C})$, and
\item if $\mathcal{C}$ has a final object, so does $\text{SR}(\mathcal{C})$.
\end{enumerate}
Let $X \in \Ob(\mathcal{C})$.
\begin{enumerate}
\item the category $\text{SR}(\mathcal{C}, X)$ has coproducts
and $F$ commutes with them,
\item if $\mathcal{C}$ has fibre products, then $\text{SR}(\mathcal{C}, X)$
has finite limits and
$F : \text{SR}(\mathcal{C}, X) \to \textit{PSh}(\mathcal{C})/h_X$
commutes with them.
\end{enumerate}
\end{lemma}

\begin{proof}
Proof of the results on $\text{SR}(\mathcal{C})$.
Proof of (1). The coproduct of $\{U_i\}_{i \in I}$ and $\{V_j\}_{j \in J}$ is
$\{U_i\}_{i \in I} \amalg \{V_j\}_{j \in J}$, in other words, the family
of objects whose index set is $I \amalg J$ and for an element
$k \in I \amalg J$ gives $U_i$ if $k = i \in I$ and gives $V_j$ if
$k = j \in J$. Similarly for coproducts
of families of objects. It is clear that $F$ commutes with these.

\medskip\noindent
Proof of (2). For $U$ in $\Ob(\mathcal{C})$ consider the object $\{U\}$ of
$\text{SR}(\mathcal{C})$. It is clear that
$\Mor_{\text{SR}(\mathcal{C})}(\{U\}, K)) = F(K)(U)$
for $K \in \Ob(\text{SR}(\mathcal{C}))$. Since limits of presheaves
are computed at the level of sections
(Sites, Section \ref{sites-section-limits-colimits-PSh})
we conclude that $F$ commutes with limits.

\medskip\noindent
Proof of (3). Suppose given a morphism
$(\alpha, f_i) : \{U_i\}_{i \in I} \to \{V_j\}_{j \in J}$
and a morphism
$(\beta, g_k) : \{W_k\}_{k \in K} \to \{V_j\}_{j \in J}$.
The fibred product of these morphisms is given by
$$
\{ U_i \times_{f_i, V_j, g_k} W_k\}_{(i, j, k) \in I \times J \times K
\text{ such that } j = \alpha(i) = \beta(k)}
$$
The fibre products exist if $\mathcal{C}$ has fibre products.

\medskip\noindent
Proof of (4). The product of $\{U_i\}_{i \in I}$ and $\{V_j\}_{j \in J}$ is
$\{U_i \times V_j\}_{i \in I, j \in J}$. The products exist if
$\mathcal{C}$ has products.

\medskip\noindent
Proof of (5). The equalizer of two maps
$(\alpha, f_i), (\alpha', f'_i) : \{U_i\}_{i \in I} \to \{V_j\}_{j \in J}$
is
$$
\{
\text{Eq}(f_i, f'_i : U_i \to V_{\alpha(i)})
\}_{i \in I,\ \alpha(i) = \alpha'(i)}
$$
The equalizers exist if $\mathcal{C}$ has equalizers.

\medskip\noindent
Proof of (6). If $X$ is a final object of $\mathcal{C}$, then
$\{X\}$ is a final object of $\text{SR}(\mathcal{C})$.

\medskip\noindent
Proof of the statements about $\text{SR}(\mathcal{C}, X)$.
These follow from the results above applied to the category
$\mathcal{C}/X$ using that
$\text{SR}(\mathcal{C}/X) = \text{SR}(\mathcal{C}, X)$ and that
$\textit{PSh}(\mathcal{C}/X) = \textit{PSh}(\mathcal{C})/h_X$
(Sites, Lemma \ref{sites-lemma-essential-image-j-shriek} applied
to $\mathcal{C}$ endowed with the chaotic topology). However
we also argue directly as follows.
It is clear that the coproduct of
$\{U_i \to X\}_{i \in I}$ and $\{V_j \to X\}_{j \in J}$
is $\{U_i \to X\}_{i \in I} \amalg \{V_j \to X\}_{j \in J}$
and similarly for coproducts of
families of families of morphisms with target $X$.
The object $\{X \to X\}$ is a final
object of $\text{SR}(\mathcal{C}, X)$.
Suppose given a morphism
$(\alpha, f_i) : \{U_i \to X\}_{i \in I} \to \{V_j \to X\}_{j \in J}$
and a morphism
$(\beta, g_k) : \{W_k \to X\}_{k \in K} \to \{V_j \to X\}_{j \in J}$.
The fibred product of these morphisms is given by
$$
\{ U_i \times_{f_i, V_j, g_k} W_k \to X \}_{(i, j, k) \in I \times J \times K
\text{ such that } j = \alpha(i) = \beta(k)}
$$
The fibre products exist by the assumption that
$\mathcal{C}$ has fibre products.
Thus $\text{SR}(\mathcal{C}, X)$ has finite limits,
see Categories, Lemma \ref{categories-lemma-finite-limits-exist}.
We omit verifying the statements on the functor $F$ in this case.
\end{proof}





\section{Hypercoverings}
\label{section-hypercoverings}

\noindent
If we assume our category is a site, then we can make the following
definition.

\begin{definition}
\label{definition-covering-SR}
Let $\mathcal{C}$ be a site. Let
$f = (\alpha, f_i) : \{U_i\}_{i \in I} \to \{V_j\}_{j \in J}$
be a morphism in the category $\text{SR}(\mathcal{C})$.
We say that $f$ is a {\it covering} if for every $j \in J$ the
family of morphisms $\{U_i \to V_j\}_{i \in I, \alpha(i) = j}$
is a covering for the site $\mathcal{C}$.
Let $X$ be an object of $\mathcal{C}$.
A morphism $K \to L$ in $\text{SR}(\mathcal{C}, X)$ is
a {\it covering} if its image in $\text{SR}(\mathcal{C})$ is
a covering.
\end{definition}

\begin{lemma}
\label{lemma-covering-permanence}
Let $\mathcal{C}$ be a site.
\begin{enumerate}
\item A composition of coverings in $\text{SR}(\mathcal{C})$
is a covering.
\item If $K \to L$ is a covering in $\text{SR}(\mathcal{C})$
and $L' \to L$ is a morphism, then $L' \times_L K$ exists
and $L' \times_L K \to L'$ is a covering.
\item If $\mathcal{C}$ has products of pairs, and
$A \to B$ and $K \to L$ are coverings in $\text{SR}(\mathcal{C})$,
then $A \times K \to B \times L$ is a covering.
\end{enumerate}
Let $X \in \Ob(\mathcal{C})$. Then (1) and (2) holds for
$\text{SR}(\mathcal{C}, X)$ and (3) holds if $\mathcal{C}$
has fibre products.
\end{lemma}

\begin{proof}
Part (1) is immediate from the axioms of a site.
Part (2) follows by the construction of fibre products
in $\text{SR}(\mathcal{C})$ in the proof of
Lemma \ref{lemma-coprod-prod-SR}
and the requirement that the morphisms in a covering
of $\mathcal{C}$ are representable.
Part (3) follows by thinking of $A \times K \to B \times L$
as the composition $A \times K \to B \times K \to B \times L$
and hence a composition of basechanges of coverings.
The final statement follows because $\text{SR}(\mathcal{C}, X) =
\text{SR}(\mathcal{C}/X)$.
\end{proof}

\noindent
By Lemma \ref{lemma-coprod-prod-SR} and
Simplicial, Lemma \ref{simplicial-lemma-existence-cosk}
the coskeleton of a truncated simplicial object of
$\text{SR}(\mathcal{C}, X)$ exists if $\mathcal{C}$ has fibre products.
Hence the following definition makes sense.

\begin{definition}
\label{definition-hypercovering}
Let $\mathcal{C}$ be a site. Assume $\mathcal{C}$ has fibre products.
Let $X \in \Ob(\mathcal{C})$ be an object of $\mathcal{C}$.
A {\it hypercovering of $X$} is a simplicial object
$K$ of $\text{SR}(\mathcal{C}, X)$ such that
\begin{enumerate}
\item The object $K_0$ is a covering of $X$ for the site $\mathcal{C}$.
\item For every $n \geq 0$ the canonical morphism
$$
K_{n + 1} \longrightarrow (\text{cosk}_n \text{sk}_n K)_{n + 1}
$$
is a covering in the sense defined above.
\end{enumerate}
\end{definition}

\noindent
Condition (1) makes sense since each object of
$\text{SR}(\mathcal{C}, X)$ is after all a family
of morphisms with target $X$. It could also be
formulated as saying that the morphism of $K_0$ to
the final object of $\text{SR}(\mathcal{C}, X)$
is a covering.

\begin{example}
\label{example-cech}
Let $\{U_i \to X\}_{i \in I}$ be a covering of the site $\mathcal{C}$.
Set $K_0 = \{U_i \to X\}_{i \in I}$.
Then $K_0$ is a $0$-truncated simplicial object of
$\text{SR}(\mathcal{C}, X)$. Hence we may form
$$
K = \text{cosk}_0 K_0.
$$
Clearly $K$ passes condition (1) of Definition \ref{definition-hypercovering}.
Since all the morphisms $K_{n + 1} \to (\text{cosk}_n \text{sk}_n K)_{n + 1}$
are isomorphisms by
Simplicial, Lemma \ref{simplicial-lemma-cosk-up}
it also passes condition (2). Note that
the terms $K_n$ are the usual
$$
K_n = \{
U_{i_0} \times_X U_{i_1} \times_X \ldots \times_X U_{i_n} \to X
\}_{(i_0, i_1, \ldots, i_n) \in I^{n + 1}}
$$
\end{example}

\begin{lemma}
\label{lemma-hypercoverings-set}
Let $\mathcal{C}$ be a site with fibre products.
Let $X \in \Ob(\mathcal{C})$ be an object of $\mathcal{C}$.
The collection of all hypercoverings of $X$ forms a set.
\end{lemma}

\begin{proof}
Since $\mathcal{C}$ is a site, the set of all coverings of
$X$ forms a set. Thus we see that the collection
of possible $K_0$ forms a set. Suppose we have shown that
the collection of all possible $K_0, \ldots, K_n$ form
a set. Then it is enough to show that given
$K_0, \ldots, K_n$ the collection of all possible
$K_{n + 1}$ forms a set. And this is clearly true since
we have to choose $K_{n + 1}$ among all possible coverings
of $(\text{cosk}_n \text{sk}_n K)_{n + 1}$.
\end{proof}

\begin{remark}
\label{remark-hypercoverings-really-set}
The lemma does not just say that there is a cofinal
system of choices of hypercoverings that is a set,
but that really the hypercoverings form a set.
\end{remark}

\noindent
The category of presheaves on $\mathcal{C}$ has
finite (co)limits. Hence the functors $\text{cosk}_n$
exists for presheaves of sets.

\begin{lemma}
\label{lemma-hypercovering-F}
Let $\mathcal{C}$ be a site with fibre products.
Let $X \in \Ob(\mathcal{C})$ be an object of $\mathcal{C}$.
Let $K$ be a hypercovering of $X$.
Consider the simplicial object $F(K)$ of $\textit{PSh}(\mathcal{C})$,
endowed with its augmentation to the constant simplicial presheaf $h_X$.
\begin{enumerate}
\item The morphism of presheaves $F(K)_0 \to h_X$ becomes
a surjection after sheafification.
\item The morphism
$$
(d^1_0, d^1_1) :
F(K)_1
\longrightarrow
F(K)_0 \times_{h_X} F(K)_0
$$
becomes a surjection after sheafification.
\item For every $n \geq 1$ the morphism
$$
F(K)_{n + 1} \longrightarrow (\text{cosk}_n \text{sk}_n F(K))_{n + 1}
$$
turns into a surjection after sheafification.
\end{enumerate}
\end{lemma}

\begin{proof}
We will use the fact that if
$\{U_i \to U\}_{i \in I}$ is a covering of the site
$\mathcal{C}$, then the morphism
$$
\amalg_{i \in I} h_{U_i} \to h_U
$$
becomes surjective after sheafification, see
Sites, Lemma \ref{sites-lemma-covering-surjective-after-sheafification}.
Thus the first assertion follows immediately.

\medskip\noindent
For the second assertion, note that according to
Simplicial, Example \ref{simplicial-example-cosk0}
the simplicial object $\text{cosk}_0 \text{sk}_0 K$
has terms $K_0 \times \ldots \times K_0$. Thus
according to the definition of a hypercovering we
see that $(d^1_0, d^1_1) : K_1 \to K_0 \times K_0$ is a
covering. Hence (2) follows from the claim above
and the fact that $F$ transforms products into fibred
products over $h_X$.

\medskip\noindent
For the third, we claim that
$\text{cosk}_n \text{sk}_n F(K) =
F(\text{cosk}_n \text{sk}_n K)$ for $n \geq 1$.
To prove this, denote temporarily $F'$ the functor
$\text{SR}(\mathcal{C}, X) \to \textit{PSh}(\mathcal{C})/h_X$.
By Lemma \ref{lemma-coprod-prod-SR} the functor
$F'$ commutes with finite limits.
By our description of the $\text{cosk}_n$ functor in
Simplicial, Section \ref{simplicial-section-skeleton}
we see that $\text{cosk}_n \text{sk}_n F'(K) =
F'(\text{cosk}_n \text{sk}_n K)$.
Recall that the category used in the description of
$(\text{cosk}_n U)_m$ in
Simplicial, Lemma \ref{simplicial-lemma-existence-cosk}
is the category $(\Delta/[m])^{opp}_{\leq n}$. It is an
amusing exercise to show that $(\Delta/[m])_{\leq n}$ is
a connected category (see
Categories, Definition \ref{categories-definition-category-connected})
as soon as $n \geq 1$. Hence,
Categories, Lemma \ref{categories-lemma-connected-limit-over-X}
shows that $\text{cosk}_n \text{sk}_n F'(K) =
\text{cosk}_n \text{sk}_n F(K)$. Whence the claim.
Property (2) follows from this, because now we see that
the morphism in (2) is the result of applying the
functor $F$ to a covering as in Definition \ref{definition-covering-SR},
and the result follows from the first fact mentioned
in this proof.
\end{proof}



\section{Acyclicity}
\label{section-acyclicity}

\noindent
Let $\mathcal{C}$ be a site.
For a presheaf of sets $\mathcal{F}$ we denote $\mathbf{Z}_\mathcal{F}$
the presheaf of abelian groups defined by the rule
$$
\mathbf{Z}_\mathcal{F}(U) = \text{free abelian group on }\mathcal{F}(U).
$$
We will sometimes call this the {\it free abelian presheaf on $\mathcal{F}$}.
Of course the construction $\mathcal{F} \mapsto \mathbf{Z}_\mathcal{F}$
is a functor and it is left adjoint to the forgetful functor
$\textit{PAb}(\mathcal{C}) \to \textit{PSh}(\mathcal{C})$.
Of course the sheafification $\mathbf{Z}_\mathcal{F}^\#$ is
a sheaf of abelian groups, and the functor
$\mathcal{F} \mapsto \mathbf{Z}_\mathcal{F}^\#$ is a
left adjoint as well. We sometimes call $\mathbf{Z}_\mathcal{F}^\#$
the {\it free abelian sheaf on $\mathcal{F}$}.

\medskip\noindent
For an object $X$ of the site $\mathcal{C}$ we denote
$\mathbf{Z}_X$ the free abelian presheaf on $h_X$, and
we denote $\mathbf{Z}_X^\#$ its sheafification.

\begin{definition}
\label{definition-homology}
Let $\mathcal{C}$ be a site.
Let $K$ be a simplicial object of $\textit{PSh}(\mathcal{C})$.
By the above we get a simplicial object $\mathbf{Z}_K^\#$ of
$\textit{Ab}(\mathcal{C})$. We can take its associated
complex of abelian presheaves $s(\mathbf{Z}_K^\#)$, see
Simplicial, Section \ref{simplicial-section-complexes}.
The {\it homology of $K$} is the homology of the
complex of abelian sheaves $s(\mathbf{Z}_K^\#)$.
\end{definition}

\noindent
In other words, the {\it $i$th homology $H_i(K)$ of $K$}
is the sheaf of abelian groups $H_i(K) = H_i(s(\mathbf{Z}_K^\#))$.
In this section we worry about the homology in case $K$
is a hypercovering of an object $X$ of $\mathcal{C}$.

\begin{lemma}
\label{lemma-compare-cosk0}
Let $\mathcal{C}$ be a site.
Let $\mathcal{F} \to \mathcal{G}$ be a morphism
of presheaves of sets. Denote $K$ the simplicial
object of $\textit{PSh}(\mathcal{C})$ whose $n$th
term is the $(n + 1)$st fibre product of $\mathcal{F}$
over $\mathcal{G}$, see
Simplicial, Example \ref{simplicial-example-fibre-products-simplicial-object}.
Then, if $\mathcal{F} \to \mathcal{G}$ is surjective after
sheafification, we have
$$
H_i(K) =
\left\{
\begin{matrix}
0 & \text{if} & i > 0\\
\mathbf{Z}_\mathcal{G}^\# & \text{if} & i = 0
\end{matrix}
\right.
$$
The isomorphism in degree $0$ is given by the
morphism $H_0(K) \to \mathbf{Z}_\mathcal{G}^\#$
coming from the map $(\mathbf{Z}_K^\#)_0 =
\mathbf{Z}_\mathcal{F}^\# \to \mathbf{Z}_\mathcal{G}^\#$.
\end{lemma}

\begin{proof}
Let $\mathcal{G}' \subset \mathcal{G}$ be the image of
the morphism $\mathcal{F} \to \mathcal{G}$.
Let $U \in \Ob(\mathcal{C})$. Set
$A = \mathcal{F}(U)$ and $B = \mathcal{G}'(U)$.
Then the simplicial set $K(U)$ is equal to the simplicial
set with $n$-simplices given by
$$
A \times_B A \times_B \ldots \times_B A\ (n + 1 \text{ factors)}.
$$
By Simplicial, Lemma \ref{simplicial-lemma-cosk-minus-one-equivalence}
the morphism $K(U) \to B$ is a trivial Kan fibration.
Thus it is a homotopy equivalence
(Simplicial, Lemma \ref{simplicial-lemma-cosk-minus-one-equivalence}).
Hence applying the functor ``free abelian group on'' to this
we deduce that
$$
\mathbf{Z}_K(U) \longrightarrow \mathbf{Z}_B
$$
is a homotopy equivalence. Note that $s(\mathbf{Z}_B)$ is
the complex
$$
\ldots \to
\bigoplus\nolimits_{b \in B}\mathbf{Z} \xrightarrow{0}
\bigoplus\nolimits_{b \in B}\mathbf{Z} \xrightarrow{1}
\bigoplus\nolimits_{b \in B}\mathbf{Z} \xrightarrow{0}
\bigoplus\nolimits_{b \in B}\mathbf{Z} \to 0
$$
see Simplicial, Lemma \ref{simplicial-lemma-homology-eilenberg-maclane}.
Thus we see that
$H_i(s(\mathbf{Z}_K(U))) = 0$ for $i > 0$, and
$H_0(s(\mathbf{Z}_K(U))) = \bigoplus_{b \in B}\mathbf{Z}
= \bigoplus_{s \in \mathcal{G}'(U)} \mathbf{Z}$.
These identifications are compatible with restriction
maps.

\medskip\noindent
We conclude that $H_i(s(\mathbf{Z}_K)) = 0$ for $i > 0$ and
$H_0(s(\mathbf{Z}_K)) = \mathbf{Z}_{\mathcal{G}'}$, where here
we compute homology groups in $\textit{PAb}(\mathcal{C})$. Since
sheafification is an exact functor we deduce the result
of the lemma. Namely, the exactness implies
that $H_0(s(\mathbf{Z}_K))^\# = H_0(s(\mathbf{Z}_K^\#))$,
and similarly for other indices.
\end{proof}

\begin{lemma}
\label{lemma-acyclicity}
Let $\mathcal{C}$ be a site.
Let $f : L \to K$ be a morphism of
simplicial objects of $\textit{PSh}(\mathcal{C})$.
Let $n \geq 0$ be an integer.
Assume that
\begin{enumerate}
\item For $i < n$ the morphism $L_i \to K_i$ is an isomorphism.
\item The morphism $L_n \to K_n$ is surjective after sheafification.
\item The canonical map $L \to \text{cosk}_n \text{sk}_n L$ is an isomorphism.
\item The canonical map $K \to \text{cosk}_n \text{sk}_n K$ is an isomorphism.
\end{enumerate}
Then $H_i(f) : H_i(L) \to H_i(K)$ is an isomorphism.
\end{lemma}

\begin{proof}
This proof is exactly the same as the proof of
Lemma \ref{lemma-compare-cosk0} above. Namely,
we first let $K_n' \subset K_n$ be the sub presheaf
which is the image of the map $L_n \to K_n$. Assumption
(2) means that the sheafification of $K_n'$ is equal to
the sheafification of $K_n$. Moreover, since $L_i = K_i$
for all $i < n$ we see that get an $n$-truncated
simplicial presheaf $U$ by taking
$U_0 = L_0 = K_0, \ldots, U_{n - 1} = L_{n - 1} = K_{n - 1}, U_n = K'_n$.
Denote $K' = \text{cosk}_n U$, a simplicial presheaf.
Because we can construct $K'_m$ as a finite limit, and
since sheafification is exact, we see that
$(K'_m)^\# = K_m$. In other words, $(K')^\# = K^\#$.
We conclude, by exactness of sheafification once more,
that $H_i(K) = H_i(K')$. Thus it suffices to prove the lemma
for the morphism $L \to K'$, in other words, we may
assume that $L_n \to K_n$ is a surjective morphism
of {\it presheaves}!

\medskip\noindent
In this case, for any object $U$ of $\mathcal{C}$ we
see that the morphism of simplicial sets
$$
L(U) \longrightarrow K(U)
$$
satisfies all the assumptions of
Simplicial, Lemma \ref{simplicial-lemma-section}.
Hence it is a trivial Kan fibration. In particular it is
a homotopy equivalence
(Simplicial, Lemma \ref{simplicial-lemma-trivial-kan-homotopy}).
Thus
$$
\mathbf{Z}_L(U) \longrightarrow \mathbf{Z}_K(U)
$$
is a homotopy equivalence too. This for all $U$.
The result follows.
\end{proof}

\begin{lemma}
\label{lemma-acyclic-hypercover-sheaves}
Let $\mathcal{C}$ be a site.
Let $K$ be a simplicial presheaf.
Let $\mathcal{G}$ be a presheaf.
Let $K \to \mathcal{G}$ be an augmentation of $K$
towards $\mathcal{G}$. Assume that
\begin{enumerate}
\item The morphism of presheaves $K_0 \to \mathcal{G}$ becomes
a surjection after sheafification.
\item The morphism
$$
(d^1_0, d^1_1) :
K_1
\longrightarrow
K_0 \times_\mathcal{G} K_0
$$
becomes a surjection after sheafification.
\item For every $n \geq 1$ the morphism
$$
K_{n + 1} \longrightarrow (\text{cosk}_n \text{sk}_n K)_{n + 1}
$$
turns into a surjection after sheafification.
\end{enumerate}
Then $H_i(K) = 0$ for $i > 0$ and
$H_0(K) = \mathbf{Z}_\mathcal{G}^\#$.
\end{lemma}

\begin{proof}
Denote $K^n = \text{cosk}_n \text{sk}_n K$ for $n \geq 1$.
Define $K^0$ as the simplicial object with terms
$(K^0)_n$ equal to the $(n + 1)$-fold fibred product
$K_0 \times_\mathcal{G} \ldots \times_\mathcal{G} K_0$,
see Simplicial,
Example \ref{simplicial-example-fibre-products-simplicial-object}.
We have morphisms
$$
K \longrightarrow \ldots \to K^n \to K^{n - 1} \to \ldots \to K^1 \to K^0.
$$
The morphisms $K \to K^i$, $K^j \to K^i$ for $j \geq i \geq 1$ come
from the universal properties of the $\text{cosk}_n$ functors.
The morphism $K^1 \to K^0$ is the canonical morphism
from
Simplicial, Remark \ref{simplicial-remark-augmentation}.
We also recall that $K^0 \to \text{cosk}_1 \text{sk}_1 K^0$
is an isomorphism, see
Simplicial, Lemma \ref{simplicial-lemma-cosk-minus-one}.

\medskip\noindent
By Lemma \ref{lemma-compare-cosk0} we see that
$H_i(K^0) = 0$ for $i > 0$ and $H_0(K^0) = \mathbf{Z}_\mathcal{G}^\#$.

\medskip\noindent
Pick $n \geq 1$. Consider the morphism $K^n \to K^{n - 1}$.
It is an isomorphism on terms of degree $< n$.
Note that $K^n \to \text{cosk}_n \text{sk}_n K^n$ and
$K^{n - 1} \to \text{cosk}_n \text{sk}_n K^{n - 1}$
are isomorphisms. Note that $(K^n)_n = K_n$ and
that $(K^{n - 1})_n = (\text{cosk}_{n - 1} \text{sk}_{n - 1} K)_n$.
Hence by assumption, we have that $(K^n)_n \to (K^{n - 1})_n$
is a morphism of presheaves which becomes surjective after
sheafification. By Lemma \ref{lemma-acyclicity} we conclude that
$H_i(K^n) = H_i(K^{n - 1})$.
Combined with the above this proves the lemma.
\end{proof}

\begin{lemma}
\label{lemma-hypercovering-acyclic}
Let $\mathcal{C}$ be a site with fibre products.
Let $X$ be an object of of $\mathcal{C}$.
Let $K$ be a hypercovering of $X$.
The homology of the simplicial presheaf $F(K)$ is
$0$ in degrees $> 0$ and equal to $\mathbf{Z}_X^\#$
in degree $0$.
\end{lemma}

\begin{proof}
Combine Lemmas \ref{lemma-acyclic-hypercover-sheaves}
and \ref{lemma-hypercovering-F}.
\end{proof}












\section{{\v C}ech cohomology and hypercoverings}
\label{section-hyper-cech}

\noindent
Let $\mathcal{C}$ be a site. Consider a presheaf of
abelian groups $\mathcal{F}$ on the site $\mathcal{C}$.
It defines a functor
\begin{eqnarray*}
\mathcal{F} : \text{SR}(\mathcal{C})^{opp}
& \longrightarrow &
\textit{Ab} \\
\{U_i\}_{i \in I} &
\longmapsto &
\prod\nolimits_{i \in I} \mathcal{F}(U_i)
\end{eqnarray*}
Thus a simplicial object $K$ of $\text{SR}(\mathcal{C})$
is turned into a cosimplicial object $\mathcal{F}(K)$ of $\textit{Ab}$.
The cochain complex $s(\mathcal{F})(K))$ associated to $\mathcal{F}(K)$
(Simplicial, Section
\ref{simplicial-section-dold-kan-cosimplicial})
is called the {\v C}ech complex of $\mathcal{F}$ with
respect to the simplicial object $K$. We set
$$
\check{H}^i(K, \mathcal{F})
=
H^i(s(\mathcal{F}(K))).
$$
and we call it the $i$th {\v C}ech cohomology group
of $\mathcal{F}$ with respect to $K$.
In this section we prove analogues of some of the results for
{\v C}ech cohomology of open coverings proved in
Cohomology, Sections \ref{cohomology-section-cech},
\ref{cohomology-section-cech-functor} and
\ref{cohomology-section-cech-cohomology-cohomology}.

\begin{lemma}
\label{lemma-h0-cech}
Let $\mathcal{C}$ be a site with fibre products.
Let $X$ be an object of $\mathcal{C}$.
Let $K$ be a hypercovering of $X$.
Let $\mathcal{F}$ be a sheaf of abelian groups on $\mathcal{C}$.
Then $\check{H}^0(K, \mathcal{F}) = \mathcal{F}(X)$.
\end{lemma}

\begin{proof}
We have
$$
\check{H}^0(K, \mathcal{F})
=
\Ker(\mathcal{F}(K_0) \longrightarrow \mathcal{F}(K_1))
$$
Write $K_0 = \{U_i \to X\}$. It is a covering in the site
$\mathcal{C}$. As well, we have that $K_1 \to K_0 \times K_0$
is a covering in $\text{SR}(\mathcal{C}, X)$. Hence we may
write $K_1 = \amalg_{i_0, i_1 \in I} \{V_{i_0i_1j} \to X\}$
so that the morphism $K_1 \to K_0 \times K_0$ is given
by coverings $\{V_{i_0i_1j} \to U_{i_0} \times_X U_{i_1}\}$
of the site $\mathcal{C}$. Thus we can further identify
$$
\check{H}^0(K, \mathcal{F})
=
\Ker(
\prod\nolimits_i \mathcal{F}(U_i)
\longrightarrow
\prod\nolimits_{i_0i_1 j} \mathcal{F}(V_{i_0i_1j})
)
$$
with obvious map. The sheaf property of $\mathcal{F}$
implies that $\check{H}^0(K, \mathcal{F}) = H^0(X, \mathcal{F})$.
\end{proof}

\noindent
In fact this property characterizes the abelian sheaves among all
abelian presheaves on $\mathcal{C}$ of course.
The analogue of Cohomology, Lemma \ref{lemma-injective-trivial-cech}
in this case is the following.

\begin{lemma}
\label{lemma-injective-trivial-cech}
Let $\mathcal{C}$ be a site with fibre products.
Let $X$ be an object of $\mathcal{C}$.
Let $K$ be a hypercovering of $X$.
Let $\mathcal{I}$ be an injective sheaf of abelian groups on $\mathcal{C}$.
Then
$$
\check{H}^p(K, \mathcal{I}) =
\left\{
\begin{matrix}
\mathcal{I}(X) & \text{if} & p = 0 \\
0 & \text{if} & p > 0
\end{matrix}
\right.
$$
\end{lemma}

\begin{proof}
Observe that for any object $Z = \{U_i \to X\}$ of
$\text{SR}(\mathcal{C}, X)$ and any abelian sheaf
$\mathcal{F}$ on $\mathcal{C}$ we have
\begin{eqnarray*}
\mathcal{F}(Z)
& = &
\prod \mathcal{F}(U_i) \\
& = &
\prod \Mor_{\textit{PSh}(\mathcal{C})}(h_{U_i}, \mathcal{F})\\
& = &
\Mor_{\textit{PSh}(\mathcal{C})}(F(Z), \mathcal{F})\\
& = &
\Mor_{\textit{PAb}(\mathcal{C})}(\mathbf{Z}_{F(Z)}, \mathcal{F}) \\
& = &
\Mor_{\textit{Ab}(\mathcal{C})}(\mathbf{Z}_{F(Z)}^\#, \mathcal{F})
\end{eqnarray*}
Thus we see, for any simplicial object $K$ of
$\text{SR}(\mathcal{C}, X)$ that we have
\begin{equation}
\label{equation-identify-cech}
s(\mathcal{F}(K))
=
\Hom_{\textit{Ab}(\mathcal{C})}(s(\mathbf{Z}_{F(K)}^\#), \mathcal{F})
\end{equation}
see Definition \ref{definition-homology} for notation.
The complex of sheaves $s(\mathbf{Z}_{F(K)}^\#)$ is quasi-isomorphic
to $\mathbf{Z}_X^\#$ if $K$ is a hypercovering, see
Lemma \ref{lemma-hypercovering-acyclic}. We conclude
that if $\mathcal{I}$ is an injective abelian sheaf, and
$K$ a hypercovering, then the complex $s(\mathcal{I}(K))$
is acyclic except possibly in degree $0$.
In other words, we have
$$
\check{H}^i(K, \mathcal{I}) = 0
$$
for $i > 0$. Combined with Lemma \ref{lemma-h0-cech} the lemma is proved.
\end{proof}

\noindent
Next we come to the analogue of Cohomology on Sites, Lemma
\ref{sites-cohomology-lemma-cech-spectral-sequence}.
Let $\mathcal{C}$ be a site.
Let $\mathcal{F}$ be a sheaf of abelian groups on $\mathcal{C}$.
Recall that $\underline{H}^i(\mathcal{F})$ indicates the presheaf
of abelian groups on $\mathcal{C}$ which is defined by the
rule $\underline{H}^i(\mathcal{F}) : U \longmapsto H^i(U, \mathcal{F})$.
We extend this to $\text{SR}(\mathcal{C})$ as in the introduction
to this section.

\begin{lemma}
\label{lemma-cech-spectral-sequence}
Let $\mathcal{C}$ be a site with fibre products.
Let $X$ be an object of $\mathcal{C}$.
Let $K$ be a hypercovering of $X$.
Let $\mathcal{F}$ be a sheaf of abelian groups on $\mathcal{C}$.
There is a map
$$
s(\mathcal{F}(K))
\longrightarrow
R\Gamma(X, \mathcal{F})
$$
in $D^{+}(\textit{Ab})$ functorial in $\mathcal{F}$, which induces
natural transformations
$$
\check{H}^i(K, -) \longrightarrow H^i(X, -)
$$
as functors $\textit{Ab}(\mathcal{C}) \to \textit{Ab}$. Moreover,
there is a spectral sequence $(E_r, d_r)_{r \geq 0}$ with
$$
E_2^{p, q} = \check{H}^p(K, \underline{H}^q(\mathcal{F}))
$$
converging to $H^{p + q}(X, \mathcal{F})$.
This spectral sequence is functorial in $\mathcal{F}$ and
in the hypercovering $K$.
\end{lemma}

\begin{proof}
We could prove this by the same method as employed in the corresponding
lemma in the chapter on cohomology. Instead let us prove this by a
double complex argument.

\medskip\noindent
Choose an injective resolution $\mathcal{F} \to \mathcal{I}^\bullet$
in the category of abelian sheaves on $\mathcal{C}$. Consider the
double complex $A^{\bullet, \bullet}$ with terms
$$
A^{p, q} = \mathcal{I}^q(K_p)
$$
where the differential $d_1^{p, q} : A^{p, q} \to A^{p + 1, q}$
is the one coming from the differential $\mathcal{I}^p \to \mathcal{I}^{p + 1}$
and the differential $d_2^{p, q} : A^{p, q} \to A^{p, q + 1}$ is the
one coming from the differential on the complex
$s(\mathcal{I}^p(K))$ associated to the cosimplicial abelian group
$\mathcal{I}^p(K)$ as explained above.
As usual we denote $sA^\bullet$ the simple complex associated to
the double complex $A^{\bullet, \bullet}$.
We will use the two spectral
sequences $({}'E_r, {}'d_r)$ and $({}''E_r, {}''d_r)$
associated to this double complex, see
Homology, Section \ref{homology-section-double-complex}.

\medskip\noindent
By Lemma \ref{lemma-injective-trivial-cech}
the complexes $s(\mathcal{I}^p(K))$ are acyclic in
positive degrees and have $H^0$ equal to $\mathcal{I}^p(X)$.
Hence by
Homology, Lemma \ref{homology-lemma-double-complex-gives-resolution}
and its proof the spectral sequence $({}'E_r, {}'d_r)$ degenerates,
and the natural map
$$
\mathcal{I}^\bullet(X) \longrightarrow sA^\bullet
$$
is a quasi-isomorphism of complexes of abelian groups. In particular
we conclude that $H^n(sA^\bullet) = H^n(X, \mathcal{F})$.

\medskip\noindent
The map $s(\mathcal{F}(K)) \longrightarrow R\Gamma(X, \mathcal{F})$ of
the lemma is the composition of the natural map
$s(\mathcal{F}(K)) \to sA^\bullet$ followed by the inverse
of the displayed quasi-isomorphism above. This works because
$\mathcal{I}^\bullet(X)$ is a representative of $R\Gamma(X, \mathcal{F})$.

\medskip\noindent
Consider the spectral sequence $({}''E_r, {}''d_r)_{r \geq 0}$. By
Homology, Lemma \ref{homology-lemma-ss-double-complex}
we see that
$$
{}''E_2^{p, q} = H^p_{II}(H^q_I(A^{\bullet, \bullet}))
$$
In other words, we first take cohomology with respect to
$d_1$ which gives the groups
${}''E_1^{p, q} = \underline{H}^p(\mathcal{F})(K_q)$.
Hence it is indeed the case (by the description of the differential
${}''d_1$) that
${}''E_2^{p, q} = \check{H}^p(K, \underline{H}^q(\mathcal{F}))$.
And by the other spectral sequence above we see that this one
converges to $H^n(X, \mathcal{F})$ as desired.

\medskip\noindent
We omit the proof of the statements regarding the functoriality of
the above constructions in the abelian sheaf $\mathcal{F}$ and the
hypercovering $K$.
\end{proof}









\section{Hypercoverings a la Verdier}
\label{section-hypercoverings-verdier}

\noindent
The astute reader will have noticed that all we need in order
to get the {\v C}ech to cohomology spectral sequence for a
hypercovering of an object $X$, is the
conclusion of Lemma \ref{lemma-hypercovering-F}.
Therefore the following definition makes sense.

\begin{definition}
\label{definition-hypercovering-variant}
Let $\mathcal{C}$ be a site. Assume $\mathcal{C}$ has equalizers
and fibre products. Let $\mathcal{G}$ be a presheaf of sets.
A {\it hypercovering of $\mathcal{G}$} is a simplicial object
$K$ of $\text{SR}(\mathcal{C})$ endowed with an augmentation
$F(K) \to \mathcal{G}$ such that
\begin{enumerate}
\item $F(K_0) \to \mathcal{G}$ becomes surjective
after sheafification,
\item $F(K_1) \to F(K_0) \times_\mathcal{G} F(K_0)$
becomes surjective after sheafification, and
\item $F(K_{n + 1}) \longrightarrow F((\text{cosk}_n \text{sk}_n K)_{n + 1})$
for $n \geq 1$ becomes surjective after sheafification.
\end{enumerate}
We say that a simplicial object $K$ of $\text{SR}(\mathcal{C})$
is a {\it hypercovering} if $K$ is a hypercovering of the final
object $*$ of $\textit{PSh}(\mathcal{C})$.
\end{definition}

\noindent
The assumption that $\mathcal{C}$ has fibre products and equalizers
guarantees that $\text{SR}(\mathcal{C})$ has fibre products
and equalizers and $F$ commutes with these
(Lemma \ref{lemma-coprod-prod-SR}) which suffices
to define the coskeleton functors used (see
Simplicial, Remark \ref{simplicial-remark-existence-cosk} and
Categories, Lemma \ref{categories-lemma-fibre-products-equalizers-exist}).
If $\mathcal{C}$ is general, we can replace the condition (3) by the
condition that
$F(K_{n + 1}) \longrightarrow ((\text{cosk}_n \text{sk}_n F(K))_{n + 1})$
for $n \geq 1$ becomes surjective after sheafification and the
results of this section remain valid.

\medskip\noindent
Let $\mathcal{F}$ be an abelian sheaf on $\mathcal{C}$.
In the previous section, we defined the {\v C}ech complex of $\mathcal{F}$
with respect to a simplicial object $K$ of $\text{SR}(\mathcal{C})$.
Next, given a presheaf $\mathcal{G}$ we set
$$
H^0(\mathcal{G}, \mathcal{F}) =
\Mor_{\textit{PSh}(\mathcal{C})}(\mathcal{G}, \mathcal{F}) =
\Mor_{\Sh(\mathcal{C})}(\mathcal{G}^\#, \mathcal{F}) =
H^0(\mathcal{G}^\#, \mathcal{F})
$$
with notation as in
Cohomology on Sites, Section \ref{sites-cohomology-section-limp}.
This is a left exact functor and its higher derived functors
(briefly studied in
Cohomology on Sites, Section \ref{sites-cohomology-section-limp})
will be denoted $H^i(\mathcal{G}, \mathcal{F})$.
We will show that given a hypercovering $K$ of $\mathcal{G}$,
there is a {\v C}ech to cohomology spectral sequence converging to the
cohomology $H^i(\mathcal{G}, \mathcal{F})$.
Note that if $\mathcal{G} = *$, then
$H^i(*, \mathcal{F}) = H^i(\mathcal{C}, \mathcal{F})$ recovers
the cohomology of $\mathcal{F}$ on the site $\mathcal{C}$.

\begin{lemma}
\label{lemma-h0-cech-variant}
Let $\mathcal{C}$ be a site with equalizers and fibre products.
Let $\mathcal{G}$ be a presheaf on $\mathcal{C}$.
Let $K$ be a hypercovering of $\mathcal{G}$.
Let $\mathcal{F}$ be a sheaf of abelian groups on $\mathcal{C}$.
Then $\check{H}^0(K, \mathcal{F}) = H^0(\mathcal{G}, \mathcal{F})$.
\end{lemma}

\begin{proof}
This follows from the definition of $H^0(\mathcal{G}, \mathcal{F})$
and the fact that
$$
\xymatrix{
F(K_1) \ar@<1ex>[r] \ar@<-1ex>[r] &
F(K_0) \ar[r] & \mathcal{G}
}
$$
becomes an coequalizer diagram after sheafification.
\end{proof}

\begin{lemma}
\label{lemma-injective-trivial-cech-variant}
Let $\mathcal{C}$ be a site with equalizers and fibre products.
Let $\mathcal{G}$ be a presheaf on $\mathcal{C}$.
Let $K$ be a hypercovering of $\mathcal{G}$.
Let $\mathcal{I}$ be an injective sheaf of abelian groups on $\mathcal{C}$.
Then
$$
\check{H}^p(K, \mathcal{I}) =
\left\{
\begin{matrix}
H^0(\mathcal{G}, \mathcal{I}) & \text{if} & p = 0 \\
0 & \text{if} & p > 0
\end{matrix}
\right.
$$
\end{lemma}

\begin{proof}
By (\ref{equation-identify-cech}) we have
$$
s(\mathcal{F}(K))
=
\Hom_{\textit{Ab}(\mathcal{C})}(s(\mathbf{Z}_{F(K)}^\#), \mathcal{F})
$$
The complex $s(\mathbf{Z}_{F(K)}^\#)$ is quasi-isomorphic
to $\mathbf{Z}_\mathcal{G}^\#$, see
Lemma \ref{lemma-acyclic-hypercover-sheaves}. We conclude
that if $\mathcal{I}$ is an injective abelian sheaf, then
the complex $s(\mathcal{I}(K))$ is acyclic except possibly in degree $0$.
In other words, we have $\check{H}^i(K, \mathcal{I}) = 0$
for $i > 0$. Combined with Lemma \ref{lemma-h0-cech-variant}
the lemma is proved.
\end{proof}

\begin{lemma}
\label{lemma-cech-spectral-sequence-variant}
Let $\mathcal{C}$ be a site with equalizers and fibre products.
Let $\mathcal{G}$ be a presheaf on $\mathcal{C}$.
Let $K$ be a hypercovering of $\mathcal{G}$.
Let $\mathcal{F}$ be a sheaf of abelian groups on $\mathcal{C}$.
There is a map
$$
s(\mathcal{F}(K)) \longrightarrow R\Gamma(\mathcal{G}, \mathcal{F})
$$
in $D^{+}(\textit{Ab})$ functorial in $\mathcal{F}$, which induces
a natural transformation
$$
\check{H}^i(K, -) \longrightarrow H^i(\mathcal{G}, -)
$$
of functors $\textit{Ab}(\mathcal{C}) \to \textit{Ab}$. Moreover,
there is a spectral sequence $(E_r, d_r)_{r \geq 0}$ with
$$
E_2^{p, q} = \check{H}^p(K, \underline{H}^q(\mathcal{F}))
$$
converging to $H^{p + q}(\mathcal{G}, \mathcal{F})$.
This spectral sequence is functorial in $\mathcal{F}$ and
in the hypercovering $K$.
\end{lemma}

\begin{proof}
Choose an injective resolution $\mathcal{F} \to \mathcal{I}^\bullet$
in the category of abelian sheaves on $\mathcal{C}$. Consider the
double complex $A^{\bullet, \bullet}$ with terms
$$
A^{p, q} = \mathcal{I}^q(K_p)
$$
where the differential $d_1^{p, q} : A^{p, q} \to A^{p + 1, q}$
is the one coming from the differential $\mathcal{I}^p \to \mathcal{I}^{p + 1}$
and the differential $d_2^{p, q} : A^{p, q} \to A^{p, q + 1}$ is the
one coming from the differential on the complex
$s(\mathcal{I}^p(K))$ associated to the cosimplicial abelian group
$\mathcal{I}^p(K)$ as explained above.
We will use the two spectral
sequences $({}'E_r, {}'d_r)$ and $({}''E_r, {}''d_r)$
associated to this double complex, see
Homology, Section \ref{homology-section-double-complex}.

\medskip\noindent
By Lemma \ref{lemma-injective-trivial-cech-variant} the complexes
$s(\mathcal{I}^p(K))$ are acyclic in positive degrees and have
$H^0$ equal to $H^0(\mathcal{G}, \mathcal{I}^p)$. Hence by
Homology, Lemma \ref{homology-lemma-double-complex-gives-resolution}
and its proof the spectral sequence $({}'E_r, {}'d_r)$ degenerates,
and the natural map
$$
H^0(\mathcal{G}, \mathcal{I}^\bullet) \longrightarrow
\text{Tot}(A^{\bullet, \bullet})
$$
is a quasi-isomorphism of complexes of abelian groups. The map
$s(\mathcal{F}(K)) \longrightarrow R\Gamma(\mathcal{G}, \mathcal{F})$
of the lemma is the composition of the natural map
$s(\mathcal{F}(K)) \to \text{Tot}(A^{\bullet, \bullet})$
followed by the inverse of the displayed quasi-isomorphism above.
This works because $H^0(\mathcal{G}, \mathcal{I}^\bullet)$
is a representative of $R\Gamma(\mathcal{G}, \mathcal{F})$.

\medskip\noindent
Consider the spectral sequence $({}''E_r, {}''d_r)_{r \geq 0}$. By
Homology, Lemma \ref{homology-lemma-ss-double-complex}
we see that
$$
{}''E_2^{p, q} = H^p_{II}(H^q_I(A^{\bullet, \bullet}))
$$
In other words, we first take cohomology with respect to
$d_1$ which gives the groups
${}''E_1^{p, q} = \underline{H}^p(\mathcal{F})(K_q)$.
Hence it is indeed the case (by the description of the differential
${}''d_1$) that
${}''E_2^{p, q} = \check{H}^p(K, \underline{H}^q(\mathcal{F}))$.
Since this spectral sequence converges to the cohomology of
$\text{Tot}(A^{\bullet, \bullet})$ the proof is finished.
\end{proof}

\begin{lemma}
\label{lemma-cech-spectral-sequence-verdier}
Let $\mathcal{C}$ be a site with equalizers and fibre products.
Let $K$ be a hypercovering.
Let $\mathcal{F}$ be an abelian sheaf. There is a
spectral sequence $(E_r, d_r)_{r \geq 0}$ with
$$
E_2^{p, q} = \check{H}^p(K, \underline{H}^q(\mathcal{F}))
$$
converging to the global cohomology groups $H^{p + q}(\mathcal{F})$.
\end{lemma}

\begin{proof}
This is a special case of Lemma \ref{lemma-cech-spectral-sequence-variant}.
\end{proof}







\section{Covering hypercoverings}
\label{section-covering}

\noindent
Here are some ways to construct hypercoverings.
We note that since the category
$\text{SR}(\mathcal{C}, X)$ has fibre products
the category of simplicial objects
of $\text{SR}(\mathcal{C}, X)$ has fibre products
as well, see Simplicial, Lemma \ref{simplicial-lemma-fibre-product}.

\begin{lemma}
\label{lemma-funny-gamma}
Let $\mathcal{C}$ be a site with fibre products.
Let $X$ be an object of $\mathcal{C}$.
Let $K, L, M$ be simplicial objects of $\text{SR}(\mathcal{C}, X)$.
Let $a : K \to L$, $b : M \to L$ be morphisms.
Assume
\begin{enumerate}
\item $K$ is a hypercovering of $X$,
\item the morphism $M_0 \to L_0$ is a covering, and
\item for all $n \geq 0$ in the diagram
$$
\xymatrix{
M_{n + 1} \ar[dd] \ar[rr] \ar[rd]^\gamma &
&
(\text{cosk}_n \text{sk}_n M)_{n + 1} \ar[dd] \\
&
L_{n + 1}
\times_{(\text{cosk}_n \text{sk}_n L)_{n + 1}}
(\text{cosk}_n \text{sk}_n M)_{n + 1}
\ar[ld] \ar[ru]
& \\
L_{n + 1} \ar[rr] & & (\text{cosk}_n \text{sk}_n L)_{n + 1}
}
$$
the arrow $\gamma$ is a covering.
\end{enumerate}
Then the fibre product $K \times_L M$ is a hypercovering of $X$.
\end{lemma}

\begin{proof}
The morphism $(K \times_L M)_0 = K_0 \times_{L_0} M_0 \to K_0$
is a base change of a covering by (2), hence a covering, see
Lemma \ref{lemma-covering-permanence}. And $K_0 \to \{X \to X\}$
is a covering by (1). Thus $(K \times_L M)_0 \to \{X \to X\}$
is a covering by Lemma \ref{lemma-covering-permanence}. Hence
$K \times_L M$ satisfies the first condition of Definition
\ref{definition-hypercovering}.

\medskip\noindent
We still have to check that
$$
K_{n + 1} \times_{L_{n + 1}} M_{n + 1} = (K \times_L M)_{n + 1}
\longrightarrow
(\text{cosk}_n \text{sk}_n (K \times_L M))_{n + 1}
$$
is a covering for all $n \geq 0$. We abbreviate as follows:
$A = (\text{cosk}_n \text{sk}_n K)_{n + 1}$,
$B = (\text{cosk}_n \text{sk}_n L)_{n + 1}$, and
$C = (\text{cosk}_n \text{sk}_n M)_{n + 1}$.
The functor $\text{cosk}_n \text{sk}_n$ commutes with fibre products,
see Simplicial, Lemma \ref{simplicial-lemma-cosk-fibre-product}.
Thus the right hand side above is equal to $A \times_B C$.
Consider the following commutative diagram
$$
\xymatrix{
K_{n + 1} \times_{L_{n + 1}} M_{n + 1} \ar[r] \ar[d] &
M_{n + 1} \ar[d] \ar[rd]_\gamma \ar[rrd] &
& \\
K_{n + 1} \ar[r] \ar[rd] &
L_{n + 1} \ar[rrd] &
L_{n + 1} \times_B C \ar[l] \ar[r] &
C \ar[d] \\
&
A \ar[rr] &
&
B
}
$$
This diagram shows that
$$
K_{n + 1} \times_{L_{n + 1}} M_{n + 1}
=
(K_{n + 1} \times_B C)
\times_{(L_{n + 1} \times_B C), \gamma}
M_{n + 1}
$$
Now, $K_{n + 1} \times_B C \to A \times_B C$
is a base change of the covering $K_{n + 1} \to A$
via the morphism $A \times_B C \to A$, hence is a
covering. By assumption (3) the morphism $\gamma$ is a covering.
Hence the morphism
$$
(K_{n + 1} \times_B C)
\times_{(L_{n + 1} \times_B C), \gamma}
M_{n + 1}
\longrightarrow
K_{n + 1} \times_B C
$$
is a covering as a base change of a covering.
The lemma follows as a composition of coverings
is a covering.
\end{proof}

\begin{lemma}
\label{lemma-product-hypercoverings}
Let $\mathcal{C}$ be a site with fibre products.
Let $X$ be an object of $\mathcal{C}$.
If $K, L$ are hypercoverings of $X$, then
$K \times L$ is a hypercovering of $X$.
\end{lemma}

\begin{proof}
You can either verify this directly, or use
Lemma \ref{lemma-funny-gamma} above and check that $L \to \{X \to X\}$
has property (3).
\end{proof}


\noindent
Let $\mathcal{C}$ be a site with fibre products.
Let $X$ be an object of $\mathcal{C}$.
Since the category $\text{SR}(\mathcal{C}, X)$ has coproducts and
finite limits, it is permissible to speak about the objects
$U \times K$ and $\Hom(U, K)$ for certain simplicial sets $U$
(for example those with finitely many nondegenerate simplices)
and any simplicial object $K$ of $\text{SR}(\mathcal{C}, X)$.
See Simplicial, Sections
\ref{simplicial-section-product-with-simplicial-sets} and
\ref{simplicial-section-hom-from-simplicial-sets}.

\begin{lemma}
\label{lemma-covering}
Let $\mathcal{C}$ be a site with fibre products.
Let $X$ be an object of $\mathcal{C}$.
Let $K$ be a hypercovering of $X$.
Let $k \geq 0$ be an integer.
Let $u : Z \to K_k$ be a covering in
in $\text{SR}(\mathcal{C}, X)$.
Then there exists a morphism of hypercoverings
$f: L \to K$ such that $L_k \to K_k$
factors through $u$.
\end{lemma}

\begin{proof}
Denote $Y = K_k$. Let $C[k]$ be the cosimplicial set defined in
Simplicial, Example \ref{simplicial-example-simplex-cosimplicial-set}.
We will use the description of $\Hom(C[k], Y)$ and $\Hom(C[k], Z)$
given in
Simplicial, Lemma \ref{simplicial-lemma-morphism-into-product}.
There is a canonical morphism
$K \to \Hom(C[k], Y)$ corresponding to $\text{id} : K_k = Y \to Y$.
Consider the morphism $\Hom(C[k], Z) \to \Hom(C[k], Y)$
which on degree $n$ terms is the morphism
$$
\prod\nolimits_{\alpha : [k] \to [n]} Z
\longrightarrow
\prod\nolimits_{\alpha : [k] \to [n]} Y
$$
using the given morphism $Z \to Y$ on each factor. Set
$$
L = K \times_{\Hom(C[k], Y)} \Hom(C[k], Z).
$$
The morphism $L_k \to K_k$ sits in to a commutative diagram
$$
\xymatrix{
L_k \ar[r] \ar[d] &
\prod_{\alpha : [k] \to [k]} Z \ar[r]^-{\text{pr}_{\text{id}_{[k]}}} \ar[d] &
Z \ar[d] \\
K_k \ar[r] &
\prod_{\alpha : [k] \to [k]} Y \ar[r]^-{\text{pr}_{\text{id}_{[k]}}} &
Y
}
$$
Since the composition of the two bottom arrows is the identity
we conclude that we have the desired factorization.

\medskip\noindent
We still have to show that $L$ is a hypercovering of $X$.
To see this we will use Lemma \ref{lemma-funny-gamma}.
Condition (1) is satisfied by assumption.
For (2), the morphism
$$
\Hom(C[k], Z)_0 \to \Hom(C[k], Y)_0
$$
is a covering because it is isomorphic to $Z \to Y$ as
there is only one morphism $[k] \to [0]$.

\medskip\noindent
Let us consider condition (3) for $n = 0$. Then, since
$(\text{cosk}_0 T)_1 = T \times T$
(Simplicial, Example \ref{simplicial-example-cosk0})
and since $\Hom(C[k], Z)_1 = \prod_{\alpha : [k] \to [1]} Z$
we obtain the diagram
$$
\xymatrix{
\prod\nolimits_{\alpha : [k] \to [1]} Z \ar[r] \ar[d] &
Z \times Z \ar[d] \\
\prod\nolimits_{\alpha : [k] \to [1]} Y \ar[r] &
Y \times Y
}
$$
with horizontal arrows corresponding to the projection onto the factors
corresponding to the two nonsurjective $\alpha$. Thus the arrow $\gamma$
is the morphism
$$
\prod\nolimits_{\alpha : [k] \to [1]} Z
\longrightarrow
\prod\nolimits_{\alpha : [k] \to [1]\text{ not onto}} Z
\times
\prod\nolimits_{\alpha : [k] \to [1]\text{ onto}} Y
$$
which is a product of coverings and hence a covering by
Lemma \ref{lemma-covering-permanence}.

\medskip\noindent
Let us consider condition (3) for $n > 0$. We claim there is an
injective map $\tau : S' \to S$ of finite sets, such that for any
object $T$ of $\text{SR}(\mathcal{C}, X)$ the morphism
\begin{equation}
\label{equation-map}
\Hom(C[k], T)_{n + 1} \to
(\text{cosk}_n \text{sk}_n \Hom(C[k], T))_{n + 1}
\end{equation}
is isomorphic to the projection $\prod_{s \in S} T \to \prod_{s' \in S'} T$
functorially in $T$. If this is true, then we see, arguing as in the previous
paragraph, that the arrow $\gamma$ is the morphism
$$
\prod\nolimits_{s \in S} Z
\longrightarrow
\prod\nolimits_{s \in S'} Z
\times
\prod\nolimits_{s \not\in \tau(S')} Y
$$
which is a product of coverings and hence a covering by
Lemma \ref{lemma-covering-permanence}. By construction, we have
$\Hom(C[k], T)_{n + 1} = \prod_{\alpha : [k] \to [n + 1]} T$
(see Simplicial, Lemma \ref{simplicial-lemma-morphism-into-product}).
Correspondingly we take $S = \text{Map}([k], [n + 1])$.
On the other hand, Simplicial, Lemma \ref{simplicial-lemma-formula-limit},
provides a description of points of
$(\text{cosk}_n \text{sk}_n \Hom(C[k], T))_{n + 1}$
as sequences $(f_0, \ldots, f_{n + 1})$ of points of $\Hom(C[k], T)_n$
satisfying $d^n_{j - 1} f_i = d^n_i f_j$ for $0 \leq i < j \leq n + 1$.
We can write $f_i = (f_{i, \alpha})$ with $f_{i, \alpha}$ a point of $T$
and $\alpha \in \text{Map}([k], [n])$. The conditions translate into
$$
f_{i, \delta^n_{j - 1} \circ \beta} = f_{j, \delta_i^n \circ \beta}
$$
for any $0 \leq i < j \leq n + 1$ and $\beta : [k] \to [n - 1]$. Thus we
see that
$$
S' = \{0, \ldots, n + 1\} \times \text{Map}([k], [n]) / \sim
$$
where the equivalence relation is generated by the equivalences
$$
(i, \delta^n_{j - 1} \circ \beta) \sim (j, \delta_i^n \circ \beta)
$$
for $0 \leq i < j \leq n + 1$ and $\beta : [k] \to [n - 1]$.
A computation (omitted) shows that the morphism (\ref{equation-map})
corresponds to the map $S' \to S$ which sends $(i, \alpha)$ to
$\delta^{n + 1}_i \circ \alpha \in S$. (It may be a comfort to the
reader to see that this map is well defined by part (1) of
Simplicial, Lemma \ref{simplicial-lemma-relations-face-degeneracy}.)
To finish the proof it suffices to show that if
$\alpha, \alpha' : [k] \to [n]$ and $0 \leq i < j \leq n + 1$
are such that
$$
\delta^{n + 1}_i \circ \alpha = \delta^{n + 1}_j \circ \alpha'
$$
then we have $\alpha = \delta^n_{j - 1} \circ \beta$
and $\alpha' = \delta_i^n \circ \beta$ for some $\beta : [k] \to [n - 1]$.
This is easy to see and omitted.
\end{proof}

\begin{lemma}
\label{lemma-covering-sheaf}
Let $\mathcal{C}$ be a site with fibre products.
Let $X$ be an object of $\mathcal{C}$.
Let $K$ be a hypercovering of $X$.
Let $n \geq 0$ be an integer.
Let $u : \mathcal{F} \to F(K_n)$ be a morphism
of presheaves which becomes surjective
on sheafification.
Then there exists a morphism of hypercoverings
$f: L \to K$ such that $F(f_n) : F(L_n) \to F(K_n)$
factors through $u$.
\end{lemma}

\begin{proof}
Write $K_n = \{U_i \to X\}_{i \in I}$.
Thus the map $u$ is a morphism of presheaves of sets
$u : \mathcal{F} \to \amalg h_{u_i}$.
The assumption on $u$ means that for every
$i \in I$ there exists a covering $\{U_{ij} \to U_i\}_{j \in I_i}$
of the site $\mathcal{C}$ and a morphism of presheaves
$t_{ij} : h_{U_{ij}} \to \mathcal{F}$ such that
$u \circ t_{ij}$ is the map $h_{U_{ij}} \to h_{U_i}$
coming from the morphism $U_{ij} \to U_i$.
Set $J = \amalg_{i \in I} I_i$, and let
$\alpha : J \to I$ be the obvious map.
For $j \in J$ denote $V_j = U_{\alpha(j)j}$. Set
$Z = \{V_j \to X\}_{j \in J}$.
Finally, consider the morphism
$u' : Z \to K_n$ given by $\alpha : J \to I$
and the morphisms $V_j = U_{\alpha(j)j} \to U_{\alpha(j)}$
above. Clearly, this is a covering in the
category $\text{SR}(\mathcal{C}, X)$, and by
construction $F(u') : F(Z) \to F(K_n)$ factors through $u$.
Thus the result follows from Lemma \ref{lemma-covering} above.
\end{proof}


\section{Adding simplices}
\label{section-adding-simplices}

\noindent
In this section we prove some technical lemmas which we will need later.
Let $\mathcal{C}$ be a site with fibre products.
Let $X$ be an object of $\mathcal{C}$.
As we pointed out in Section \ref{section-covering} above,
the objects $U \times K$ and $\Hom(U, K)$
for certain simplicial sets $U$
and any simplicial object $K$ of $\text{SR}(\mathcal{C}, X)$
are defined. See Simplicial, Sections
\ref{simplicial-section-product-with-simplicial-sets} and
\ref{simplicial-section-hom-from-simplicial-sets}.

\begin{lemma}
\label{lemma-one-more-simplex}
Let $\mathcal{C}$ be a site with fibre products.
Let $X$ be an object of $\mathcal{C}$.
Let $K$ be a hypercovering of $X$.
Let $U \subset V$ be simplicial sets, with $U_n, V_n$
finite nonempty for all $n$.
Assume that $U$ has finitely many nondegenerate simplices.
Suppose $n \geq 0$ and $x \in V_n$,
$x \not \in U_n$ are such that
\begin{enumerate}
\item $V_i = U_i$ for $i < n$,
\item $V_n = U_n \cup \{x\}$,
\item any $z \in V_j$, $z \not \in U_j$ for $j > n$
is degenerate.
\end{enumerate}
Then the morphism
$$
\Hom(V, K)_0
\longrightarrow
\Hom(U, K)_0
$$
of $\text{SR}(\mathcal{C}, X)$ is a covering.
\end{lemma}

\begin{proof}
If $n = 0$, then it follows easily that $V = U \amalg \Delta[0]$
(see below). In this case $\Hom(V, K)_0 =
\Hom(U, K)_0 \times K_0$. The result, in this case, then follows
from Lemma \ref{lemma-covering-permanence}.

\medskip\noindent
Let $a : \Delta[n] \to V$ be the morphism associated to $x$
as in Simplicial, Lemma \ref{simplicial-lemma-simplex-map}.
Let us write $\partial \Delta[n] = i_{(n-1)!} \text{sk}_{n - 1} \Delta[n]$
for the $(n - 1)$-skeleton of $\Delta[n]$.
Let $b : \partial \Delta[n] \to U$ be the restriction
of $a$ to the $(n - 1)$ skeleton of $\Delta[n]$. By
Simplicial, Lemma \ref{simplicial-lemma-glue-simplex}
we have $V = U \amalg_{\partial \Delta[n]} \Delta[n]$. By
Simplicial, Lemma
\ref{simplicial-lemma-hom-from-coprod}
we get that
$$
\xymatrix{
\Hom(V, K)_0 \ar[r] \ar[d] &
\Hom(U, K)_0 \ar[d] \\
\Hom(\Delta[n], K)_0 \ar[r] &
\Hom(\partial \Delta[n], K)_0
}
$$
is a fibre product square. Thus it suffices to show that
the bottom horizontal arrow is a covering. By
Simplicial, Lemma \ref{simplicial-lemma-cosk-shriek}
this arrow is identified with
$$
K_n \to (\text{cosk}_{n - 1} \text{sk}_{n - 1} K)_n
$$
and hence is a covering by definition of a hypercovering.
\end{proof}

\begin{lemma}
\label{lemma-add-simplices}
Let $\mathcal{C}$ be a site with fibre products.
Let $X$ be an object of $\mathcal{C}$.
Let $K$ be a hypercovering of $X$.
Let $U \subset V$ be simplicial sets, with $U_n, V_n$
finite nonempty for all $n$.
Assume that $U$ and $V$ have finitely many nondegenerate simplices.
Then the morphism
$$
\Hom(V, K)_0
\longrightarrow
\Hom(U, K)_0
$$
of $\text{SR}(\mathcal{C}, X)$ is a covering.
\end{lemma}

\begin{proof}
By Lemma \ref{lemma-one-more-simplex}
above, it suffices to prove a simple lemma
about inclusions of simplicial sets $U \subset V$ as in the
lemma. And this is exactly the result of
Simplicial, Lemma \ref{simplicial-lemma-add-simplices}.
\end{proof}

\begin{lemma}
\label{lemma-degeneracy-maps-coverings}
Let $\mathcal{C}$ be a site with fibre products. Let $X$ be an object of
$\mathcal{C}$. Let $K$ be a hypercovering of $X$. Then
\begin{enumerate}
\item $K_n$ is a covering of $X$ for each $n \geq 0$,
\item $d^n_i : K_n \to K_{n - 1}$ is a covering for all $n \geq 1$
and $0 \leq i \leq n$.
\end{enumerate}
\end{lemma}

\begin{proof}
Recall that $K_0$ is a covering of $X$ by
Definition \ref{definition-hypercovering}
and that this is equivalent to saying that
$K_0 \to \{X \to X\}$ is a covering in the sense
of Definition \ref{definition-covering-SR}.
Hence (1) follows from (2) because it will prove that
the composition
$K_n \to K_{n - 1} \to \ldots \to K_0 \to \{X \to X\}$
is a covering by Lemma \ref{lemma-covering-permanence}.

\medskip\noindent
Proof of (2). Observe that
$\Mor(\Delta[n], K)_0 = K_n$ by
Simplicial, Lemma \ref{simplicial-lemma-exists-hom-from-simplicial-set-finite}.
Therefore (2) follows from Lemma \ref{lemma-add-simplices}
applied to the $n + 1$ different inclusions $\Delta[n - 1] \to \Delta[n]$.
\end{proof}

\begin{remark}
\label{remark-P-covering}
A useful special case of Lemmas \ref{lemma-add-simplices} and
\ref{lemma-degeneracy-maps-coverings} is the following.
Suppose we have a category $\mathcal{C}$ having fibre products.
Let $P \subset \text{Arrows}(\mathcal{C})$ be a subset
stable under base change, stable under composition,
and containing all isomorphisms. Then one says a
{\it $P$-hypercovering} is an augmentation $a : U \to X$
from a simplicial object of $\mathcal{C}$ such that
\begin{enumerate}
\item $U_0 \to X$ is in $P$,
\item $U_1 \to U_0 \times_X U_0$ is in $P$,
\item $U_{n + 1} \to (\text{cosk}_n\text{sk}_n U)_{n + 1}$
is in $P$ for $n \geq 1$.
\end{enumerate}
The category $\mathcal{C}/X$ has all finite limits, hence the
coskeleta used in the formulation above exist
(see Categories, Lemma \ref{categories-lemma-finite-limits-exist}).
Then we claim that the morphisms $U_n \to X$ and $d^n_i : U_n \to U_{n - 1}$
are in $P$. This follows from the aforementioned
lemmas by turning $\mathcal{C}$ into a site whose coverings
are $\{f : V \to U\}$ with $f \in P$ and taking $K$ given by
$K_n = \{U_n \to X\}$.
\end{remark}


\section{Homotopies}
\label{section-homotopies}

\noindent
Let $\mathcal{C}$ be a site with fibre products.
Let $X$ be an object of $\mathcal{C}$.
Let $L$ be a simplicial object of $\text{SR}(\mathcal{C}, X)$.
According to
Simplicial, Lemma \ref{simplicial-lemma-exists-hom-from-simplicial-set-finite}
there exists an object $\Hom(\Delta[1], L)$
in the category $\text{Simp}(\text{SR}(\mathcal{C}, X))$ which represents the
functor
$$
T
\longmapsto
\Mor_{\text{Simp}(\text{SR}(\mathcal{C}, X))}(\Delta[1] \times T, L)
$$
There is a canonical morphism
$$
\Hom(\Delta[1], L) \to L \times L
$$
coming from $e_i : \Delta[0] \to \Delta[1]$ and the identification
$\Hom(\Delta[0], L) = L$.

\begin{lemma}
\label{lemma-hom-hypercovering}
Let $\mathcal{C}$ be a site with fibre products.
Let $X$ be an object of $\mathcal{C}$.
Let $L$ be a simplicial object of $\text{SR}(\mathcal{C}, X)$.
Let $n \geq 0$. Consider the commutative diagram
\begin{equation}
\label{equation-diagram}
\xymatrix{
\Hom(\Delta[1], L)_{n + 1} \ar[r] \ar[d] &
(\text{cosk}_n \text{sk}_n \Hom(\Delta[1], L))_{n + 1} \ar[d] \\
(L \times L)_{n + 1} \ar[r] &
(\text{cosk}_n \text{sk}_n (L \times L))_{n + 1}
}
\end{equation}
coming from the morphism defined above.
We can identify the terms in this diagram as follows,
where
$\partial \Delta[n + 1] = i_{n!}\text{sk}_n \Delta[n + 1]$
is the $n$-skeleton of the $(n + 1)$-simplex:
\begin{eqnarray*}
\Hom(\Delta[1], L)_{n + 1}
& = &
\Hom(\Delta[1] \times \Delta[n + 1], L)_0 \\
(\text{cosk}_n \text{sk}_n \Hom(\Delta[1], L))_{n + 1}
& = &
\Hom(\Delta[1] \times \partial \Delta[n + 1], L)_0 \\
(L \times L)_{n + 1}
& = &
\Hom(
(\Delta[n + 1] \amalg \Delta[n + 1], L)_0 \\
(\text{cosk}_n \text{sk}_n (L \times L))_{n + 1}
& = &
\Hom(
\partial \Delta[n + 1]
\amalg
\partial \Delta[n + 1], L)_0
\end{eqnarray*}
and the morphism between these objects of $\text{SR}(\mathcal{C}, X)$
come from the commutative diagram of simplicial sets
\begin{equation}
\label{equation-dual-diagram}
\xymatrix{
\Delta[1] \times \Delta[n + 1] &
\Delta[1] \times \partial\Delta[n + 1] \ar[l] \\
\Delta[n + 1] \amalg \Delta[n + 1] \ar[u] &
\partial\Delta[n + 1] \amalg \partial\Delta[n + 1]
\ar[l] \ar[u]
}
\end{equation}
Moreover the fibre product of the bottom arrow and the
right arrow in (\ref{equation-diagram}) is equal to
$$
\Hom(U, L)_0
$$
where $U \subset \Delta[1] \times \Delta[n + 1]$
is the smallest simplicial subset such that both
$\Delta[n + 1] \amalg \Delta[n + 1]$ and
$\Delta[1] \times \partial\Delta[n + 1]$ map into it.
\end{lemma}

\begin{proof}
The first and third equalities are
Simplicial, Lemma \ref{simplicial-lemma-exists-hom-from-simplicial-set-finite}.
The second and fourth follow from the cited lemma combined with
Simplicial, Lemma \ref{simplicial-lemma-cosk-shriek}.
The last assertion follows from the fact that
$U$ is the push-out of the bottom and right arrow of the
diagram (\ref{equation-dual-diagram}), via
Simplicial, Lemma \ref{simplicial-lemma-hom-from-coprod}.
To see that $U$ is equal to this push-out it suffices
to see that the intersection of
$\Delta[n + 1] \amalg \Delta[n + 1]$ and
$\Delta[1] \times \partial\Delta[n + 1]$
in $\Delta[1] \times \Delta[n + 1]$ is equal to
$\partial\Delta[n + 1] \amalg \partial\Delta[n + 1]$.
This we leave to the reader.
\end{proof}

\begin{lemma}
\label{lemma-homotopy}
Let $\mathcal{C}$ be a site with fibre products.
Let $X$ be an object of $\mathcal{C}$.
Let $K, L$ be hypercoverings of $X$.
Let $a, b : K \to L$ be morphisms of hypercoverings.
There exists a morphism of hypercoverings
$c : K' \to K$ such that $a \circ c$ is homotopic
to $b \circ c$.
\end{lemma}

\begin{proof}
Consider the following commutative diagram
$$
\xymatrix{
K' \ar@{=}[r]^-{def} \ar[rd]_c &
K \times_{(L \times L)} \Hom(\Delta[1], L)
\ar[r] \ar[d] & \Hom(\Delta[1], L) \ar[d] \\
& K \ar[r]^{(a, b)} & L \times L
}
$$
By the functorial property of $\Hom(\Delta[1], L)$
the composition of the horizontal morphisms
corresponds to a morphism $K' \times \Delta[1] \to L$ which
defines a homotopy between $c \circ a$ and $c \circ b$.
Thus if we can show that $K'$ is a
hypercovering of $X$, then we obtain the lemma.
To see this we will apply Lemma \ref{lemma-funny-gamma}
to the pair of morphisms $K \to L \times L$
and $\Hom(\Delta[1], L) \to L \times L$.
Condition (1) of Lemma \ref{lemma-funny-gamma} is satisfied.
Condition (2) of Lemma \ref{lemma-funny-gamma} is true because
$\Hom(\Delta[1], L)_0 = L_1$, and the morphism
$(d^1_0, d^1_1) : L_1 \to L_0 \times L_0$ is a
covering of $\text{SR}(\mathcal{C}, X)$ by our
assumption that $L$ is a hypercovering.
To prove condition (3) of Lemma \ref{lemma-funny-gamma}
we use Lemma \ref{lemma-hom-hypercovering} above. According
to this lemma the morphism $\gamma$ of condition (3) of Lemma
\ref{lemma-funny-gamma} is the morphism
$$
\Hom(\Delta[1] \times \Delta[n + 1], L)_0
\longrightarrow
\Hom(U, L)_0
$$
where $U \subset \Delta[1] \times \Delta[n + 1]$.
According to Lemma \ref{lemma-add-simplices}
this is a covering and hence the claim has been proven.
\end{proof}

\begin{remark}
\label{remark-contractible-category}
Note that the crux of the proof is to use
Lemma \ref{lemma-add-simplices}. This lemma
is completely general and does not care about the
exact shape of the simplicial sets (as long as they
have only finitely many nondegenerate simplices).
It seems altogether reasonable to expect a result
of the following kind:
Given any morphism $a : K \times \partial \Delta[k]
\to L$, with $K$ and $L$ hypercoverings, there
exists a morphism of hypercoverings $c : K' \to K$
and a morphism  $g : K' \times \Delta[k] \to L$
such that
$g|_{K' \times \partial \Delta[k]} =
a \circ (c \times \text{id}_{\partial \Delta[k]})$.
In other words, the category of hypercoverings is in
a suitable sense contractible.
\end{remark}

















\section{Cohomology and hypercoverings}
\label{section-cohomology}

\noindent
Let $\mathcal{C}$ be a site with fibre products.
Let $X$ be an object of $\mathcal{C}$.
Let $\mathcal{F}$ be a sheaf of abelian groups on $\mathcal{C}$.
Let $K, L$ be hypercoverings of $X$.
If $a, b : K \to L$ are homotopic maps,
then $\mathcal{F}(a), \mathcal{F}(b) : \mathcal{F}(K) \to \mathcal{F}(L)$
are homotopic maps, see
Simplicial, Lemma \ref{simplicial-lemma-functorial-homotopy}.
Hence have the same effect on cohomology groups of the associated
cochain complexes, see
Simplicial, Lemma \ref{simplicial-lemma-homotopy-s-Q}.
We are going to use this to define the colimit over all
hypercoverings.

\medskip\noindent
Let us temporarily denote $\text{HC}(\mathcal{C}, X)$
the category of hypercoverings of $X$. We have seen that
this is a category and not a ``big'' category,
see Lemma \ref{lemma-hypercoverings-set}.
This will be the index category for our diagram, see
Categories, Section \ref{categories-section-limits} for notation.
Consider the diagram
$$
\check{H}^i(-, \mathcal{F}) :
\text{HC}(\mathcal{C}, X)
\longrightarrow
\textit{Ab}.
$$
By Lemma \ref{lemma-product-hypercoverings} and
Lemma \ref{lemma-homotopy}, and the remark on homotopies above,
this diagram is directed, see
Categories, Definition \ref{categories-definition-directed}.
Thus the colimit
$$
\check{H}^i_{\text{HC}}(X, \mathcal{F})
=
\colim_{K \in \text{HC}(\mathcal{C}, X)}
\check{H}^i(K, \mathcal{F})
$$
has a particularly simple description (see location cited).

\begin{theorem}
\label{theorem-cohomology-hypercoverings}
Let $\mathcal{C}$ be a site with fibre products.
Let $X$ be an object of $\mathcal{C}$. Let $i \geq 0$.
The functors
\begin{eqnarray*}
\textit{Ab}(\mathcal{C}) & \longrightarrow & \textit{Ab} \\
\mathcal{F} & \longmapsto & H^i(X, \mathcal{F}) \\
\mathcal{F} & \longmapsto & \check{H}^i_{\text{HC}}(X, \mathcal{F})
\end{eqnarray*}
are canonically isomorphic.
\end{theorem}

\begin{proof}[Proof using spectral sequences.]
Suppose that $\xi \in H^p(X, \mathcal{F})$ for some $p \geq 0$.
Let us show that $\xi$ is in the image of the map
$\check{H}^p(X, \mathcal{F}) \to H^p(X, \mathcal{F})$ of
Lemma \ref{lemma-cech-spectral-sequence}
for some hypercovering $K$ of $X$.

\medskip\noindent
This is true if $p = 0$ by Lemma \ref{lemma-h0-cech}.
If $p = 1$, choose a {\v C}ech hypercovering $K$ of $X$ as in
Example \ref{example-cech} starting with a covering
$K_0 = \{U_i \to X\}$ in the site $\mathcal{C}$ such that
$\xi|_{U_i} = 0$, see
Cohomology on Sites,
Lemma \ref{sites-cohomology-lemma-kill-cohomology-class-on-covering}.
It follows immediately from the spectral sequence
in Lemma \ref{lemma-cech-spectral-sequence} that $\xi$ comes
from an element of $\check{H}^1(K, \mathcal{F})$ in this case.
In general, choose any hypercovering $K$ of $X$ such
that $\xi$ maps to zero in $\underline{H}^p(\mathcal{F})(K_0)$
(using Example \ref{example-cech} and
Cohomology on Sites,
Lemma \ref{sites-cohomology-lemma-kill-cohomology-class-on-covering}
again).
By the spectral sequence of Lemma \ref{lemma-cech-spectral-sequence}
the obstruction for $\xi$ to come from an element of
$\check{H}^p(K, \mathcal{F})$ is a sequence of elements
$\xi_1, \ldots, \xi_{p - 1}$ with
$\xi_q \in \check{H}^{p - q}(K, \underline{H}^q(\mathcal{F}))$
(more precisely the images of the $\xi_q$ in certain subquotients
of these groups).

\medskip\noindent
We can inductively replace the hypercovering $K$ by refinements
such that the obstructions $\xi_1, \ldots, \xi_{p - 1}$ restrict to zero
(and not just the images
in the subquotients -- so no subtlety here). Indeed, suppose we have
already managed to reach the situation where
$\xi_{q + 1}, \ldots, \xi_{p - 1}$ are zero.
Note that $\xi_q \in \check{H}^{p - q}(K, \underline{H}^q(\mathcal{F}))$
is the class of some element
$$
\tilde \xi_q \in
\underline{H}^q(\mathcal{F})(K_{p - q}) =
\prod H^q(U_i, \mathcal{F})
$$
if $K_{p - q} = \{U_i \to X\}_{i \in I}$. Let $\xi_{q, i}$
be the component of $\tilde \xi_q$ in $H^q(U_i, \mathcal{F})$.
As $q \geq 1$ we can use
Cohomology on Sites,
Lemma \ref{sites-cohomology-lemma-kill-cohomology-class-on-covering}
yet again to choose coverings $\{U_{i, j} \to U_i\}$
of the site such that each restriction $\xi_{q, i}|_{U_{i, j}} = 0$.
Consider the object $Z = \{U_{i, j} \to X\}$ of the category
$\text{SR}(\mathcal{C}, X)$ and its obvious morphism
$u : Z \to K_{p - q}$. It is clear that $u$ is a covering, see
Definition \ref{definition-covering-SR}. By
Lemma \ref{lemma-covering} there
exists a morphism $L \to K$ of hypercoverings of $X$ such that
$L_{p - q} \to K_{p - q}$ factors through $u$. Then clearly the
image of $\xi_q$ in $\underline{H}^q(\mathcal{F})(L_{p - q})$.
is zero. Since the spectral sequence of
Lemma \ref{lemma-cech-spectral-sequence}
is functorial this means that after replacing $K$ by $L$ we reach the
situation where $\xi_q, \ldots, \xi_{p - 1}$ are all zero.
Continuing like this we end up with a hypercovering where they are all
zero and hence $\xi$ is in the image of the map
$\check{H}^p(X, \mathcal{F}) \to H^p(X, \mathcal{F})$.

\medskip\noindent
Suppose that $K$ is a hypercovering of $X$, that
$\xi \in \check{H}^p(K, \mathcal{F})$ and that the image of
$\xi$ under the map
$\check{H}^p(X, \mathcal{F}) \to H^p(X, \mathcal{F})$ of
Lemma \ref{lemma-cech-spectral-sequence}
is zero. To finish the proof of the theorem we have to show that
there exists a morphism of hypercoverings $L \to K$ such that
$\xi$ restricts to zero in $\check{H}^p(L, \mathcal{F})$.
By the spectral sequence of Lemma \ref{lemma-cech-spectral-sequence}
the vanishing of the image of $\xi$ in $H^p(X, \mathcal{F})$
means that there exist elements $\xi_1, \ldots, \xi_{p - 2}$
with $\xi_q \in \check{H}^{p - 1 - q}(K, \underline{H}^q(\mathcal{F}))$
(more precisely the images of these in certain subquotients)
such that the images $d_{q + 1}^{p - 1 - q, q}\xi_q$ (in the spectral
sequence) add up to $\xi$. Hence by exactly the same mechanism as above
we can find a morphism of hypercoverings $L \to K$ such that
the restrictions of the elements $\xi_q$, $q = 1, \ldots, p - 2$
in $\check{H}^{p - 1 - q}(L, \underline{H}^q(\mathcal{F}))$ are zero.
Then it follows that $\xi$ is zero since the morphism $L \to K$
induces a morphism of spectral sequences according to
Lemma \ref{lemma-cech-spectral-sequence}.
\end{proof}

\begin{proof}[Proof without using spectral sequences.]
We have seen the result for $i = 0$, see Lemma \ref{lemma-h0-cech}.
We know that the functors $H^i(X, -)$ form a universal $\delta$-functor, see
Derived Categories, Lemma \ref{derived-lemma-higher-derived-functors}.
In order to prove the theorem it suffices to show that
the sequence of functors $\check{H}^i_{HC}(X, -)$ forms a
$\delta$-functor. Namely we know that {\v C}ech cohomology
is zero on injective sheaves (Lemma \ref{lemma-injective-trivial-cech})
and then we can apply
Homology, Lemma \ref{homology-lemma-efface-implies-universal}.

\medskip\noindent
Let
$$
0 \to \mathcal{F} \to \mathcal{G} \to \mathcal{H} \to 0
$$
be a short exact sequence of abelian sheaves on $\mathcal{C}$.
Let $\xi \in \check{H}^p_{HC}(X, \mathcal{H})$. Choose a hypercovering
$K$ of $X$ and an element $\sigma \in \mathcal{H}(K_p)$ representing
$\xi$ in cohomology. There is a corresponding exact sequence of
complexes
$$
0 \to s(\mathcal{F}(K)) \to s(\mathcal{G}(K)) \to s(\mathcal{H}(K))
$$
but we are not assured that there is a zero on the right also and this
is the only thing that
prevents us from defining $\delta(\xi)$ by a simple application of the
snake lemma. Recall that
$$
\mathcal{H}(K_p) = \prod \mathcal{H}(U_i)
$$
if $K_p = \{U_i \to X\}$. Let $\sigma =\prod \sigma_i$ with
$\sigma_i \in \mathcal{H}(U_i)$. Since $\mathcal{G} \to \mathcal{H}$ is
a surjection of sheaves we see that there exist coverings
$\{U_{i, j} \to U_i\}$ such that $\sigma_i|_{U_{i, j}}$ is the
image of some element $\tau_{i, j} \in \mathcal{G}(U_{i, j})$.
Consider the object $Z = \{U_{i, j} \to X\}$ of the category
$\text{SR}(\mathcal{C}, X)$ and its obvious morphism
$u : Z \to K_p$. It is clear that $u$ is a covering, see
Definition \ref{definition-covering-SR}. By
Lemma \ref{lemma-covering} there
exists a morphism $L \to K$ of hypercoverings of $X$ such that
$L_p \to K_p$ factors through $u$. After replacing $K$ by $L$
we may therefore assume that $\sigma$ is the image of an
element $\tau \in \mathcal{G}(K_p)$. Note that $d(\sigma) = 0$,
but not necessarily $d(\tau) = 0$. Thus $d(\tau) \in \mathcal{F}(K_{p + 1})$
is a cocycle. In this situation we define
$\delta(\xi)$ as the class of the cocycle $d(\tau)$ in
$\check{H}^{p + 1}_{HC}(X, \mathcal{F})$.

\medskip\noindent
At this point there are several things to verify:
(a) $\delta(\xi)$ does not depend on the choice of $\tau$,
(b) $\delta(\xi)$ does not depend on the choice of the hypercovering
$L \to K$ such that $\sigma$ lifts, and
(c) $\delta(\xi)$ does not depend on the initial hypercovering and
$\sigma$ chosen to represent $\xi$. We omit the verification of
(a), (b), and (c); the independence of the choices of the hypercoverings
really comes down to Lemmas \ref{lemma-product-hypercoverings}
and \ref{lemma-homotopy}. We also omit the verification that
$\delta$ is functorial with respect to morphisms of short exact
sequences of abelian sheaves on $\mathcal{C}$.

\medskip\noindent
Finally, we have to verify that with this definition of $\delta$
our short exact sequence of abelian sheaves above leads to a
long exact sequence of {\v C}ech cohomology groups.
First we show that if $\delta(\xi) = 0$ (with $\xi$ as above) then
$\xi$ is the image of some element
$\xi' \in \check{H}^p_{HC}(X, \mathcal{G})$.
Namely, if $\delta(\xi) = 0$, then, with notation as above, we
see that the class of $d(\tau)$ is zero in
$\check{H}^{p + 1}_{HC}(X, \mathcal{F})$. Hence there exists
a morphism of hypercoverings $L \to K$ such that the restriction
of $d(\tau)$ to an element of $\mathcal{F}(L_{p + 1})$ is
equal to $d(\upsilon)$ for some $\upsilon \in \mathcal{F}(L_p)$.
This implies that $\tau|_{L_p} + \upsilon$ form a
cocycle, and determine a class $\xi' \in \check{H}^p(L, \mathcal{G})$
which maps to $\xi$ as desired.

\medskip\noindent
We omit the proof that if $\xi' \in \check{H}^{p + 1}_{HC}(X, \mathcal{F})$
maps to zero in $\check{H}^{p + 1}_{HC}(X, \mathcal{G})$, then it is
equal to $\delta(\xi)$ for some $\xi \in \check{H}^p_{HC}(X, \mathcal{H})$.
\end{proof}

\noindent
Next, we deduce Verdier's case of
Theorem \ref{theorem-cohomology-hypercoverings}
by a sleight of hand.

\begin{proposition}
\label{proposition-cohomology-hypercoverings}
Let $\mathcal{C}$ be a site with fibre products and products of pairs.
Let $\mathcal{F}$ be an abelian sheaf on $\mathcal{C}$.
Let $i \geq 0$. Then
\begin{enumerate}
\item for every $\xi \in H^i(\mathcal{F})$ there exists a hypercovering
$K$ such that $\xi$ is in the image of the canonical map
$\check{H}^i(K, \mathcal{F}) \to H^i(\mathcal{F})$, and
\item if $K, L$ are hypercoverings and $\xi_K \in \check{H}^i(K, \mathcal{F})$,
$\xi_L \in \check{H}^i(L, \mathcal{F})$ are elements mapping
to the same element of $H^i(\mathcal{F})$, then there exists
a hypercovering $M$ and morphisms $M \to K$ and $M \to L$ such
that $\xi_K$ and $\xi_L$ map to the same element of
$\check{H}^i(M, \mathcal{F})$.
\end{enumerate}
In other words, modulo set theoretical issues, the cohomology
groups of $\mathcal{F}$ on $\mathcal{C}$ are the colimit of
the {\v C}ech cohomology groups of $\mathcal{F}$ over all hypercoverings.
\end{proposition}

\begin{proof}
This result is a trivial consequence of
Theorem \ref{theorem-cohomology-hypercoverings}.
Namely, we can artificially replace $\mathcal{C}$ with a slightly
bigger site $\mathcal{C}'$ such that
(I) $\mathcal{C}'$ has a final object $X$ and (II)
hypercoverings in $\mathcal{C}$ are more or less the
same thing as hypercoverings of $X$ in $\mathcal{C}'$.
But due to the nature of things, there is quite a bit of
bookkeeping to do.

\medskip\noindent
Let us call a family of morphisms $\{U_i \to U\}$ in $\mathcal{C}$
with fixed target a {\it weak covering} if the sheafification of the
map $\prod_{i \in I} h_{U_i} \to h_U$ becomes surjective.
We construct a new site $\mathcal{C}'$ as follows
\begin{enumerate}
\item as a category set $\Ob(\mathcal{C}') = \Ob(\mathcal{C}) \amalg \{X\}$
and add a unique morphism to $X$ from every object of $\mathcal{C}'$,
\item $\mathcal{C}'$ has fibre products as fibre products and products
of pairs exist in $\mathcal{C}$,
\item coverings of $\mathcal{C}'$ are weak coverings of $\mathcal{C}$
together with those $\{U_i \to X\}_{i \in I}$ such that either $U_i = X$
for some $i$, or $U_i \not = X$ for all $i$ and the map
$\prod h_{U_i} \to *$ of presheaves on $\mathcal{C}$ becomes
surjective after sheafification on $\mathcal{C}$,
\item we apply Sets, Lemma \ref{sets-lemma-coverings-site}
to restrict the coverings to obtain our site $\mathcal{C}'$.
\end{enumerate}
Then $\Sh(\mathcal{C}') = \Sh(\mathcal{C})$ because the inclusion
functor $\mathcal{C} \to \mathcal{C}'$ is a special cocontinuous functor
(see Sites, Definition \ref{sites-definition-special-cocontinuous-functor}).
We omit the straightforward verifications.

\medskip\noindent
Choose a covering $\{U_i \to X\}$ of $\mathcal{C}'$ such that $U_i$ is an
object of $\mathcal{C}$ for all $i$ (possible because
$\mathcal{C} \to \mathcal{C}'$ is special cocontinuous).
Then $K_0 = \{U_i \to X\}$ is a covering in the
site $\mathcal{C}'$ constructed above. We view $K_0$ as an object of
$\text{SR}(\mathcal{C}', X)$ and we set $K_{init} = \text{cosk}_0(K_0)$.
Then $K_{init}$ is a hypercovering of $X$, see
Example \ref{example-cech}. Note that every $K_{init, n}$ has the shape
$\{W_j \to X\}$ with $W_j \in \Ob(\mathcal{C})$.

\medskip\noindent
Proof of (1). Choose $\xi \in H^i(\mathcal{F}) = H^i(X, \mathcal{F}')$
where $\mathcal{F}'$ is the abelian sheaf on $\mathcal{C}'$ corresponding
to $\mathcal{F}$ on $\mathcal{C}$. By
Theorem \ref{theorem-cohomology-hypercoverings}
there exists a morphism of hypercoverings $K' \to K_{init}$
of $X$ in $\mathcal{C}'$ such that $\xi$ comes from an element
of $\check{H}^i(K', \mathcal{F})$.
Write $K'_n = \{U_{n, j} \to X\}$. Now since $K'_n$ maps to
$K_{init, n}$ we see that $U_{n, j}$ is an object of $\mathcal{C}$.
Hence we can define a simplicial object $K$ of $\text{SR}(\mathcal{C})$
by setting $K_n = \{U_{n, j}\}$. Since coverings in
$\mathcal{C}'$ consisting of families of morphisms of $\mathcal{C}$
are weak coverings, we see that $K$ is a hypercovering in the sense
of Definition \ref{definition-hypercovering-variant}.
Finally, since $\mathcal{F}'$ is the unique sheaf on $\mathcal{C}'$
whose restriction to $\mathcal{C}$ is equal to $\mathcal{F}$
we see that the {\v C}ech complexes $s(\mathcal{F}(K))$
and $s(\mathcal{F}'(K'))$ are identical and (1) follows.
(Compatibility with map into cohomology groups omitted.)

\medskip\noindent
Proof of (2). Let $K$ and $L$ be hypercoverings in $\mathcal{C}$.
Let $K'$ and $L'$ be the simplicial objects of $\text{SR}(\mathcal{C}', X)$
gotten from $K$ and $L$ by the functor
$\text{SR}(\mathcal{C}) \to \text{SR}(\mathcal{C}', X)$,
$\{U_i\} \mapsto \{U_i \to X\}$. As before we have equality of
{\v C}ech complexes and hence we obtain $\xi_{K'}$ and
$\xi_{L'}$ mapping to the same cohomology class of $\mathcal{F}'$
over $\mathcal{C}'$. After possibly enlarging our choice
of coverings in $\mathcal{C}'$ (due to a set theoretical issue)
we may assume that $K'$ and $L'$ are hypercoverings of $X$ in
$\mathcal{C}'$; this is true by our definition of hypercoverings in
Definition \ref{definition-hypercovering-variant} and
the fact that weak coverings in $\mathcal{C}$ give coverings in
$\mathcal{C}'$. By
Theorem \ref{theorem-cohomology-hypercoverings}
there exists a hypercovering $M'$ of $X$ in $\mathcal{C}'$
and morphisms $M' \to K'$, $M' \to L'$, and $M' \to K_{init}$
such that $\xi_{K'}$ and $\xi_{L'}$ restrict to the same element of
$\check{H}^i(M', \mathcal{F})$. Unwinding this statement as above
we find that (2) is true.
\end{proof}





\section{Hypercoverings of spaces}
\label{section-hypercoverings-spaces}

\noindent
The theory above is mildly interesting even in the case of topological
spaces. In this case we can work out what a hypercovering is and see
what the result actually says.

\medskip\noindent
Let $X$ be a topological space. Consider the site $X_{Zar}$
of Sites, Example \ref{sites-example-site-topological}. Recall that
an object of $X_{Zar}$ is simply an open of $X$ and that morphisms
of $X_{Zar}$ correspond simply to inclusions. So what is a
hypercovering of $X$ for the site $X_{Zar}$?

\medskip\noindent
Let us first unwind Definition \ref{definition-SR}.
An object of $\text{SR}(X_{Zar}, X)$ is simply given by a set
$I$ and for each $i \in I$ an open $U_i \subset X$.
Let us denote this by $\{U_i\}_{i \in I}$ since there can be no
confusion about the morphism $U_i \to X$.
A morphism $\{U_i\}_{i \in I} \to \{V_j\}_{j \in J}$
between two such objects is given by a map of sets
$\alpha : I \to J$ such that $U_i \subset V_{\alpha(i)}$ for all
$i \in I$. When is such a morphism a covering? This is the case
if and only if for every $j \in J$ we have
$V_j = \bigcup_{i\in I, \ \alpha(i) = j} U_i$ (and is
a covering in the site $X_{Zar}$).

\medskip\noindent
Using the above we get the following description of a hypercovering
in the site $X_{Zar}$. A hypercovering of $X$ in $X_{Zar}$
is given by the following data
\begin{enumerate}
\item a simplicial set $I$ (see
Simplicial, Section \ref{simplicial-section-simplicial-set}), and
\item for each $n \geq 0$ and every $i \in I_n$ an open set $U_i \subset X$.
\end{enumerate}
We will denote such a collection of data by the notation $(I, \{U_i\})$.
In order for this to be a hypercovering of $X$ we require
the following properties
\begin{itemize}
\item for $i \in I_n$ and $0 \leq a \leq n$
we have $U_i \subset U_{d^n_a(i)}$,
\item for $i \in I_n$ and $0 \leq a \leq n$ we have $U_i = U_{s^n_a(i)}$,
\item we have
\begin{equation}
\label{equation-covering-X}
X = \bigcup\nolimits_{i \in I_0} U_i,
\end{equation}
\item for every $i_0, i_1 \in I_0$, we have
\begin{equation}
\label{equation-covering-two}
U_{i_0} \cap U_{i_1} =
\bigcup\nolimits_{i \in I_1, \ d^1_0(i) = i_0, \ d^1_1(i) = i_1} U_i,
\end{equation}
\item for every $n \geq 1$ and every
$(i_0, \ldots, i_{n + 1}) \in (I_n)^{n + 2}$ such that
$d^n_{b - 1}(i_a) = d^n_a(i_b)$ for all $0\leq a < b\leq n + 1$
we have
\begin{equation}
\label{equation-covering-general}
U_{i_0} \cap \ldots \cap U_{i_{n + 1}} =
\bigcup\nolimits_{i \in I_{n + 1},
\ d^{n + 1}_a(i) = i_a, \ a = 0, \ldots, n + 1} U_i,
\end{equation}
\item each of the open coverings (\ref{equation-covering-X}),
(\ref{equation-covering-two}), and (\ref{equation-covering-general})
is an element of $\text{Cov}(X_{Zar})$
(this is a set theoretic condition, bounding
the size of the index sets of the coverings).
\end{itemize}
Conditions (\ref{equation-covering-X}) and
(\ref{equation-covering-two}) should be familiar from the
chapter on sheaves on spaces for example, and condition
(\ref{equation-covering-general}) is the natural generalization.

\begin{remark}
\label{remark-not-covering-set}
One feature of this description is that if one of the multiple
intersections $U_{i_0} \cap \ldots \cap U_{i_{n + 1}}$ is empty then
the covering on the right hand side may be the empty covering.
Thus it is not automatically the case that the maps
$I_{n + 1} \to (\text{cosk}_n\text{sk}_n I)_{n + 1}$ are surjective.
This means that the geometric realization of $I$ may be an interesting
(non-contractible) space.

\medskip\noindent
In fact, let $I'_n \subset I_n$ be the subset
consisting of those simplices $i \in I_n$ such that
$U_i \not = \emptyset$. It is easy to see that $I' \subset I$
is a subsimplicial set, and that $(I', \{U_i\})$ is a hypercovering.
Hence we can always refine a hypercovering to a hypercovering where
none of the opens $U_i$ is empty.
\end{remark}

\begin{remark}
\label{remark-repackage-into-simplicial-space}
Let us repackage this information in yet another way.
Namely, suppose that $(I, \{U_i\})$ is a hypercovering of
the topological space $X$. Given this data we can construct
a simplicial topological space $U_\bullet$ by setting
$$
U_n = \coprod\nolimits_{i \in I_n} U_i,
$$
and where for given $\varphi : [n] \to [m]$ we let
morphisms $U(\varphi) : U_n \to U_m$ be the morphism
coming from the inclusions $U_i \subset U_{\varphi(i)}$
for $i \in I_n$. This simplicial topological space comes
with an augmentation $\epsilon : U_\bullet \to X$.
With this morphism the simplicial space $U_\bullet$ becomes
a hypercovering of $X$ along which one has cohomological descent
in the sense of \cite[Expos\'e Vbis]{SGA4}.
In other words, $H^n(U_\bullet, \epsilon^*\mathcal{F}) = H^n(X, \mathcal{F})$.
(Insert future reference here to cohomology over simplicial
spaces and cohomological descent formulated in those terms.)
Suppose that $\mathcal{F}$ is an abelian sheaf on $X$.
In this case the spectral sequence of Lemma \ref{lemma-cech-spectral-sequence}
becomes the spectral sequence with $E_1$-term
$$
E_1^{p, q} = H^q(U_p, \epsilon_q^*\mathcal{F})
\Rightarrow
H^{p + q}(U_\bullet, \epsilon^*\mathcal{F}) = H^{p + q}(X, \mathcal{F})
$$
comparing the total cohomology of $\epsilon^*\mathcal{F}$
to the cohomology groups of $\mathcal{F}$ over the pieces
of $U_\bullet$. (Insert future reference to this spectral sequence
here.)
\end{remark}

\noindent
In topology we often want to find hypercoverings of $X$ which
have the property that all the $U_i$ come from a given basis for the topology
of $X$ and that all the coverings
(\ref{equation-covering-two}) and (\ref{equation-covering-general})
are from a given cofinal collection of coverings.
Here are two example lemmas.

\begin{lemma}
\label{lemma-basis-hypercovering}
Let $X$ be a topological space.
Let $\mathcal{B}$ be a basis for the topology of $X$.
There exists a hypercovering $(I, \{U_i\})$ of $X$
such that each $U_i$ is an element of $\mathcal{B}$.
\end{lemma}

\begin{proof}
Let $n \geq 0$.
Let us say that an {\it $n$-truncated hypercovering} of $X$ is
given by an $n$-truncated simplicial set $I$ and for each
$i \in I_a$, $0 \leq a \leq n$ an open $U_i$ of $X$ such that
the conditions defining a hypercovering hold whenever they make sense.
In other words we require the inclusion relations and covering
conditions only when all simplices that occur in them
are $a$-simplices with $a \leq n$. The lemma follows if we can prove
that given a $n$-truncated hypercovering $(I, \{U_i\})$ with
all $U_i \in \mathcal{B}$ we can extend it to an $(n + 1)$-truncated
hypercovering without adding any $a$-simplices for $a \leq n$.
This we do as follows. First we consider the $(n + 1)$-truncated
simplicial set $I'$ defined by
$I' = \text{sk}_{n + 1}(\text{cosk}_n I)$.
Recall that
$$
I'_{n + 1} =
\left\{
\begin{matrix}
(i_0, \ldots, i_{n + 1}) \in (I_n)^{n + 2} \text{ such that}\\
d^n_{b - 1}(i_a) = d^n_a(i_b) \text{ for all }0\leq a < b\leq n + 1
\end{matrix}
\right\}
$$
If $i' \in I'_{n + 1}$ is degenerate, say $i' = s^n_a(i)$ then we set
$U_{i'} = U_i$ (this is forced on us anyway by the second condition).
We also set $J_{i'} = \{i'\}$ in this case.
If $i' \in I'_{n + 1}$ is nondegenerate, say
$i' = (i_0, \ldots, i_{n + 1})$, then we choose a set
$J_{i'}$ and an open covering
\begin{equation}
\label{equation-choose-covering}
U_{i_0} \cap \ldots \cap U_{i_{n + 1}} =
\bigcup\nolimits_{i \in J_{i'}} U_i,
\end{equation}
with $U_i \in \mathcal{B}$ for $i \in J_{i'}$.
Set
$$
I_{n + 1} = \coprod\nolimits_{i' \in I'_{n + 1}} J_{i'}
$$
There is a canonical map $\pi : I_{n + 1} \to I'_{n + 1}$ which is
a bijection over the set of degenerate simplices in $I'_{n + 1}$ by
construction.
For $i \in I_{n + 1}$ we define $d^{n + 1}_a(i) = d^{n + 1}_a(\pi(i))$.
For $i \in I_n$ we define $s^n_a(i) \in I_{n + 1}$ as the unique
simplex lying over the degenerate simplex $s^n_a(i) \in I'_{n + 1}$.
We omit the verification that this defines an $(n + 1)$-truncated
hypercovering of $X$.
\end{proof}

\begin{lemma}
\label{lemma-quasi-separated-quasi-compact-hypercovering}
Let $X$ be a topological space.
Let $\mathcal{B}$ be a basis for the topology of $X$.
Assume that
\begin{enumerate}
\item $X$ is quasi-compact,
\item each $U \in \mathcal{B}$ is quasi-compact open, and
\item the intersection of any two quasi-compact opens in
$X$ is quasi-compact.
\end{enumerate}
Then there exists a hypercovering $(I, \{U_i\})$ of $X$ with the
following properties
\begin{enumerate}
\item each $U_i$ is an element of the basis $\mathcal{B}$,
\item each of the $I_n$ is a finite set, and in particular
\item each of the coverings  (\ref{equation-covering-X}),
(\ref{equation-covering-two}), and (\ref{equation-covering-general})
is finite.
\end{enumerate}
\end{lemma}

\begin{proof}
This follows directly from the construction in the proof of
Lemma \ref{lemma-basis-hypercovering} if we choose finite coverings
by elements of $\mathcal{B}$ in (\ref{equation-choose-covering}).
Details omitted.
\end{proof}







\section{Constructing hypercoverings}
\label{section-hypercovering-sites}

\noindent
Let $\mathcal{C}$ be a site. In this section we will think of a
simplicial object of $\text{SR}(\mathcal{C})$ as follows.
As usual, we set $K_n = K([n])$ and we denote $K(\varphi) : K_n \to K_m$
the morphism associated to $\varphi : [m] \to [n]$.
We may write $K_n = \{U_{n, i}\}_{i \in I_n}$. For
$\varphi : [m] \to [n]$ the morphism $K(\varphi) : K_n \to K_m$
is given by a map $\alpha(\varphi) : I_n \to I_m$ and morphisms
$f_{\varphi, i} : U_{n, i} \to U_{m, \alpha(\varphi)(i)}$
for $i \in I_n$. The fact that $K$ is a simplicial object of
$\text{SR}(\mathcal{C})$ implies that $(I_n, \alpha(\varphi))$
is a simplicial set
and that $f_{\psi, \alpha(\varphi)(i)} \circ f_{\varphi, i} =
f_{\varphi \circ \psi, i}$ when $\psi : [l] \to [m]$.

\begin{lemma}
\label{lemma-split}
Let $\mathcal{C}$ be a site. Let $K$ be an $r$-truncated simplicial object
of $\text{SR}(\mathcal{C})$. The following are equivalent
\begin{enumerate}
\item $K$ is split (Simplicial, Definition \ref{simplicial-definition-split}),
\item $f_{\varphi, i} : U_{n, i} \to U_{m, \alpha(\varphi)(i)}$
is an isomorphism for $r \geq n \geq 0$,
$\varphi : [m] \to [n]$ surjective, $i \in I_n$, and
\item $f_{\sigma^n_j, i} : U_{n, i} \to U_{n + 1, \alpha(\sigma^n_j)(i)}$
is an isomorphism for $0 \leq j \leq n < r$, $i \in I_n$.
\end{enumerate}
The same holds for simplicial objects if in (2) and (3)
we set $r = \infty$.
\end{lemma}

\begin{proof}
The splitting of a simplicial set is unique and is given by
the nondegenerate indices $N(I_n)$ in each degree $n$, see
Simplicial, Lemma \ref{simplicial-lemma-splitting-simplicial-sets}.
The coproduct of two objects $\{U_i\}_{i \in I}$ and $\{U_j\}_{j \in J}$
of $\text{SR}(\mathcal{C})$ is given by $\{U_l\}_{l \in I \amalg J}$
with obvious notation. Hence a splitting of $K$ must be given by
$N(K_n) = \{U_i\}_{i \in N(I_n)}$. The equivalence of (1) and (2)
now follows by unwinding the definitions. The equivalence of (2)
and (3) follows from the fact that any surjection
$\varphi : [m] \to [n]$ is a composition of morphisms
$\sigma^k_j$ with $k = n, n + 1, \ldots, m - 1$.
\end{proof}

\begin{lemma}
\label{lemma-hypercovering-object}
Let $\mathcal{C}$ be a site with fibre products.
Let $\mathcal{B} \subset \Ob(\mathcal{C})$ be a subset.
Assume
\begin{enumerate}
\item any object $U$ of $\mathcal{C}$ has a covering
$\{U_j \to U\}_{j \in J}$ with $U_j \in \mathcal{B}$, and
\item if $\{U_j \to U\}_{j \in J}$ is a covering
with $U_j \in \mathcal{B}$ and $\{U' \to U\}$ is a morphism with
$U' \in \mathcal{B}$, then $\{U_j \to U\}_{j \in J} \amalg \{U' \to U\}$
is a covering.
\end{enumerate}
Then for any $X$ in $\mathcal{C}$ there is a hypercovering $K$
of $X$ such that $K_n = \{U_{n, i}\}_{i \in I_n}$
with $U_{n, i} \in \mathcal{B}$ for all $i \in I_n$.
\end{lemma}

\begin{proof}
A warmup for this proof is the proof of
Lemma \ref{lemma-basis-hypercovering} and
we encourage the reader to read that proof first.

\medskip\noindent
First we replace $\mathcal{C}$ by the site $\mathcal{C}/X$.
After doing so we may assume that $X$ is the final object
of $\mathcal{C}$ and that $\mathcal{C}$ has all finite limits
(Categories, Lemma \ref{categories-lemma-finite-limits-exist}).

\medskip\noindent
Let $n \geq 0$. Let us say that an
{\it $n$-truncated $\mathcal{B}$-hypercovering of $X$}
is given by an $n$-truncated simplicial object $K$
of $\text{SR}(\mathcal{C})$
such that for $i \in I_a$, $0 \leq a \leq n$
we have $U_{a, i} \in \mathcal{B}$ and such that
$K_0$ is a covering of $X$ and
$K_{a + 1} \to (\text{cosk}_a \text{sk}_a K)_{a + 1}$
for $a = 0, \ldots, n - 1$
is a covering as in Definition \ref{definition-covering-SR}.

\medskip\noindent
Since $X$ has a covering $\{U_{0, i} \to X\}_{i \in I_0}$
with $U_i \in \mathcal{B}$ by assumption, we get a $0$-truncated
$\mathcal{B}$-hypercovering of $X$. Observe that any $0$-truncated
$\mathcal{B}$-hypercovering of $X$ is split, see
Lemma \ref{lemma-split}.

\medskip\noindent
The lemma follows if we can prove for $n \geq 0$ that given a
split $n$-truncated $\mathcal{B}$-hypercovering $K$ of $X$
we can extend it to a
split $(n + 1)$-truncated $\mathcal{B}$-hypercovering of $X$.

\medskip\noindent
Construction of the extension. Consider the $(n + 1)$-truncated simplicial
object $K' = \text{sk}_{n + 1}(\text{cosk}_n K)$ of $\text{SR}(\mathcal{C})$.
Write
$$
K'_{n + 1} = \{U'_{n + 1, i}\}_{i \in I'_{n + 1}}
$$
Since $K = \text{sk}_n K'$ we have $K_a = K'_a$ for $0 \leq a \leq n$.
For every $i' \in I'_{n + 1}$ we choose a covering
\begin{equation}
\label{equation-choose-covering-B}
\{g_{n + 1, j} : U_{n + 1, j} \to U'_{n + 1, i'}\}_{j \in J_{i'}}
\end{equation}
with $U_{n + 1, j} \in \mathcal{B}$ for $j \in J_{i'}$.
This is possible by our assumption on $\mathcal{B}$ in the lemma.
For $0 \leq m \leq n$ denote $N_m \subset I_m$ the subset of
nondegenerate indices. We set
$$
I_{n + 1} =
\coprod\nolimits_{\varphi : [n + 1] \to [m]\text{ surjective, }0\leq m \leq n}
N_m \amalg
\coprod\nolimits_{i' \in I'_{n + 1}} J_{i'}
$$
For $j \in I_{n + 1}$ we set
$$
U_{n + 1, j} =
\left\{
\begin{matrix}
U_{m, i} & \text{if} &
j = (\varphi, i) & \text{where} & \varphi : [n + 1] \to [m], i \in N_m \\
U_{n + 1, j} & \text{if} &
j \in J_{i'} & \text{where} & i' \in I'_{n + 1}
\end{matrix}
\right.
$$
with obvious notation. We set $K_{n + 1} = \{U_{n + 1, j}\}_{j \in I_{n + 1}}$.
By construction $U_{n + 1, j}$ is an element
of $\mathcal{B}$ for all $j \in I_{n + 1}$. Let us define compatible
maps
$$
I_{n + 1} \to I'_{n + 1}
\quad\text{and}\quad
K_{n + 1} \to K'_{n + 1}
$$
Namely, the first map is given by
$(\varphi, i) \mapsto \alpha'(\varphi)(i)$ and
$(j \in J_{i'}) \mapsto i'$.
For the second map we use the morphisms
$$
f'_{\varphi, i} : U_{m, i} \to U'_{n + 1, \alpha'(\varphi)(i)}
\quad\text{and}\quad
g_{n + 1, j} : U_{n + 1, j} \to U'_{n + 1, i'}
$$
We claim the morphism
$$
K_{n + 1} \to K'_{n + 1} =
(\text{cosk}_n \text{sk}_n K')_{n + 1} =
(\text{cosk}_n K)_{n + 1}
$$
is a covering as in Definition \ref{definition-covering-SR}.
Namely, if $i' \in I'_{n + 1}$, then either $i'$ is nondegenerate
and the inverse image of $i'$ in $I_{n + 1}$ is equal to $J_{i'}$
and we get a covering of $U'_{n + 1, i'}$ by our choice
(\ref{equation-choose-covering-B}), or $i'$ is degenerate and
the inverse image of $i'$ in $I_{n + 1}$ is
$J_{i'} \amalg \{(\varphi, i)\}$ for a unique pair $(\varphi, i)$
and we get a covering by our choice (\ref{equation-choose-covering-B})
and assumption (2) of the lemma.

\medskip\noindent
To finish the proof we have to define the morphisms
$K(\varphi) : K_{n + 1} \to K_m$ corresponding to morphisms
$\varphi : [m] \to [n + 1]$, $0 \leq m \leq n$ and the morphisms
$K(\varphi) : K_m \to K_{n + 1}$ corresponding to morphisms
$\varphi : [n + 1] \to [m]$, $0 \leq m \leq n$
satisfying suitable composition relations.
For the first kind we use the composition
$$
K_{n + 1} \to K'_{n + 1} \xrightarrow{K'(\varphi)} K'_m = K_m
$$
to define $K(\varphi) : K_{n + 1} \to K_m$.
For the second kind, suppose given $\varphi : [n + 1] \to [m]$,
$0 \leq m \leq n$. We define the corresponding morphism
$K(\varphi) : K_m \to K_{n + 1}$ as follows:
\begin{enumerate}
\item for $i \in I_m$ there is a unique surjective map
$\psi : [m] \to [m_0]$ and a unique $i_0 \in I_{m_0}$ nondegenerate
such that $\alpha(\psi)(i_0) = i$\footnote{For example, if $i$ is
nondegenerate, then $m = m_0$ and $\psi = \text{id}_{[m]}$.},
\item we set $\varphi_0 = \psi_0 \circ \varphi : [n + 1] \to [m_0]$
and we map
$i \in I_m$ to $(\varphi_0, i_0) \in I_{n + 1}$, in other words,
$\alpha(\varphi)(i) = (\varphi_0, i_0)$, and
\item the morphism
$f_{\varphi, i} : U_{m, i} \to U_{n + 1, \alpha(\varphi)(i)} = U_{m_0, i_0}$
is the inverse of the isomorphism $f_{\psi, i_0} : U_{m_0, i_0} \to U_{m, i}$
(see Lemma \ref{lemma-split}).
\end{enumerate}
We omit the straightforward but cumbersome verification that this defines
a split $(n + 1)$-truncated $\mathcal{B}$-hypercovering of $X$
extending the given $n$-truncated one. In fact, everything is clear
from the above, except for the verification that the morphisms
$K(\varphi)$ compose correctly for all $\varphi : [a] \to [b]$
with $0 \leq a, b \leq n + 1$.
\end{proof}

\begin{lemma}
\label{lemma-hypercovering-site}
Let $\mathcal{C}$ be a site with equalizers and fibre products.
Let $\mathcal{B} \subset \Ob(\mathcal{C})$ be a subset. Assume
that any object of $\mathcal{C}$ has a covering
whose members are elements of $\mathcal{B}$.
Then there is a hypercovering $K$ such that
$K_n = \{U_i\}_{i \in I_n}$ with $U_i \in \mathcal{B}$
for all $i \in I_n$.
\end{lemma}

\begin{proof}
This proof is almost the same as the proof of
Lemma \ref{lemma-hypercovering-object}. We will
only explain the differences.

\medskip\noindent
Let $n \geq 1$. Let us say that an
{\it $n$-truncated $\mathcal{B}$-hypercovering}
is given by an $n$-truncated simplicial
object $K$ of $\text{SR}(\mathcal{C})$
such that for $i \in I_a$, $0 \leq a \leq n$
we have $U_{a, i} \in \mathcal{B}$ and such that
\begin{enumerate}
\item $F(K_0)^\# \to *$ is surjective,
\item $F(K_1)^\# \to F(K_0)^\# \times F(K_0)^\#$ is surjective,
\item $F(K_{a + 1})^\# \to F((\text{cosk}_a \text{sk}_a K)_{a + 1})^\#$
for $a = 1, \ldots, n - 1$ is surjective.
\end{enumerate}
We first explicitly construct a split $1$-truncated $\mathcal{B}$-hypercovering.

\medskip\noindent
Take $I_0 = \mathcal{B}$ and $K_0 = \{U\}_{U \in \mathcal{B}}$.
Then (1) holds by our assumption on $\mathcal{B}$. Set
$$
\Omega =
\{(U, V, W, a, b) \mid U, V, W \in \mathcal{B}, a : U \to V, b : U \to W\}
$$
Then we set $I_1 = I_0 \amalg \Omega$. For $i \in I_1$ we set
$U_{1, i} = U_{0, i}$ if $i \in I_0$ and $U_{1, i} = U$
if $i = (U, V, W, a, b) \in \Omega$. The map
$K(\sigma^0_0) : K_0 \to K_1$ corresponds to the
inclusion $\alpha(\sigma^0_0) : I_0 \to I_1$
and the identity $f_{\sigma^0_0, i} : U_{0, i} \to U_{1, i}$
on objects. The maps $K(\delta^1_0), K(\delta^1_1) : K_1 \to K_0$
correspond to the two maps $I_1 \to I_0$ which are the
identity on $I_0 \subset I_1$ and map $(U, V, W, a, b) \in \Omega \subset I_1$
to $V$, resp.\ $W$. The corresponding morphisms
$f_{\delta^1_0, i}, f_{\delta^1_1, i} : U_{1, i} \to U_{0, i}$ are
the identity if $i \in I_0$ and $a, b$ in case $i = (U, V, W, a, b) \in \Omega$.
The reason that (2) holds is that any section of
$F(K_0)^\# \times F(K_0)^\#$ over an object $U$ of $\mathcal{C}$
comes, after replacing $U$ by the members of a covering,
from a map $U \to F(K_0) \times F(K_0)$.
This in turn means we have $V, W \in \mathcal{B}$
and two morphisms $U \to V$ and $U \to W$. Further replacing
$U$ by the members of a covering we may assume $U \in \mathcal{B}$
as desired.

\medskip\noindent
The lemma follows if we can prove that given a split
$n$-truncated $\mathcal{B}$-hypercovering $K$ for $n \geq 1$
we can extend it to a split $(n + 1)$-truncated $\mathcal{B}$-hypercovering.
Here the argument proceeds exactly as in the proof of
Lemma \ref{lemma-hypercovering-object}.
We omit the precise details, except for the following comments.
First, we do not need assumption (2) in the proof of the current
lemma as we do not need the morphism
$K_{n + 1} \to (\text{cosk}_n K)_{n + 1}$ to be covering;
we only need it to induce a surjection on associated sheaves of sets
which follows from
Sites, Lemma \ref{sites-lemma-covering-surjective-after-sheafification}.
Second, the assumption that $\mathcal{C}$ has fibre products and equalizers
guarantees that $\text{SR}(\mathcal{C})$ has fibre products
and equalizers and $F$ commutes with these
(Lemma \ref{lemma-coprod-prod-SR}). This suffices
assure us the coskeleton functors used exist (see
Simplicial, Remark \ref{simplicial-remark-existence-cosk} and
Categories, Lemma \ref{categories-lemma-fibre-products-equalizers-exist}).
\end{proof}

\begin{lemma}
\label{lemma-hypercovering-morphism-sites}
Let $f : \mathcal{C} \to \mathcal{D}$ be a morphism of sites
given by the functor $u : \mathcal{D} \to \mathcal{C}$.
Assume $\mathcal{D}$ and $\mathcal{C}$ have equalizers and
fibre products and $u$ commutes with them.
If a simplicial object $K$ of $\text{SR}(\mathcal{D})$
is a hypercovering, then $u(K)$ is a hypercovering.
\end{lemma}

\begin{proof}
If we write $K_n = \{U_{n, i}\}_{i \in I_n}$ as in the introduction
to this section, then $u(K)$ is the object of $\text{SR}(\mathcal{C})$
given by $u(K_n) = \{u(U_i)\}_{i \in I_n}$.
By Sites, Lemma \ref{sites-lemma-pullback-representable-sheaf}
we have $f^{-1}h_U^\# = h_{u(U)}^\#$ for $U \in \Ob(\mathcal{D})$.
This means that $f^{-1}F(K_n)^\# = F(u(K_n))^\#$ for all $n$.
Let us check the conditions (1), (2), (3) for $u(K)$ to be a
hypercovering from Definition \ref{definition-hypercovering-variant}.
Since $f^{-1}$ is an exact functor, we find that
$$
F(u(K_0))^\# = f^{-1}F(K_0)^\# \to f^{-1}* = *
$$
is surjective as a pullback of a surjective map and we get (1).
Similarly,
$$
F(u(K_1))^\# = f^{-1}F(K_1)^\# \to
f^{-1} (F(K_0) \times F(K_0))^\# = F(u(K_0))^\# \times F(u(K_0))^\#
$$
is surjective as a pullback and we get (2). For condition (3),
in order to conclude by the same method it suffices if
$$
F((\text{cosk}_n \text{sk}_n u(K))_{n + 1})^\# =
f^{-1}F((\text{cosk}_n \text{sk}_n K)_{n + 1})^\#
$$
The above shows that $f^{-1}F(-) = F(u(-))$. Thus it suffices to show
that $u$ commutes with the limits used in defining
$(\text{cosk}_n \text{sk}_n K)_{n + 1}$ for $n \geq 1$.
By Simplicial, Remark \ref{simplicial-remark-existence-cosk}
these limits are finite connected limits and $u$ commutes with these
by assumption.
\end{proof}

\begin{lemma}
\label{lemma-hypercovering-continuous-functor}
Let $\mathcal{C}$, $\mathcal{D}$ be sites. Let
$u : \mathcal{D} \to \mathcal{C}$ be a continuous functor.
Assume $\mathcal{D}$ and $\mathcal{C}$ have fibre products
and $u$ commutes with them. Let $Y \in \mathcal{D}$ and
$K \in \text{SR}(\mathcal{D}, Y)$ a hypercovering of $Y$.
Then $u(K)$ is a hypercovering of $u(Y)$.
\end{lemma}

\begin{proof}
This is easier than the proof of Lemma \ref{lemma-hypercovering-morphism-sites}
because the notion of being a hypercovering of an object is stronger, see
Definitions \ref{definition-hypercovering} and \ref{definition-covering-SR}.
Namely, $u$ sends coverings to coverings by the definition of
a morphism of sites. It suffices to check $u$ commutes with the
limits used in defining
$(\text{cosk}_n \text{sk}_n K)_{n + 1}$ for $n \geq 1$.
This is clear because the induced functor
$\mathcal{D}/Y \to \mathcal{C}/X$ commutes with all finite limits
(and source and target have all finite limits by
Categories, Lemma \ref{categories-lemma-finite-limits-exist}).
\end{proof}

\begin{lemma}
\label{lemma-w-contractible}
Let $\mathcal{C}$ be a site. Let $\mathcal{B} \subset \Ob(\mathcal{C})$
be a subset. Assume
\begin{enumerate}
\item $\mathcal{C}$ has fibre products,
\item for all $X \in \Ob(\mathcal{C})$ there exists a finite covering
$\{U_i \to X\}_{i \in I}$ with $U_i \in \mathcal{B}$,
\item if $\{U_i \to X\}_{i \in I}$ is a finite covering with
$U_i \in \mathcal{B}$ and $U \to X$ is a morphism with $U \in \mathcal{B}$,
then $\{U_i \to X\}_{i \in I} \amalg \{U \to X\}$ is a covering.
\end{enumerate}
Then for every $X$ there exists a hypercovering $K$ of $X$
such that each $K_n = \{U_{n, i} \to X\}_{i \in I_n}$ with
$I_n$ finite and $U_{n, i} \in \mathcal{B}$.
\end{lemma}

\begin{proof}
This lemma is the analogue of
Lemma \ref{lemma-quasi-separated-quasi-compact-hypercovering}
for sites. To prove the lemma we follow exactly the proof of
Lemma \ref{lemma-hypercovering-object}
paying attention to the following two points
\begin{enumerate}
\item[(a)] We choose our initial covering $\{U_{0, i} \to X\}_{i \in I_0}$
with $U_{0, i} \in \mathcal{B}$ such that the index set $I_0$ is finite, and
\item[(b)] in choosing the coverings
(\ref{equation-choose-covering-B})
we choose $J_{i'}$ finite.
\end{enumerate}
The reader sees easily that with these modifications we end up
with finite index sets $I_n$ for all $n$.
\end{proof}

\begin{remark}
\label{remark-taking-disjoint-unions}
Let $\mathcal{C}$ be a site. Let
$K$ and $L$ be objects of $\text{SR}(\mathcal{C})$.
Write $K = \{U_i\}_{i \in I}$ and $L = \{V_j\}_{j \in J}$.
Assume $U = \coprod_{i \in I} U_i$ and $V = \coprod_{j \in J} V_j$
exist. Then we get
$$
\Mor_{\text{SR}(\mathcal{C})}(K, L) \longrightarrow \Mor_\mathcal{C}(U, V)
$$
as follows. Given $f : K \to L$ given by $\alpha : I \to J$
and $f_i : U_i \to V_{\alpha(i)}$ we obtain a transformation of functors
$$
\Mor_\mathcal{C}(V, -) =
\prod\nolimits_{j \in J} \Mor_\mathcal{C}(V_j, -)
\to
\prod\nolimits_{i \in I} \Mor_\mathcal{C}(U_i, -) =
\Mor_\mathcal{C}(U, -)
$$
sending $(g_j)_{j \in J}$ to
$(g_{\alpha(i)} \circ f_i)_{i \in I}$. Hence the Yoneda lemma
produces the corresponding map $U \to V$. Of course, $U \to V$
maps the summand $U_i$ into the summand $V_{\alpha(i)}$ via
the morphism $f_i$.
\end{remark}

\begin{remark}
\label{remark-take-unions-hypercovering}
Let $\mathcal{C}$ be a site. Assume $\mathcal{C}$ has
fibre products and equalizers and let $K$ be a hypercovering.
Write $K_n = \{U_{n, i}\}_{i \in I_n}$. Suppose that
\begin{enumerate}
\item[(a)] $U_n = \coprod_{i \in I_n} U_{n, i}$ exists, and
\item[(b)] $\coprod_{i \in I_n} h_{U_{n, i}} \to h_{U_n}$ induces
an isomorphism on sheafifications.
\end{enumerate}
Then we get another simplicial object $L$ of $\text{SR}(\mathcal{C})$
with $L_n = \{U_n\}$, see
Remark \ref{remark-taking-disjoint-unions}.
Now we claim that $L$ is a hypercovering.
To see this we check conditions (1), (2), (3) of
Definition \ref{definition-hypercovering-variant}.
Condition (1) follows from (b) and (1) for $K$.
Condition (2) follows in exactly the same way.
Condition (3) follows because
\begin{align*}
F((\text{cosk}_n \text{sk}_n L)_{n + 1})^\#
& =
((\text{cosk}_n \text{sk}_n F(L)^\#)_{n + 1}) \\
& =
((\text{cosk}_n \text{sk}_n F(K)^\#)_{n + 1}) \\
& =
F((\text{cosk}_n \text{sk}_n K)_{n + 1})^\#
\end{align*}
for $n \geq 1$ and hence the condition for $K$ implies the condition for
$L$ exactly as in (1) and (2).
Note that $F$ commutes with connected limits and sheafification is exact
proving the first and last equality; the middle equality follows as
$F(K)^\# = F(L)^\#$ by (b).
\end{remark}

\begin{remark}
\label{remark-take-unions-hypercovering-X}
Let $\mathcal{C}$ be a site. Let $X \in \Ob(\mathcal{C})$.
Assume $\mathcal{C}$ has fibre products and let $K$ be a hypercovering of $X$.
Write $K_n = \{U_{n, i}\}_{i \in I_n}$. Suppose that
\begin{enumerate}
\item[(a)] $U_n = \coprod_{i \in I_n} U_{n, i}$ exists,
\item[(b)] given morphisms
$(\alpha, f_i) : \{U_i\}_{i \in I} \to \{V_j\}_{j \in J}$ and
$(\beta, g_k) : \{W_k\}_{k \in K} \to \{V_j\}_{j \in J}$
in $\text{SR}(\mathcal{C})$ such that
$U = \coprod U_i$, $V = \coprod V_j$, and $W = \coprod W_j$
exist, then $U \times_V W =
\coprod_{(i, j, k), \alpha(i) = j = \beta(k)} U_i \times_{V_i} W_k$,
\item[(c)] if $(\alpha, f_i) : \{U_i\}_{i \in I} \to \{V_j\}_{j \in J}$
is a covering in the sense of
Definition \ref{definition-covering-SR}
and $U = \coprod U_i$ and $V = \coprod V_j$ exist,
then the corresponding morphism $U \to V$
of Remark \ref{remark-taking-disjoint-unions}
is a covering of $\mathcal{C}$.
\end{enumerate}
Then we get another simplicial object $L$ of $\text{SR}(\mathcal{C})$
with $L_n = \{U_n\}$, see
Remark \ref{remark-taking-disjoint-unions}.
Now we claim that $L$ is a hypercovering of $X$.
To see this we check conditions (1), (2) of
Definition \ref{definition-hypercovering}.
Condition (1) follows from (c) and (1) for $K$
because (1) for $K$ says $K_0 = \{U_{0, i}\}_{i \in I_0}$
is a covering of $\{X\}$ in the sense of
Definition \ref{definition-covering-SR}.
Condition (2) follows because $\mathcal{C}/X$ has
all finite limits hence $\text{SR}(\mathcal{C}/X)$
has all finite limits, and condition (b) says the
construction of ``taking disjoint unions'' commutes
with these fimite limits. Thus the morphism
$$
L_{n + 1} \longrightarrow (\text{cosk}_n \text{sk}_n L)_{n + 1}
$$
is a covering as it is the consequence of applying our
``taking disjoint unions'' functor to the morphism
$$
K_{n + 1} \longrightarrow (\text{cosk}_n \text{sk}_n K)_{n + 1}
$$
which is assumed to be a covering in the sense of
Definition \ref{definition-covering-SR} by (2) for $K$.
This makes sense because property (b) in particular assures
us that if we start with a finite diagram of
semi-representable objects over $X$
for which we can take disjoint unions, then
the limit of the diagram in $\text{SR}(\mathcal{C}/X)$
still is a semi-representable object over $X$ for which
we can take disjoint unions.
\end{remark}





\input{chapters}

\bibliography{my}
\bibliographystyle{amsalpha}

\end{document}
